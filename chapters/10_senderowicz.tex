\documentclass[output=paper,colorlinks,citecolor=brown
% ,hidelinks
% showindex
]{langscibook}
\author{Yaron M. Senderowicz\orcid{}\affiliation{Tel-Aviv University}}

\title{The first-person perspective and second-order desires}
\abstract{In this paper I argue that persons who entertain non-instrumental second-order desires—desires to desire something—must have the deliberative first-order desires they wish to have. The necessary connection between these types of desires consists on the fact that the subjects that have them are not self-blind, i.e., if they have the second-order desires, they must be consciously self-acquainted with them. I demonstrate that since the difference between the second-order and first-order deliberative desires does not concern their intentional content, it is merely a verbal difference. The upshot of this discussion is that it is not really possible to recursively generate a set of non-instrumental deliberative desires. This significant cognitive constraint is implied by the role of first-person self-acquaintance within rational cognition.

I first present the characteristics of non-instrumental second-order desires. I then explain why entertaining a desire that belongs to this type necessarily implies that the person who has it must have the first-order deliberative desire she wish to have. I continue by illustrating the cognitive difference between the first-person and third-person ascription of deliberative second-order desires. I conclude by singling out the differences and the connections between deliberative desires and spontaneous desires that have the same content.  
}


\IfFileExists{../localcommands.tex}{
   \addbibresource{../localbibliography.bib}
   \newcommand{\orcid}[1]{}

\usepackage{orcidlink}

\usepackage{tabularx,multicol}
\usepackage{url}
\urlstyle{same}


\usepackage{langsci-optional}
\usepackage{langsci-lgr}
\usepackage{langsci-gb4e}

% for texlive 2022
\usepackage{langsci-branding} 

% Müller


% \usepackage{biblatex-series-number-checks}

% \usepackage{eng-date}

% \usepackage{german}%% Das Buch ist nicht deutsch. Hör auf, solche Sachen zu laden


% \usepackage{tikz-dependency}
% \usepackage{tikz}
% \usepackage{tikz-qtree}

% \usepackage{hologo}

% 3_pullum.tex

% This does not work complains about recommanding epsilon
% We use a special adapted version, which is in the repro.
\usepackage{langsci-textipa}



% 8_levine

\usepackage{./styles/lg-macro2}
\usepackage{bm}
\usepackage{umoline}
\usepackage{pifont}
\usepackage{pstricks,pst-node,pst-tree}
\usepackage{ulem}
\usepackage{mathrsfs}
\usepackage{bussproofs}




%\usepackage{tikz,tikz-qtree}

%\usepackage{gb4e0}
%\noautomath

%\usepackage[letterpaper,margin=1.2in]{geometry}



% 14_kornai

%\usepackage{xypic} % seems not to be needed
% \usepackage[matrix,arrow]{xy}
%\usepackage{amsmath}
% \usepackage{subcaption}
% \usepackage{wrapfig}


   
\SetupAffiliations{output in groups = false,
                   orcid placement = after,
                   separator between two = {\bigskip\\},
                   separator between multiple = {\bigskip\\},
                   separator between final two = {\bigskip\\}
                   }

% ORCIDs in langsci-affiliations 
\definecolor{orcidlogocol}{cmyk}{0,0,0,1}
\RenewDocumentCommand{\LinkToORCIDinAffiliations}{ +m }
  {%
    \,\orcidlink{#1}%
  }


\makeatletter
\let\thetitle\@title
\let\theauthor\@author
\makeatother

\newcommand{\togglepaper}[1][0]{
   \bibliography{../localbibliography}
   \papernote{\scriptsize\normalfont
     \theauthor.
     \titleTemp.
     To appear in:
     E. Di Tor \& Herr Rausgeberin (ed.).
     Booktitle in localcommands.tex.
     Berlin: Language Science Press. [preliminary page numbering]
   }
   \pagenumbering{roman}
   \setcounter{chapter}{#1}
   \addtocounter{chapter}{-1}
}

\newbool{bookcompile}
\booltrue{bookcompile}
\newcommand{\bookorchapter}[2]{\ifbool{bookcompile}{#1}{#2}}


% Cite and cross-reference other chapters
\newcommand{\crossrefchaptert}[2][]{\citet*[#1]{chapters/#2}, Chapter~\ref{chap-#2} of this volume} 
\newcommand{\crossrefchapterp}[2][]{(\citealp*[#1]{chapters/#2}, Chapter~\ref{chap-#2} of this volume)}
\newcommand{\crossrefchapteralt}[2][]{\citealt*[#1]{chapters/#2}, Chapter~\ref{chap-#2} of this volume}
\newcommand{\crossrefchapteralp}[2][]{\citealp*[#1]{chapters/#2}, Chapter~\ref{chap-#2} of this volume}

\newcommand{\crossrefcitet}[2][]{\citet*[#1]{chapters/#2}} 
\newcommand{\crossrefcitep}[2][]{\citep*[#1]{chapters/#2}}
\newcommand{\crossrefcitealt}[2][]{\citealt*[#1]{chapters/#2}}
\newcommand{\crossrefcitealp}[2][]{\citealp*[#1]{chapters/#2}}


\newcommand{\sub}[1]{\textsubscript{\scriptsize\textrm{#1}}}

% Müller

\newcommand{\page}{}

\let\citew\citet

\def\underRevision{Revise and resubmit}

\let\textbfemph\emph

\newcommand{\todostefan}[1]{\todo[color=orange!80]{\footnotesize #1}\xspace}
\newcommand{\todosatz}[1]{\todo[color=red!40]{\footnotesize #1}\xspace}

\newcommand{\inlinetodostefan}[1]{\todo[color=green!40,inline]{\footnotesize #1}\xspace}

\newcommand{\inlinetodoopt}[1]{\todo[color=green!40,inline]{\footnotesize #1}\xspace}
\newcommand{\inlinetodoobl}[1]{\todo[color=red!40,inline]{\footnotesize #1}\xspace}

\newcommand{\itd}[1]{\inlinetodoobl{#1}}
\newcommand{\itdobl}[1]{\inlinetodoobl{#1}}
\newcommand{\itdopt}[1]{\inlinetodoopt{#1}}

\newcommand{\addpages}{\todostefan{add pages}}

%% % taken from https://tex.stackexchange.com/a/95079/18561
\newbox\usefulbox

\makeatletter
\def\getslant #1{\strip@pt\fontdimen1 #1}

\def\skoverline #1{\mathchoice
 {{\setbox\usefulbox=\hbox{$\m@th\displaystyle #1$}%
    \dimen@ \getslant\the\textfont\symletters \ht\usefulbox
    \divide\dimen@ \tw@ 
    \kern\dimen@ 
    \overline{\kern-\dimen@ \box\usefulbox\kern\dimen@ }\kern-\dimen@ }}
 {{\setbox\usefulbox=\hbox{$\m@th\textstyle #1$}%
    \dimen@ \getslant\the\textfont\symletters \ht\usefulbox
    \divide\dimen@ \tw@ 
    \kern\dimen@ 
    \overline{\kern-\dimen@ \box\usefulbox\kern\dimen@ }\kern-\dimen@ }}
 {{\setbox\usefulbox=\hbox{$\m@th\scriptstyle #1$}%
    \dimen@ \getslant\the\scriptfont\symletters \ht\usefulbox
    \divide\dimen@ \tw@ 
    \kern\dimen@ 
    \overline{\kern-\dimen@ \box\usefulbox\kern\dimen@ }\kern-\dimen@ }}
 {{\setbox\usefulbox=\hbox{$\m@th\scriptscriptstyle #1$}%
    \dimen@ \getslant\the\scriptscriptfont\symletters \ht\usefulbox
    \divide\dimen@ \tw@ 
    \kern\dimen@ 
    \overline{\kern-\dimen@ \box\usefulbox\kern\dimen@ }\kern-\dimen@ }}%
 {}}
\makeatother

% 1_intro.tex

% For the block quote:

\usepackage[most]{tcolorbox}
\definecolor{linequote}{RGB}{224,215,188}
\definecolor{backquote}{RGB}{249,245,233}
\newtcolorbox{myquote}[1][]{%
    enhanced, breakable, 
    size=minimal,
    frame hidden, boxrule=0pt,
    sharp corners,
    colback=backquote,
    #1
}

% 2_gibson.tex


% Example(s) Environments
% 12pt, No new-lines after example number is printed

\newcounter{examplectr}
\newcounter{fnexamplectr}

% Note: don't use subexamples in footnotes.

% This line is to overcome a bug in cmu-art style: it prints counter
% values to the aux file using \theaux... rather than using \the...
\def\theauxexamplectr{\theexamplectr}

\newcounter{subexamplectr}
\def\theauxsubexamplectr{\thesubexamplectr}
\def\theauxfnexamplectr{\thefnexamplectr}

\renewcommand{\theexamplectr}{\arabic{examplectr}}
% This command causes example numbers to appear without following periods

\renewcommand{\thefnexamplectr}{\roman{fnexamplectr}}
% This command causes example numbers to appear without following periods

\renewcommand{\thesubexamplectr}{\theexamplectr\alph{subexamplectr}}
% This command gives the number of an example and subexample as e.g. 1a, 2b

\newlength{\wdth}
\newcommand{\strike}[1]{\settowidth{\wdth}{#1}\rlap{\rule[.5ex]{\wdth}{1pt}}#1}

\newcommand{\exref}[1]{(\ref{#1})}
% This command puts reference numbers with parentheses
% surrounding them 

% The environment ``examples'' gives a list of examples, one on each line,
% numbered with a lower case alphabetic character
\newenvironment{examples}%
   { \vspace{-\baselineskip}
     \begin{list}%
     \textrm{\alph{subexamplectr}.}%
     {\usecounter{subexamplectr}
     \setlength{\topsep}{-\parskip}
     \setlength{\itemsep}{-2pt}
     \setlength{\leftmargin}{0.5in}
     \setlength{\rightmargin}{0in} } }%
   { \end{list}}

% The environment ``myexample'' outputs an arabic counter ``examplectr''
% surrounded by parentheses.
\newenvironment{myexample}
   { \vspace{20pt}
     \noindent
     \begin{minipage}{\textwidth}    % minipage environment disallows
                 % breaks across pages

     \refstepcounter{examplectr}     % step the counter and cause this
                 % section to be referenced by the
                 % counter ``examplectr''
     (\arabic{examplectr})}%
   { \vspace{20pt}
     \end{minipage}}

\newenvironment{myfnexample}
   { \vspace{2pt}
     \noindent
     \begin{minipage}{\textwidth}    % minipage environment disallows
                 % breaks across pages

     \refstepcounter{fnexamplectr}     % step the counter and cause this
                 % section to be referenced by the
                 % counter ``examplectr''
     (\roman{fnexamplectr})}%
   { \vspace{2pt}c
     \end{minipage}}
    
\newcommand*\circled[1]{\tikz[baseline=(char.base)]{
            \node[shape=circle,draw,inner sep=2pt] (char) {#1};}}



% 3_pullum.tex


% %%%  GKP:  I put in these pointless commands to kill off a bug elsewhere
% %%%        that tries to \newcommand \it (etc.) as \itshape (etc.), but
% %%%        fails because they haven't been defined.
% \newcommand{\it}{\relax}
% \newcommand{\bf}{\relax}
% \newcommand{\sc}{\relax}
% \newcommand{\rm}{\relax}

%%% GKP:  Two additional commands that I need
\newcommand{\data}[1]{\textit{#1}}
\newcommand{\blank}{\rule{1.2em}{0.5pt}}



% 8_levine

% ???
%\renewcommand{\emph}{\textit}
%\renewcommand{\em}{\it}

% not used \newcommand{\cites}[1]{\citeauthor{#1}'s~\citeyearpar{#1}}

% \renewcommand{\SetInfLen}{\setpremisesend{0pt}\setpremisesspace{10pt}\setnamespace{0pt}}

\newcommand{\pt}[1]{\ensuremath{\mathsf{#1}}}
\newcommand{\ptv}[1]{\ensuremath{\textsf{\textsl{#1}}}}

%\newcommand{\sv}[1]{\ensuremath{\bm{\mathcal{#1}}}}
\newcommand{\sv}[1]{\ensuremath{\mathcal{#1}}}

\newcommand{\sX}{\sv{X}}
\newcommand{\sF}{\sv{F}}
\newcommand{\sG}{\sv{G}}
%
% \renewcommand{\lex}{\SF}
% \renewcommand{\syncat}[1]{\ensuremath{\mathrm{#1}}}
% \newcommand{\syncatVar}[1]{\ensuremath{\mathit{#1}}}
%
% \newcommand{\RuleName}[1]{\textrm{#1}}
%
% \newcommand{\SemTyp}{\textsf{Sem}}
%
% \newcommand{\something}{\vdots\,\,\,\,\,\,\vdots}
%
% \newcommand{\pb}{\phantom{[}}
%
% \renewcommand{\E}{\ensuremath{\bm{\epsilon}}\xspace}
%
% \newcommand{\greeka}{\upalpha}
% \newcommand{\greekb}{\upbeta}
% \newcommand{\greekd}{\updelta}
\newcommand{\greekp}{\upvarphi}
\newcommand{\greekr}{\uprho}
\newcommand{\greeks}{\upsigma}
% \newcommand{\greekt}{\uptau}
% \newcommand{\greeko}{\upomega}
% \newcommand{\greekz}{\upzeta}
%
% % Do not do this!!!!!
% % \renewcommand{\labelenumi}{(\roman{enumi})}
%
% \newcommand{\upa}[2]{\ensuremath{\syncat{#1}|\syncat{#2}}}
% \newcommand{\dna}[2]{\upa{#2}{#1}}
%

%
% \newcommand{\Lemma}{\ensuremath{\vdots\hskip.5cm\vdots}\noLine}
%
% \newcommand{\la}{\ensuremath{\langle}}
% \newcommand{\ra}{\ensuremath{\rangle}}
%
% \renewcommand{\I}{\iota}
%
% \renewcommand{\sem}{\ensuremath}
%
% \newcommand{\LemmaShort}{\ensuremath{\vdots\hskip.2cm\vdots\hskip.2cm\vdots}\noLine}
% \newcommand{\LemmaShortAlt}{\ensuremath{\vdots\hskip.2cm\vdots}}
%
%
% \newcommand{\NoSem}{%
% \renewcommand{\LexEnt}[3]{##1; \syncat{##3}}
% \renewcommand{\LexEntTwoLine}[3]{\renewcommand{\arraystretch}{.8}%
% \begin{array}[b]{l} ##1;  \\ \syncat{##3} \end{array}}
% \renewcommand{\LexEntThreeLine}[3]{\renewcommand{\arraystretch}{.8}%
% \begin{array}[b]{l} ##1; \\ \syncat{##3} \end{array}}}
%
%
% \newcommand{\NoSemVar}{%
% \renewcommand{\LexEnt}[3]{##1; \syncat{##3}}
% \renewcommand{\LexEntTwoLine}[3]{\renewcommand{\arraystretch}{.8}%
% \begin{array}{l} ##1;  \\ \syncat{##3} \end{array}}
% \renewcommand{\LexEntThreeLine}[3]{\renewcommand{\arraystretch}{.8}%
% \begin{array}{l} ##1; \\ \syncat{##3} \end{array}}}
%
% \newcommand{\vs}{\raisebox{.05em}{\ensuremath{\upharpoonright}}}
%
% \newcommand{\AXX}[1]{\raisebox{-7mm}{\ensuremath{#1}}}
%
% \newcommand{\hypml}[2]{\left[\!\!#1\!\!\right]^{#2}}
%
% \newcommand{\alt}[2]{$\left\{\begin{array}{c}
% \hskip-.7ex\textrm{#1}\hskip-.7ex \\
% \hskip-.7ex\textrm{#2}\hskip-.7ex
%         \end{array}
% \right\}$}
%
% \newcommand{\altalt}[2]{\{#1/#2\}}
%
%
%
% \newcommand{\altt}[3]{$\left\{\begin{array}{c}
% \hskip-.7ex\textrm{#1}\hskip-.7ex \\
% \hskip-.7ex\textrm{#2}\hskip-.7ex \\
% \hskip-.7ex\textrm{#3}\hskip-.7ex
% \end{array}
% \right\}$}
%
% \newcommand{\alttt}[4]{$\left\{\begin{array}{c}
% \hskip-.7ex\textrm{#1}\hskip-.7ex \\
% \hskip-.7ex\textrm{#2}\hskip-.7ex \\
% \hskip-.7ex\textrm{#3}\hskip-.7ex \\
% \hskip-.7ex\textrm{#4}\hskip-.7ex
% \end{array}
% \right\}$}
%
%
% %%%%for bussproof
%
% \def\defaultHypSeparation{\hskip0.1in}
% \def\ScoreOverhang{0pt}
%
%
% %%\newcommand{\MultiLine}[1]{\renewcommand{\arraystretch}{.8}%
% %%\ensuremath{\begin{array}{l} #1 \end{array}}}
%
\newcommand{\MultiLine}[1]{\renewcommand{\arraystretch}{.8}%
\ensuremath{\begin{array}[b]{l} #1 \end{array}}}

%
%
% \newcommand{\MultiLineMod}[1]{%
% \ensuremath{\begin{array}[t]{l} #1 \end{array}}}
%
%
% %%%%%\AFourMargin
% %%\JLSubmissionMargin
%
% %%\setlength\topmargin{-1cm}
% %%\setlength\textheight{23cm}
% %%%%%\setlength\textwidth{13.5cm}
%
% %\setstretch{1.2}
%
% % not used \newcommand{\hyp}[2]{[ #]^{#2}}
%
\newcommand{\LexEnt}[3]{#1; \ensuremath{#2}; \syncat{#3}}
%
% \newcommand{\LexEntTwoLine}[3]{\renewcommand{\arraystretch}{.8}%
% \begin{array}[b]{l} #1; \\ \ensuremath{#2};  \syncat{#3} \end{array}}
%
% \newcommand{\LexEntThreeLine}[3]{\renewcommand{\arraystretch}{.8}%
% \begin{array}[b]{l} #1; \\ \ensuremath{#2}; \\ \syncat{#3} \end{array}}
%
%
% \newcommand{\LexEntFiveLine}[5]{\renewcommand{\arraystretch}{.8}%
% \begin{array}{l} #1 \\ #2; \\ \ensuremath{#3} \\ \ensuremath{#4}; \\ \syncat{#5} \end{array}}
%
%
% \newcommand{\LexEntFourLine}[4]{\renewcommand{\arraystretch}{.8}%
% \begin{array}{l} \pt{#1} \\ \pt{#2}; \\ \syncat{#4} \end{array}}
%
% \newcommand{\ManySomething}{\renewcommand{\arraystretch}{.8}%
% \raisebox{-3mm}{\begin{array}[b]{c} \vdots \,\,\,\,\,\, \vdots \\
% \vdots \,\,\,\,\,\, \vdots \end{array}}}
%
%
% \newcommand{\lemma}[1]{\renewcommand{\arraystretch}{.8}%
% \begin{array}[b]{c} \vdots \,\,\,\,\,\, \vdots \\ #1 \end{array}}
%
% \newcommand{\lemmarev}[1]{\renewcommand{\arraystretch}{.8}%
% \begin{array}[b]{c} #1 \\ \vdots \,\,\,\,\,\, \vdots \end{array}}
%
% \newcommand{\p}{\ensuremath{\upvarphi}}
%
% \newcommand{\Not}{\leavevmode\llap{\textbf{\smc{NOT:}} }}
%
% \newcommand{\Conj}{\fs{\bsp{\mathit{X}}{\mathit{X}}}{\mathit{X}}}
% \newcommand{\ConjY}{\fs{\bsp{\mathit{Y}}{\mathit{Y}}}{\mathit{Y}}}
% \newcommand{\sameLE}{\dna{(\upa{(\upa{S}{\mathit{X}})}{NP})}{(\upa{S}{\mathit{X}})}}
%
% \newcommand{\derivcenter}[2][1.1]{%
% \SetInfLen
% \attop{\vskip3ex
% \resizebox{#1\linewidth}{!}{\hskip-#1in
% #2}}}
%
% \newcommand{\derivcenterAlt}[2][.98]{%
% \SetInfLen
% \attop{\vskip3ex
% \resizebox{#1\linewidth}{!}{\hskip-#1in \hskip.5in
% #2}}\vspace{.5ex}}
%
% \renewcommand{\O}{\circ}  Do not recommand!
\newcommand{\BobsO}{\circ}
%
% \newcommand{\derivcenterMod}[2][1.1]{%
% \renewcommand{\LexEntThreeLine}[3]{\renewcommand{\arraystretch}{.8}%
% \raisebox{.4ex}{\ensuremath{\begin{array}{l} ##1; \\ \ensuremath{##2}; \\ \syncat{##3} \end{array}}}}
% \SetInfLen
% \attop{\vskip3ex
% \resizebox{#1\linewidth}{!}{\hskip-#1in
% #2}}}
%
%
% \newcommand{\shortarrow}{\xspace\hskip-1.2ex\scalebox{.5}[1]{\ensuremath{\bm{\rightarrow}}}\hskip-.5ex\xspace}
%
% \newcommand{\SemInt}[1]{\mbox{$[\![ \textrm{#1} ]\!]$}}
%
% \def\maru#1{{\ooalign{\hfil
%   \ifnum#1>999 \resizebox{.25\width}{\height}{#1}\else%
%   \ifnum#1>99 \resizebox{.33\width}{\height}{#1}\else%
%   \ifnum#1>9 \resizebox{.5\width}{\height}{#1}\else #1%
%   \fi\fi\fi%
% \/\hfil\crcr%
% \raise.167ex\hbox{\mathhexbox20D}}}}
%
\newcommand{\HypSpace}{\hskip-.8ex}
\newcommand{\RaiseHeight}{\raisebox{2.2ex}}
% \newcommand{\RaiseHeightLess}{\raisebox{1ex}}
%
% \newcommand{\fW}{\ensuremath{\mathfrak{W}}}
%
\newcommand{\ThreeColHyp}[1]{\RaiseHeight{\Bigg[}\HypSpace#1\HypSpace\RaiseHeight{\Bigg]}}
% \newcommand{\TwoColHyp}[1]{\RaiseHeightLess{\Big[}\HypSpace#1\HypSpace\RaiseHeightLess{\Big]}}
%
%
% %\newcommand{\maskref}[1]{\textsl{\textbf{[reference omitted for refereeing]}}}
% \newcommand{\maskref}[1]{#1}
%
% \newcommand{\DerivSize}{\small}
% \newcommand{\AppDerivSize}{\footnotesize}
%
% \renewcommand{\sem}{\ensuremath}


% \newcommand{\greekp}{{\color{green}π}}
% \newcommand{\greekr}{{\color{green}\textrho}}
% \newcommand{\greeks}{{\color{green}\textsigma}}
\newcommand{\ptfont}[1]{\texttt{#1}}                % what does ptfont do? Where is it defined?
                                % Question
%\newcommand{\ptfont}{\ttfamily}

%\newcommand{\grey}{\color{gray}}
\newcommand{\grey}[1]{\colorbox{mycolor}{#1}}
\definecolor{mycolor}{gray}{0.8}

\newcommand{\gap}{\longrule}
\newcommand{\gp}{\gap}
\newcommand{\vs}{\raisebox{.05em}{\ensuremath{\upharpoonright}}}
% \newcommand{\sub}[1]{\textsubscript[#1]}
\newcommand{\E}{\emph}
\newcommand{\B}{\textbf}
\newcommand{\f}{{\color{green}f}}  % Question what does f do? It does not have any output in the
                                % original PDF
%\newcommand{\Lemma}{{\color{pink}Lemma}}
\newcommand{\Lemma}{\ensuremath{\vdots\hskip.5cm\vdots}\noLine}

%\newcommand{\calP}{{\color{pink}calP}} % Sebastian
\newcommand{\calP}{\ensuremath{\mathcal{P}}}


\newcommand{\maru}[1]{\ooalign{\hfil#1\/\hfil\crcr
      \raise.05ex\hbox{\LARGE\mathhexbox20D}}}


%\newcommand{\sem}[2][M\!,g]{\mbox{$[\![ \mathrm{#2} ]\!]^{#1}$}}
\newcommand{\sem}{\ensuremath}

%
\newcommand{\trns}[1]{\textbf{#1}\xspace}

\newcommand{\bs}{{\textbackslash}}
\newcommand{\bsl}{{\bs}}


\newcommand{\fb}[1]{\textsubscript{#1}}

\newcommand{\syncat}[1]{\ensuremath{\mathrm{#1}}}
\newcommand{\term}[1]{\textit{#1}}
\newcommand{\LemmaAlt}{\ensuremath{\vdots\hskip.5cm\vdots}}

%\renewcommand{\O}{ø}

   %% hyphenation points for line breaks
%% Normally, automatic hyphenation in LaTeX is very good
%% If a word is mis-hyphenated, add it to this file
%%
%% add information to TeX file before \begin{document} with:
%% %% hyphenation points for line breaks
%% Normally, automatic hyphenation in LaTeX is very good
%% If a word is mis-hyphenated, add it to this file
%%
%% add information to TeX file before \begin{document} with:
%% %% hyphenation points for line breaks
%% Normally, automatic hyphenation in LaTeX is very good
%% If a word is mis-hyphenated, add it to this file
%%
%% add information to TeX file before \begin{document} with:
%% \include{localhyphenation}
\hyphenation{
    par-a-digm
}

\hyphenation{
    par-a-digm
}

\hyphenation{
    par-a-digm
}

   \boolfalse{bookcompile}
   \togglepaper[23]%%chapternumber
}{}

\begin{document}
\maketitle

\section{Introduction}

A second-order desire is a desire that the person having it will desire that p be the case or that she will act in a certain way. The formation of second-order desires involves uses of the first-person pronoun or its mental analogue. First-person uses of “I” serve indispensable roles in human cognition \citep{castaneda19661999,castaneda1967indicators,castaneda1968logic,perry1979problem,lewis1979attitudes,chisholm1981first}.\footnote{Perry pointed out the cognitive value of the first-person use of “I” which functions as an essential indexical. According to Lewis, the intentional content of a mental state consist in the self-attribution of properties to oneself. Chisholm explicates the content of intentional states in a similar way. However, these accounts do not address the ways in which the rational constraints related to first-person self-acquaintance determine the necessary connection between some mental states.} According to \citet{burge1998reason}, the indispensability of the first-person concept to rational cognition consists among other things in the fact that reasons must sometimes be rationally applicable to immediately affect an attitude or an action.\footnote{In what follows, I clarify why the feature pointed out by Burge is linked to the rational constraints related to first-person self-acquaintance.} According to \citet{shoemaker1996first}, first-person self-acquaintance determines a fundamental characteristic of rational mental states. As he notes, considerations related to Moore's paradox \citep{moore1903refutation} require one to acknowledge the fact that rational human creatures cannot be self-blind. A self-blind creature can know his mental states only in the third-person mode of knowledge \citep[pp.~30-31]{shoemaker1996first}. According to \citet[p.~31]{shoemaker1996first},

\begin{quote}
    To deny the possibility of self-blindness is to hold that it is implicit in the nature of certain mental states that any  subject of such states that has the capacity to conceive of  itself as having them will be aware of having them when it does, or at least will become aware of this under certain conditions (e.g., if it reflects on the matter).
\end{quote}


That self-blindness is not possible implies imperative constraints relevant to the cognitive and epistemic architecture of rational beings.

Shoemaker’s main concern is the impossibility of self blindness with respect to the concept of belief. My goal in this paper is to point out the fundamental roll of first-person self-acquaintance with respect to another kind of mental state—non-instrumental, second-order desires. First-person self-acquaintance with this type of second-order desires entails that the person having them necessarily has the relevant deliberative first-order desires. The third-personal attribution of deliberative second-order desires does not involve a similar kind of necessary connection.\footnote{As will be clarified in what follows, my argument for the necessary connection between the relevant second-order and first-order desires does not depend on Shoemaker's argument.}
	
	In the sections below, I first briefly state the commonly accepted features of desires and similar pro-attitudes. I then continue by clarifying the nature of deliberative non-instrumental second-order desires. Finally, I explain why realizing the relevant second-order desire entails that the same person who has the second-order desire must have the desired first-order desire. As I demonstrate below, the upshot of this discussion is that it is not really possible to recursively generate a set of deliberative desires. This seems to be a fact about human cognition. For the difference between the second-order deliberative desire and the first-order deliberative desire does not concern their intentional content; it is merely a verbal difference.
	
\section{The features of desires}

Desires are pro-attitudes that have at least the following features:

\begin{itemize}
\item[(a)]	The content of a desire determines its conditions of satisfaction. 

\item[(b)]	Desires are individuated by the subject having them and by their content.

\item[(c)]	A person may have conflicting desires, that is, desires that cannot be satisfied together. 

\item[(d)]	Desires move the persons having them to aspire for their satisfaction.
\end{itemize}

The content of a desire represents what one desires. The content of a particular desire is only one part of its individuating conditions. Two desires with the same content are not identical qua particular mental states if different persons have them. 
Condition (d) requires some clarification. In contrast to beliefs, desires do not represent how things are but how one wishes them to be. Aiming at being satisfied is a necessary (reflexive) characteristic of desires. It cannot consist in having a separate desire [S]Dx the content of which is that Dx be satisfied. For given that [S]Dx also aims at being satisfied, if aiming that Dx be satisfied consists in a different desire [S]Dx, each desire must be connected to an infinite set of separate desires. Since this supposition is incoherent, aiming at being satisfied must be an intrinsic (reflexive) constituent of desires. 

	Aiming at being satisfied is equivalent to being moved to act in ways that satisfy the content of the desire. However, an agent may have conflicting desires Dx and Dy, desire that cannot be satisfied together. Therefore, being moved to act in ways that satisfy the content of a given desire Dx does not entail that the agent will eventually do something in order to satisfy Dx. She could choose to satisfy a conflicting desire Dy. However, this is irrelevant to the fact that desires intrinsically aim at their own satisfaction, i.e., that aiming at being satisfied is a necessary constituent of desires. Desires, all kinds if desires, move the agent having them to pursue the realization of what their content represents.
	
In what follows I will assume that (a)-(d) are sufficient conditions for characterizing a mental state as a desire. There is, however, an additional condition that seems to be part of the concept of desire:

\begin{itemize}
\item[(e)]	Desires do not necessitate the realization of the states of affairs that satisfy them.
\end{itemize}

There is a strong presumption to the effect that both first- and second-order desires must satisfy condition (e) above. Yet, as I demonstrate below, at least one type of second-order desires does not satisfy (e). 

\section{Features of deliberative second-order desires}

I begin by briefly explicating the nature of non-instrumental desires, in order to single out the relevant kind of second-order desire. 
An instrumental desire is a desire the satisfaction of which is required in order to attain something else one desires. The satisfaction of a non-instrumental desire is conceived as being good in itself. The person having it does not desire what she desires for the sake of something else.

Deliberative desires are desires persons have upon a process of deliberation. After such a process, a person may choose to realize a given state of affair or to carry out a given course of action. Being moved to realize a given state of affairs upon a process of deliberation satisfies the sufficient conditions of desires noted above. 

Deliberative desires differ from spontaneous desires. The starving person's desire for food, the caring mother's desire to assist her suffering infant, the aroused person's desire for sexual intercourse, and the desire to distance oneself from precarious situations are all examples of spontaneous desires.

Some spontaneous desires are animalistic or instinctual. However, spontaneous desires could also have been acquired in a process of training or education. 

An irrational desire is a desire one \textit{knows} to consist of an impossible state of affairs and which would therefore be unsatisfiable. In what follows I assume that deliberative desires are \textit{rational} desires.

Finally, second-order desires are \textit{conscious} desires. A person having a second-order desire consciously desires that she herself desire that p be the case. She reflectively knows that she has the second-order desire. Second-order desires are deliberative desires. Although some of them are instrumental, some are desired for their own sake. For example, a person addicted to heroin has the \textit{spontaneous} desire to take the drug. She may also have the \textit{deliberative} second-order wish to desire to refrain from taking it. The desire she wishes to realize—to refrain from taking the drug—could be something she conceives to be good in itself and not merely for the sake of something else she desires.

\section{Difference of content and separate existence}

Second-order desires seem to be states that unanimously satisfy condition (e) above. Consider first the connection between the individuation of a particular desire and its conditions of satisfaction. Let Dp be Jill’s desire that her mother will be separated from her vehement, abusive father, and let DDp be Jill’s wish to desire that her mother will be separated from her abusive father. Dp and DDp do not have the same content, for the state of affairs that satisfies DDp does not entail the state of affairs that satisfies Dp, and the state of affairs that satisfies Dp does not entail the state of affairs that satisfies DDp. The state of affairs that satisfies Dp is that Jill’s mother will be separated from her abusive father, and the state of affairs that consists of Jill's desire that her mother will be separated from her abusive father does not entail that Jill’s mother will be separated from her abusive father. Jill can desire that her mother will be separated from her abusive father, even if Jill’s mother is not separated from her abusive father. Similarly, the state of affairs that satisfies Dp does not entail the state of affairs that satisfies DDp. Even if Jill’s mother is separated from Jill’s abusive father at some future time, this does not necessitate the existence of Jill's desire that her mother will be separated from her abusive father.

According to condition (e), entertaining a desire Dp does not entail that the state of affairs that satisfies it is realized. The fact that the state of affairs that consists of \textit{having} Dp and the state of affairs that \textit{satisfies} Dp are different states of affairs seems to provide a substantial reason for the unrestricted application of condition (e). S having the second-order desire DDp and S having the first-order desire Dp seem to be two distinct states of affairs. It seems that as far as their \textit{objective content} is concerned, there is no reason to claim that if a deliberative second-order desire DDp is realized by S, S must also realize Dp. It seems that deliberative second-order desires and the first-order desires that are their objects could exist separately. As \citet{hume2000treatise} (book I, part III, sect III) already claimed, the distinguishability of the content of two mental states entails their separate existence.

The last claim seems to be supported also by the following consideration. It is evidently possible to rationally desire that \textit{someone else} desire to perform an action X, or that she desires that p be the case. For example, after considerable contemplation, Jill may conclude that her mother should be separated from her abusive father. Say that Jill's mother does not want to be separated from her husband. Upon deliberation, Jill forms the desire that her mother will desire to be separated from her abusive father. Clearly, her desire that her mother will desire to be separated from her abusive father does not entail that her mother desires this. But if one of Jill’s mother first-order desires could be the object of Jill’s deliberative desire without entailing the existence of Jill’s mother’s pertinent desire, why can't a similar desire—a desire that has the same objective content—be the object of one of Jill's second-order desires without entailing the existence of her relevant first-order desire? In this case, the difference between the respective content of the first-order and second-order desires also seems to entail their separate existence. 

\section{Self-knowledge and deliberative second-order desires}

\citet{frankfurt1988importance} distinguished between two situations that, in his view, may be described by “A wants to want to X”. In one of these situations, the fact that A wants to want to X does not entail that A wants to X. The case of the psychotherapist who wants to be moved by the desire to take the drug to better understand his narcotic addict patients, but without desiring to X, exemplifies the first situation. It should be noted that the case of the psychotherapist is clearly one of an instrumental desire. The desire to take the drug is not desired for its own sake, but to provide a better understanding of narcotic addicts. The other situation that could be described by “A wants to want to X” is a situation in which according to \citet[p.~15]{frankfurt1988importance}, A wants the desire to X “to provide the motive for what he actually does”. As Frankfurt notes (ibid.), in this case “A wants to want to X” entails that “A already has the desire to X”. Frankfurt's unwilling addict—the narcotic addict that desires to desire to refrain from taking the drug—exemplifies the second situation described by “A wants to want to X”. 
It should be noted, however, that in Frankfurt theory the concept of the will is not identical to the concept of desire. In his view, the will is the desire that leads all the way to action. My concern here is with cases similar to Frankfurt's unwilling addict.  I wish to clarify why in this type of case “A wants to want to X” entails that A wants to X.     

Let DDx represent a deliberative non-instrumental, second-order desire, and let Dx represent the desired desire. Does the fact that a given subject S has DDx entail that S has Dx? As I wish to suggest now, there are reasons to claim that if a given person has DDx, she must also have Dx.
Consider first the following statements:

\begin{enumerate}
\item[1.]\label{item-jill-one}	Jill has the deliberative second-order desire to desire to refrain from smoking.\footnote{I assume that Jill's desire to desire to refrain from smoking is a deliberative non-instrumental second-order desire.}

\item[2.]\label{item-jill-two}	Jill has no first-order deliberative desire to refrain from smoking. 
\end{enumerate}

On first approximation it seems that (1) and (2) are compatible. Yet, since Jill is first-personally conscious that she desires to desire to refrain from smoking, she must know what is involved in her desire to desire to refrain from smoking. She must know the content of her desire, and she must be consciously moved to realize what the content of her desire represents. The following statement must therefore be added: 

\begin{itemize}
\item[3.]\label{item-jill-three}	Jill knows the content of her deliberative wish that she herself desire to refrain from smoking, and she is consciously moved to realize what the content of her second-order desire represents. 
\end{itemize}
\itdopt{label mechaism does not work}

Are \REF{item-jill-one}--\REF{item-jill-three} compatible together? It is indisputable, I suppose, that having the second-order \textit{deliberative} desire to desire to refrain from smoking does not entail that one has the first-order \textit{spontaneous} desire to refrain from smoking. The question is whether entertaining the relevant second-order desire entail having the deliberative first-order desire to refrain from smoking. It can be shown that if the desire is a deliberative first-order desire, \REF{item-jill-one}--\REF{item-jill-three} do not cohere together. In order to see why, consider the two sets of statements below:

\begin{itemize}
\item[4.]\label{item-jill-four}	Jill desires to drink water.

\item[5.]\label{item-jill5}	Jill knows that water is not the same liquid as oil. 

\item[6.]\label{item-jill-six}	Jill does not desire to drink oil.
\end{itemize}
\noindent
and

\begin{itemize}
\item[7.]\label{item-jill-seven}	Jill desires to drink water.

\item[8.]\label{item-jill8}	Jill knows that water = H\textsubscript{2}O, i.e., that it contains hydrogen.

\item[9.]\label{item-jill-nine}	Jill does not wish to drink a portion of liquid containing hydrogen.
\end{itemize}

Statements \REF{item-jill-four}--\REF{item-jill-six} cohere together. However, if Jill is a rational person \REF{item-jill-seven}--\REF{item-jill-nine} is not a coherent set. If Jill knows that the water she wishes to drink is a liquid that necessarily contains hydrogen, her desire to drink water is rationally bound to the desire to drink liquid containing hydrogen. She cannot rationally desire to drink water and desire not to drink a liquid that contains hydrogen. 

Is \REF{item-jill-one}--\REF{item-jill-three} similar to \REF{item-jill-four}--\REF{item-jill-six} or to \REF{item-jill-seven}--\REF{item-jill-nine}? It can be shown that \REF{item-jill-one}--\REF{item-jill-three} is as incoherent as \REF{item-jill-seven}--\REF{item-jill-nine}.
We may begin by noting that the claim that Jill could rationally entertain the non-instrumental second-order wish to desire to refrain from smoking without having the deliberative first-order desire to refrain from smoking in fact means that she could be \textit{moved} to have the first-order non-instrumental desire to refrain from smoking while being \textit{indifferent} as to whether she herself will refrain from smoking. This means that although Jill wishes to be in a state in which she herself is moved to realize the act of refraining from smoking, a state that she conceives to be good in itself, she is nevertheless \textit{not moved} to refrains from smoking. But is it possible for Jill to rationally desire to be in a state in which she is moved to refrain from smoking, qua something desired for its own sake, while being indifferent as to whether she refrains from smoking? The claim that this is possible in fact means that Jill could be self-consciously and rationally moved to be in a state which as she knows aims and being satisfied and is satisfied only if the person that has it refrains from smoking, and nevertheless, her being in this state does not involve her being moved to refrain from smoking. However, this is incoherent. It seems that Jill could deliberatively wish to desire to refrain from smoking qua something she conceives to be good in itself only if she conceives \textit{the act of refraining from smoking} to be a good act. However, given that a rational person is moved to realize what she conceives to be good, this characterization is in fact equivalent to depicting Jill as being moved not to refraining from smoking.

We can tackle this issue also from a different viewpoint by raising the following question: What could \textit{explain} Jill’s conscious wish to desire not to smoke? Since Jill’s second-order desire is not an instrumental desire, and since it is a rational self-conscious desire, the only thing that is able to explain this is her being moved not to smoke. Stated differently, Jill could not self-consciously and rationally be moved to desire not to smoke as something desired for its own sake without being moved to realize what the desire she wishes to have is about. For she must know that it is not possible to desire not to smoke without being moved to realize a state of not smoking. In other words, by being rationally moved to be moved to refrain from smoking she must be self-consciously moved to realize what the desired first-order desire is about. She must be moved to refrain from smoking; she must have the first-order deliberative desire she wishes to have.

Jill could desire something and nevertheless fail to satisfy her desire. She might also desire something and nevertheless act in ways that do not aim to satisfy it, if she prefers to satisfy a conflicting desire, or due to the weakness of her will. But the idea that Jill could \textit{fail to be moved} to refrain from smoking even if she is rationally moved to desire this qua something desired for its own sake is erroneous. Although she could fail to refrain from smoking, by being rationally moved to desire what she desires, Jill must be rationally moved to realize what the content of the desire she wishes to have represents. She must have the first-order deliberative desire that she desires to have.

\section{First-person self-acquaintance and deliberative second-order desires}

What could be the grounds that motivate the supposition that it is possible for Jill to deliberatively desire that she herself desire to refrain from smoking even if she is not moved to refrain from smoking? I suggest, that there are at least two such grounds. First, one is motivated to hold this supposition if one disregards the crucial role of first-person self-acquaintance in deliberative second-order desires. Second, this view seems to unjustifiably identify the mere concept of desire with that of \textit{spontaneous} desires. It disregards the fact that deliberative desires are pro-attitudes similar to spontaneous desires, and that deliberative desires could have the same \textit{intentional content} as a given spontaneous desire, although they differ from them qua mental state. 

In this section, I address the first ground and in the next section I address the second ground. 

It is, I suppose, undeniable that Jill could entertain the deliberative desire that her husband Ben desire to refrain from smoking, even if Ben is not moved to refrain from smoking. There is no reason to suppose that Ben is necessarily moved to refrain from smoking just because his wife desires this. Moreover, Ben could even know that his wife wishes that he will desire to refrain from smoking, and, nevertheless, he will not desire to refrain from smoking. The fact that he is left indifferent to what Jill desires does not indicate that he has any irrational desire that cannot be satisfied. Rather, he should be described as someone who has failed to implement Jill's reasons and to be rationally moved to refrain from smoking. In the same vein, let us presume, for the sake of argument, that it is possible to third-personally attribute a second-order desire to a person by means of some technique of mind reading that is based on some sort of perceptual evidence. If Jill were self-blind, she could have third-personally established in this way that “this” person wishes to desire to refrain from smoking. In addition, she could have established in a similar way that “this” person is identical to a given person that does not desire to refrain from smoking, and that “Jill” refers to “this” person. Let us presume for the sake of argument that this possibility to attribute a second-order desire to a person on the basis of third-personal, perceptual information without attributing to her the corresponding first-order desire is coherent. It seems that this type of case does not differ from Jill’s deliberative wish that her husband Ben will refrain from smoking. Nevertheless, Jill’s knowledge that a given person that deliberatively wishes to desire to refrain from smoking is identical to the person that does not wish to refrain from smoking differs from the self-knowledge that her utterance: “I wish to desire to refrain from smoking, but I do not desire to refrain from smoking” expresses. If Jill is self-blind, she will fail to know that “I am Jill”, i.e., that she is the person that wishes to desire this, the person that her use of “I” designates. As in the case of her deliberative wish that her husband Ben desire to refrain from smoking, in this case too she may fail to be immediately moved to refrain from smoking, even if she grasps the reasons that motivate her deliberative second-order desire. However, this is not the case if Jill knows that she herself is the person that wishes to desire not to smoke, that is, if  she says or thinks “I wish to desire not to smoke”. As \citet{perry1979problem} explained, reference to oneself by means of the indexical expression “I” cognitively differs from reference to oneself by means of other types of singular expressions.  But if Jill is first-personally conscious that she herself non-instrumentally wishes to desire to refrain from smoking, her wish is irrational if she is moved to desire this while deliberatively remaining indifferent as to whether she will refrain from smoking. Consciously and rationally desiring that a given state of affairs be realized is identical to being moved to realize it. Since desires necessarily move the person that have them to satisfy them, it is irrational for Jill to self-consciously desire to entertain a cognitive state that is satisfied only if she refrains from smoking—a state desired for its own sake—while being indifferent to her refraining to smoke.

According to \citet{burge1998reason} “The first-person concept fixes the locus of responsibility and marks the immediate rational relevance of a rational evaluation to rational implementation on the attitude being evaluated—to epistemic or practical agency” (p. 253). Deliberative second-order desires exemplify the fact that reasons must sometimes be applicable to affect an attitude or an action. Also in the case of deliberative second-order desires, the first-person concept and its role in rational agency is involved in determining the cognitive structure of rational agents.   




\section{Deliberative and spontaneous desires}
    
Spontaneous desires have animalistic or instinctual character. One does not choose to have them. In contrast to that, deliberative desires are based on reason and reflection; they are generated by means of reflective deliberation, i.e., by examining how a given course of action is related to what one conceives to be good. So far, I claimed that a person who has the deliberative wish to desire something must have the corresponding \textit{deliberative} first-order desire that she wishes to have. However, this does not rule out the possibility of deliberatively wishing to spontaneously desire that p be the case or to perform an action X without spontaneously desiring this. A person may have a deliberative second-order desire without having the relevant first-order spontaneous desire. She could fail to \textit{spontaneously} instantiate the first-order desire she deliberately wishes to have. If there are some courses of action that one can carry out only if one spontaneously desires to carry them out, it seems that a person could have a deliberative second-order desire without having the (spontaneous) first-order desire that satisfies it.

It should be noted, however, that this possibility could be regarded as a counter example to my main claim here only if the person that has the deliberative second-order desire does not entertain the corresponding deliberative first-order desire. Yet, as I clarify in what follows, the paradigmatic examples of persons that deliberatively wish to spontaneously desire to act in a certain way and nevertheless do not spontaneously desire this are not cases in which the person at stake does not have the pertinent \textit{deliberative} first-order desires. 

Consider the following example: Jane and Ralph have been married for fifteen years. Jane remembers having been physically attracted to her husband. However, although Ralph continues to be physically attracted to Jane, Jane now does not spontaneously desire to have any sort of intimate physical relations with her husband. She loves him deeply and has great respect for him. She also finds him interesting and amusing and likes the life they share, but does not spontaneously desire to have intimate physical relations with him. When she now sees him expectantly waiting for her, she remorsefully thinks to herself: “How I wish I desired to have sex with Ralph!”        
Let us suppose that Jane's wish to desire to have intimate relations with her beloved husband is a non-instrumental desire. She desires this because having intimate physical relations with Ralph is part of what Jane conceives to be good in itself. But it seems that in her current state Jane does not desire to have intimate physical relations with her husband, even though she deliberatively desires to spontaneously desire this. Hence, it seems that she could first-personally self-ascribe the rational wish to desire to have sexual intercourse with her husband even though she does not desire to have intimate physical relations with him. 

In order to examine what is involved in this objection, it should be first noted that although deliberative desires differ from spontaneous desires, they could have the same content. A person may spontaneously desire not to smoke and deliberatively desire not to smoke. Spontaneously desiring not to smoke differs, qua mental state, from deliberatively desiring not to smoke. Each of these mental states has a different phenomenological character—they do not "feel" the same. Nevertheless, they could be together the states of one and the same person. The fact that the deliberative and the spontaneous desires are distinct mental states explains why a person can fail to spontaneously instantiate the desire she deliberatively wishes to have. But must she have the corresponding deliberative first-order desire, \textit{even if} she fails to spontaneously desire what she wishes to desire? 

As noted earlier, wishing to desire that a given state of affairs will be realized while remaining utterly indifferent as to whether what its content represents is realized is as irrational as consciously desiring to drink water, knowing that water necessarily contain hydrogen, without desiring to drink a liquid containing hydrogen. It seems to follow that if one is rationally motivated to desire a given desire, one is necessarily moved to satisfy its content. In addition, if the second-order desire is a non-instrumental one, the agent considers the realization of what its content represents to be something that is good in itself. An agent could wish to spontaneously desire to realize a state of a affairs represented by a given content, and she could fail to be spontaneously moved to realize it. But if her wish to spontaneously desire something is a non-instrumental desire, she must also conceive the object of the desire she wishes to have as something that is good in itself. The goodness of what she desires does not depend on how she is moved to realize its content. The fact that she fails to be spontaneously moved to realize the content of the desire she wishes to have therefore does not entail that she is not \textit{rationally motivated} to realize its content. On the contrary, if her deliberative second-order desire is non-instrumental, she must be rationally motivated to realize the content of the first-order spontaneous desire she wishes to have, even if she is not spontaneously motivated to realize it. 

For example, Jane fails to realize the spontaneous first-order desire to have sexual intercourse with her husband. But does she also fail to have the corresponding first-order \textit{deliberative} desire, if she has the rational non-instrumental desire to desire this? Say that a psychiatrist offered Jane a harmless drug that would enable her to be physically, spontaneously, attracted to her husband whenever the occasion arose. Would she be motivated to take the drug? I suppose that if she deliberatively wishes to desire to have intimate physical relations with Ralph as part of her motivation to realize and enhance her love for Ralph, she would be motivated to take the drug, even if the conflicting feeling of indifference persisted in her. Moreover, even if she is overpowered by her spontaneous feeling of indifference and ultimately does not take the drug, she is rationally moved to take it. But she is moved to take the drug because she is moved to realize a process of maintaining voluntary and pleasing intimate physical relations with Ralph. In other words, Jane's mental state exemplifies the constitutive properties of a deliberative first-order wish to have intimate physical relations with Ralph. If Jane's deliberative wish to desire to have intimate physical relations with Ralph moves her to take the drug, she must also be ascribed the deliberative desire to have intimate physical relations with him. 
Although there are, on the one hand, manifest similarities between Jane's past spontaneous attraction to her husband Ralph and her deliberative preference to have voluntary and pleasing intimate physical relations with him, there are also manifest differences between the two states. Jane's former physical attraction to Ralph was a spontaneous state that she did not choose to have. She simply had it. In contrast, her current interest in having pleasing intimate physical relations with Ralph consists in the reasons that would move her to take the drug. Nevertheless, if she has the non-instrumental deliberative second-order desire to spontaneously desire to have sexual intercourse with Ralph, she must have a deliberative first-order desire with the same content. 

In view of the above, the case of the indifferent wife and similar cases are therefore not cases in which the person who deliberatively wishes to spontaneously desire something does not have a relevant deliberative first-order desire. 

\section{Conclusion}

We are naturally constituted to spontaneously desire certain kinds of goals. We do not choose to have the spontaneous desires that we do in fact have, and we choose not to have some of the spontaneous desires that we have. But we also possess the capacity for rational deliberation that could move us to pursue the realization of certain goals. Jill has not chosen to experience her current spontaneous desire to smoke. But the claim that her deliberative desire to refrain from smoking is not based on her rational capacity to choose is mistaken. Although her spontaneous desires to smoke or to refrain from smoking cannot be depicted as states she \textit{deliberately} realizes, she could certainly be depicted as a person who deliberatively prefers to refrain from smoking. However, it is incoherent to suppose that a person who self-consciously prefers to desire not to smoke qua something desired for its own sake could fail to realize the deliberative wish to refrain from smoking. No similar incoherence characterizes the case in which a person third-personally ascribes to an agent a desire that someone will desire something. 

The difference between the first- and third-person ascriptions of the relevant second-order desires therefore indicates the limits of the third-personal relation to oneself. It reveals one sense in which first-person self-acquaintance has an indispensable role in rational cognition and rational conduct.   



\printbibliography[heading=subbibliography,notkeyword=this]
\end{document}

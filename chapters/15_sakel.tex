\documentclass[output=paper,colorlinks,citecolor=brown
% ,hidelinks
% showindex
]{langscibook}

\author{Jeanette Sakel\orcid{}\affiliation{University of the West of England}}

\title[Investigating grammatical borrowing in Mosetén]{Investigating grammatical borrowing in Mosetén through historical sources}

\abstract{Using historical sources (1804—1913), this paper investigates the influence of Spanish on the grammar of modern Mosetén, an indigenous language spoken in the foothills of the Bolivian Andes and the adjoining Amazon basin. Focusing on the categories of gender agreement and phrasal word order, I argue that modern Mosetén gender agreement follows Spanish patterns, while word order rules are in part affected by intensive language contact with Spanish. Speaker variation, as observed in modern Mosetén, appears to be present in the historical data already. Yet, changes in use-patterns and frequencies may be the reason for the extension of some grammatical categories, meaning modern Mosetén grammar is closer to Spanish than the language observed in the original historical sources.}


\IfFileExists{../localcommands.tex}{
   \addbibresource{../localbibliography.bib}
   \newcommand{\orcid}[1]{}

\usepackage{orcidlink}

\usepackage{tabularx,multicol}
\usepackage{url}
\urlstyle{same}


\usepackage{langsci-optional}
\usepackage{langsci-lgr}
\usepackage{langsci-gb4e}

% for texlive 2022
\usepackage{langsci-branding} 

% Müller


% \usepackage{biblatex-series-number-checks}

% \usepackage{eng-date}

% \usepackage{german}%% Das Buch ist nicht deutsch. Hör auf, solche Sachen zu laden


% \usepackage{tikz-dependency}
% \usepackage{tikz}
% \usepackage{tikz-qtree}

% \usepackage{hologo}

% 3_pullum.tex

% This does not work complains about recommanding epsilon
% We use a special adapted version, which is in the repro.
\usepackage{langsci-textipa}



% 8_levine

\usepackage{./styles/lg-macro2}
\usepackage{bm}
\usepackage{umoline}
\usepackage{pifont}
\usepackage{pstricks,pst-node,pst-tree}
\usepackage{ulem}
\usepackage{mathrsfs}
\usepackage{bussproofs}




%\usepackage{tikz,tikz-qtree}

%\usepackage{gb4e0}
%\noautomath

%\usepackage[letterpaper,margin=1.2in]{geometry}



% 14_kornai

%\usepackage{xypic} % seems not to be needed
% \usepackage[matrix,arrow]{xy}
%\usepackage{amsmath}
% \usepackage{subcaption}
% \usepackage{wrapfig}


   
\SetupAffiliations{output in groups = false,
                   orcid placement = after,
                   separator between two = {\bigskip\\},
                   separator between multiple = {\bigskip\\},
                   separator between final two = {\bigskip\\}
                   }

% ORCIDs in langsci-affiliations 
\definecolor{orcidlogocol}{cmyk}{0,0,0,1}
\RenewDocumentCommand{\LinkToORCIDinAffiliations}{ +m }
  {%
    \,\orcidlink{#1}%
  }


\makeatletter
\let\thetitle\@title
\let\theauthor\@author
\makeatother

\newcommand{\togglepaper}[1][0]{
   \bibliography{../localbibliography}
   \papernote{\scriptsize\normalfont
     \theauthor.
     \titleTemp.
     To appear in:
     E. Di Tor \& Herr Rausgeberin (ed.).
     Booktitle in localcommands.tex.
     Berlin: Language Science Press. [preliminary page numbering]
   }
   \pagenumbering{roman}
   \setcounter{chapter}{#1}
   \addtocounter{chapter}{-1}
}

\newbool{bookcompile}
\booltrue{bookcompile}
\newcommand{\bookorchapter}[2]{\ifbool{bookcompile}{#1}{#2}}


% Cite and cross-reference other chapters
\newcommand{\crossrefchaptert}[2][]{\citet*[#1]{chapters/#2}, Chapter~\ref{chap-#2} of this volume} 
\newcommand{\crossrefchapterp}[2][]{(\citealp*[#1]{chapters/#2}, Chapter~\ref{chap-#2} of this volume)}
\newcommand{\crossrefchapteralt}[2][]{\citealt*[#1]{chapters/#2}, Chapter~\ref{chap-#2} of this volume}
\newcommand{\crossrefchapteralp}[2][]{\citealp*[#1]{chapters/#2}, Chapter~\ref{chap-#2} of this volume}

\newcommand{\crossrefcitet}[2][]{\citet*[#1]{chapters/#2}} 
\newcommand{\crossrefcitep}[2][]{\citep*[#1]{chapters/#2}}
\newcommand{\crossrefcitealt}[2][]{\citealt*[#1]{chapters/#2}}
\newcommand{\crossrefcitealp}[2][]{\citealp*[#1]{chapters/#2}}


\newcommand{\sub}[1]{\textsubscript{\scriptsize\textrm{#1}}}

% Müller

\newcommand{\page}{}

\let\citew\citet

\def\underRevision{Revise and resubmit}

\let\textbfemph\emph

\newcommand{\todostefan}[1]{\todo[color=orange!80]{\footnotesize #1}\xspace}
\newcommand{\todosatz}[1]{\todo[color=red!40]{\footnotesize #1}\xspace}

\newcommand{\inlinetodostefan}[1]{\todo[color=green!40,inline]{\footnotesize #1}\xspace}

\newcommand{\inlinetodoopt}[1]{\todo[color=green!40,inline]{\footnotesize #1}\xspace}
\newcommand{\inlinetodoobl}[1]{\todo[color=red!40,inline]{\footnotesize #1}\xspace}

\newcommand{\itd}[1]{\inlinetodoobl{#1}}
\newcommand{\itdobl}[1]{\inlinetodoobl{#1}}
\newcommand{\itdopt}[1]{\inlinetodoopt{#1}}

\newcommand{\addpages}{\todostefan{add pages}}

%% % taken from https://tex.stackexchange.com/a/95079/18561
\newbox\usefulbox

\makeatletter
\def\getslant #1{\strip@pt\fontdimen1 #1}

\def\skoverline #1{\mathchoice
 {{\setbox\usefulbox=\hbox{$\m@th\displaystyle #1$}%
    \dimen@ \getslant\the\textfont\symletters \ht\usefulbox
    \divide\dimen@ \tw@ 
    \kern\dimen@ 
    \overline{\kern-\dimen@ \box\usefulbox\kern\dimen@ }\kern-\dimen@ }}
 {{\setbox\usefulbox=\hbox{$\m@th\textstyle #1$}%
    \dimen@ \getslant\the\textfont\symletters \ht\usefulbox
    \divide\dimen@ \tw@ 
    \kern\dimen@ 
    \overline{\kern-\dimen@ \box\usefulbox\kern\dimen@ }\kern-\dimen@ }}
 {{\setbox\usefulbox=\hbox{$\m@th\scriptstyle #1$}%
    \dimen@ \getslant\the\scriptfont\symletters \ht\usefulbox
    \divide\dimen@ \tw@ 
    \kern\dimen@ 
    \overline{\kern-\dimen@ \box\usefulbox\kern\dimen@ }\kern-\dimen@ }}
 {{\setbox\usefulbox=\hbox{$\m@th\scriptscriptstyle #1$}%
    \dimen@ \getslant\the\scriptscriptfont\symletters \ht\usefulbox
    \divide\dimen@ \tw@ 
    \kern\dimen@ 
    \overline{\kern-\dimen@ \box\usefulbox\kern\dimen@ }\kern-\dimen@ }}%
 {}}
\makeatother

% 1_intro.tex

% For the block quote:

\usepackage[most]{tcolorbox}
\definecolor{linequote}{RGB}{224,215,188}
\definecolor{backquote}{RGB}{249,245,233}
\newtcolorbox{myquote}[1][]{%
    enhanced, breakable, 
    size=minimal,
    frame hidden, boxrule=0pt,
    sharp corners,
    colback=backquote,
    #1
}

% 2_gibson.tex


% Example(s) Environments
% 12pt, No new-lines after example number is printed

\newcounter{examplectr}
\newcounter{fnexamplectr}

% Note: don't use subexamples in footnotes.

% This line is to overcome a bug in cmu-art style: it prints counter
% values to the aux file using \theaux... rather than using \the...
\def\theauxexamplectr{\theexamplectr}

\newcounter{subexamplectr}
\def\theauxsubexamplectr{\thesubexamplectr}
\def\theauxfnexamplectr{\thefnexamplectr}

\renewcommand{\theexamplectr}{\arabic{examplectr}}
% This command causes example numbers to appear without following periods

\renewcommand{\thefnexamplectr}{\roman{fnexamplectr}}
% This command causes example numbers to appear without following periods

\renewcommand{\thesubexamplectr}{\theexamplectr\alph{subexamplectr}}
% This command gives the number of an example and subexample as e.g. 1a, 2b

\newlength{\wdth}
\newcommand{\strike}[1]{\settowidth{\wdth}{#1}\rlap{\rule[.5ex]{\wdth}{1pt}}#1}

\newcommand{\exref}[1]{(\ref{#1})}
% This command puts reference numbers with parentheses
% surrounding them 

% The environment ``examples'' gives a list of examples, one on each line,
% numbered with a lower case alphabetic character
\newenvironment{examples}%
   { \vspace{-\baselineskip}
     \begin{list}%
     \textrm{\alph{subexamplectr}.}%
     {\usecounter{subexamplectr}
     \setlength{\topsep}{-\parskip}
     \setlength{\itemsep}{-2pt}
     \setlength{\leftmargin}{0.5in}
     \setlength{\rightmargin}{0in} } }%
   { \end{list}}

% The environment ``myexample'' outputs an arabic counter ``examplectr''
% surrounded by parentheses.
\newenvironment{myexample}
   { \vspace{20pt}
     \noindent
     \begin{minipage}{\textwidth}    % minipage environment disallows
                 % breaks across pages

     \refstepcounter{examplectr}     % step the counter and cause this
                 % section to be referenced by the
                 % counter ``examplectr''
     (\arabic{examplectr})}%
   { \vspace{20pt}
     \end{minipage}}

\newenvironment{myfnexample}
   { \vspace{2pt}
     \noindent
     \begin{minipage}{\textwidth}    % minipage environment disallows
                 % breaks across pages

     \refstepcounter{fnexamplectr}     % step the counter and cause this
                 % section to be referenced by the
                 % counter ``examplectr''
     (\roman{fnexamplectr})}%
   { \vspace{2pt}c
     \end{minipage}}
    
\newcommand*\circled[1]{\tikz[baseline=(char.base)]{
            \node[shape=circle,draw,inner sep=2pt] (char) {#1};}}



% 3_pullum.tex


% %%%  GKP:  I put in these pointless commands to kill off a bug elsewhere
% %%%        that tries to \newcommand \it (etc.) as \itshape (etc.), but
% %%%        fails because they haven't been defined.
% \newcommand{\it}{\relax}
% \newcommand{\bf}{\relax}
% \newcommand{\sc}{\relax}
% \newcommand{\rm}{\relax}

%%% GKP:  Two additional commands that I need
\newcommand{\data}[1]{\textit{#1}}
\newcommand{\blank}{\rule{1.2em}{0.5pt}}



% 8_levine

% ???
%\renewcommand{\emph}{\textit}
%\renewcommand{\em}{\it}

% not used \newcommand{\cites}[1]{\citeauthor{#1}'s~\citeyearpar{#1}}

% \renewcommand{\SetInfLen}{\setpremisesend{0pt}\setpremisesspace{10pt}\setnamespace{0pt}}

\newcommand{\pt}[1]{\ensuremath{\mathsf{#1}}}
\newcommand{\ptv}[1]{\ensuremath{\textsf{\textsl{#1}}}}

%\newcommand{\sv}[1]{\ensuremath{\bm{\mathcal{#1}}}}
\newcommand{\sv}[1]{\ensuremath{\mathcal{#1}}}

\newcommand{\sX}{\sv{X}}
\newcommand{\sF}{\sv{F}}
\newcommand{\sG}{\sv{G}}
%
% \renewcommand{\lex}{\SF}
% \renewcommand{\syncat}[1]{\ensuremath{\mathrm{#1}}}
% \newcommand{\syncatVar}[1]{\ensuremath{\mathit{#1}}}
%
% \newcommand{\RuleName}[1]{\textrm{#1}}
%
% \newcommand{\SemTyp}{\textsf{Sem}}
%
% \newcommand{\something}{\vdots\,\,\,\,\,\,\vdots}
%
% \newcommand{\pb}{\phantom{[}}
%
% \renewcommand{\E}{\ensuremath{\bm{\epsilon}}\xspace}
%
% \newcommand{\greeka}{\upalpha}
% \newcommand{\greekb}{\upbeta}
% \newcommand{\greekd}{\updelta}
\newcommand{\greekp}{\upvarphi}
\newcommand{\greekr}{\uprho}
\newcommand{\greeks}{\upsigma}
% \newcommand{\greekt}{\uptau}
% \newcommand{\greeko}{\upomega}
% \newcommand{\greekz}{\upzeta}
%
% % Do not do this!!!!!
% % \renewcommand{\labelenumi}{(\roman{enumi})}
%
% \newcommand{\upa}[2]{\ensuremath{\syncat{#1}|\syncat{#2}}}
% \newcommand{\dna}[2]{\upa{#2}{#1}}
%

%
% \newcommand{\Lemma}{\ensuremath{\vdots\hskip.5cm\vdots}\noLine}
%
% \newcommand{\la}{\ensuremath{\langle}}
% \newcommand{\ra}{\ensuremath{\rangle}}
%
% \renewcommand{\I}{\iota}
%
% \renewcommand{\sem}{\ensuremath}
%
% \newcommand{\LemmaShort}{\ensuremath{\vdots\hskip.2cm\vdots\hskip.2cm\vdots}\noLine}
% \newcommand{\LemmaShortAlt}{\ensuremath{\vdots\hskip.2cm\vdots}}
%
%
% \newcommand{\NoSem}{%
% \renewcommand{\LexEnt}[3]{##1; \syncat{##3}}
% \renewcommand{\LexEntTwoLine}[3]{\renewcommand{\arraystretch}{.8}%
% \begin{array}[b]{l} ##1;  \\ \syncat{##3} \end{array}}
% \renewcommand{\LexEntThreeLine}[3]{\renewcommand{\arraystretch}{.8}%
% \begin{array}[b]{l} ##1; \\ \syncat{##3} \end{array}}}
%
%
% \newcommand{\NoSemVar}{%
% \renewcommand{\LexEnt}[3]{##1; \syncat{##3}}
% \renewcommand{\LexEntTwoLine}[3]{\renewcommand{\arraystretch}{.8}%
% \begin{array}{l} ##1;  \\ \syncat{##3} \end{array}}
% \renewcommand{\LexEntThreeLine}[3]{\renewcommand{\arraystretch}{.8}%
% \begin{array}{l} ##1; \\ \syncat{##3} \end{array}}}
%
% \newcommand{\vs}{\raisebox{.05em}{\ensuremath{\upharpoonright}}}
%
% \newcommand{\AXX}[1]{\raisebox{-7mm}{\ensuremath{#1}}}
%
% \newcommand{\hypml}[2]{\left[\!\!#1\!\!\right]^{#2}}
%
% \newcommand{\alt}[2]{$\left\{\begin{array}{c}
% \hskip-.7ex\textrm{#1}\hskip-.7ex \\
% \hskip-.7ex\textrm{#2}\hskip-.7ex
%         \end{array}
% \right\}$}
%
% \newcommand{\altalt}[2]{\{#1/#2\}}
%
%
%
% \newcommand{\altt}[3]{$\left\{\begin{array}{c}
% \hskip-.7ex\textrm{#1}\hskip-.7ex \\
% \hskip-.7ex\textrm{#2}\hskip-.7ex \\
% \hskip-.7ex\textrm{#3}\hskip-.7ex
% \end{array}
% \right\}$}
%
% \newcommand{\alttt}[4]{$\left\{\begin{array}{c}
% \hskip-.7ex\textrm{#1}\hskip-.7ex \\
% \hskip-.7ex\textrm{#2}\hskip-.7ex \\
% \hskip-.7ex\textrm{#3}\hskip-.7ex \\
% \hskip-.7ex\textrm{#4}\hskip-.7ex
% \end{array}
% \right\}$}
%
%
% %%%%for bussproof
%
% \def\defaultHypSeparation{\hskip0.1in}
% \def\ScoreOverhang{0pt}
%
%
% %%\newcommand{\MultiLine}[1]{\renewcommand{\arraystretch}{.8}%
% %%\ensuremath{\begin{array}{l} #1 \end{array}}}
%
\newcommand{\MultiLine}[1]{\renewcommand{\arraystretch}{.8}%
\ensuremath{\begin{array}[b]{l} #1 \end{array}}}

%
%
% \newcommand{\MultiLineMod}[1]{%
% \ensuremath{\begin{array}[t]{l} #1 \end{array}}}
%
%
% %%%%%\AFourMargin
% %%\JLSubmissionMargin
%
% %%\setlength\topmargin{-1cm}
% %%\setlength\textheight{23cm}
% %%%%%\setlength\textwidth{13.5cm}
%
% %\setstretch{1.2}
%
% % not used \newcommand{\hyp}[2]{[ #]^{#2}}
%
\newcommand{\LexEnt}[3]{#1; \ensuremath{#2}; \syncat{#3}}
%
% \newcommand{\LexEntTwoLine}[3]{\renewcommand{\arraystretch}{.8}%
% \begin{array}[b]{l} #1; \\ \ensuremath{#2};  \syncat{#3} \end{array}}
%
% \newcommand{\LexEntThreeLine}[3]{\renewcommand{\arraystretch}{.8}%
% \begin{array}[b]{l} #1; \\ \ensuremath{#2}; \\ \syncat{#3} \end{array}}
%
%
% \newcommand{\LexEntFiveLine}[5]{\renewcommand{\arraystretch}{.8}%
% \begin{array}{l} #1 \\ #2; \\ \ensuremath{#3} \\ \ensuremath{#4}; \\ \syncat{#5} \end{array}}
%
%
% \newcommand{\LexEntFourLine}[4]{\renewcommand{\arraystretch}{.8}%
% \begin{array}{l} \pt{#1} \\ \pt{#2}; \\ \syncat{#4} \end{array}}
%
% \newcommand{\ManySomething}{\renewcommand{\arraystretch}{.8}%
% \raisebox{-3mm}{\begin{array}[b]{c} \vdots \,\,\,\,\,\, \vdots \\
% \vdots \,\,\,\,\,\, \vdots \end{array}}}
%
%
% \newcommand{\lemma}[1]{\renewcommand{\arraystretch}{.8}%
% \begin{array}[b]{c} \vdots \,\,\,\,\,\, \vdots \\ #1 \end{array}}
%
% \newcommand{\lemmarev}[1]{\renewcommand{\arraystretch}{.8}%
% \begin{array}[b]{c} #1 \\ \vdots \,\,\,\,\,\, \vdots \end{array}}
%
% \newcommand{\p}{\ensuremath{\upvarphi}}
%
% \newcommand{\Not}{\leavevmode\llap{\textbf{\smc{NOT:}} }}
%
% \newcommand{\Conj}{\fs{\bsp{\mathit{X}}{\mathit{X}}}{\mathit{X}}}
% \newcommand{\ConjY}{\fs{\bsp{\mathit{Y}}{\mathit{Y}}}{\mathit{Y}}}
% \newcommand{\sameLE}{\dna{(\upa{(\upa{S}{\mathit{X}})}{NP})}{(\upa{S}{\mathit{X}})}}
%
% \newcommand{\derivcenter}[2][1.1]{%
% \SetInfLen
% \attop{\vskip3ex
% \resizebox{#1\linewidth}{!}{\hskip-#1in
% #2}}}
%
% \newcommand{\derivcenterAlt}[2][.98]{%
% \SetInfLen
% \attop{\vskip3ex
% \resizebox{#1\linewidth}{!}{\hskip-#1in \hskip.5in
% #2}}\vspace{.5ex}}
%
% \renewcommand{\O}{\circ}  Do not recommand!
\newcommand{\BobsO}{\circ}
%
% \newcommand{\derivcenterMod}[2][1.1]{%
% \renewcommand{\LexEntThreeLine}[3]{\renewcommand{\arraystretch}{.8}%
% \raisebox{.4ex}{\ensuremath{\begin{array}{l} ##1; \\ \ensuremath{##2}; \\ \syncat{##3} \end{array}}}}
% \SetInfLen
% \attop{\vskip3ex
% \resizebox{#1\linewidth}{!}{\hskip-#1in
% #2}}}
%
%
% \newcommand{\shortarrow}{\xspace\hskip-1.2ex\scalebox{.5}[1]{\ensuremath{\bm{\rightarrow}}}\hskip-.5ex\xspace}
%
% \newcommand{\SemInt}[1]{\mbox{$[\![ \textrm{#1} ]\!]$}}
%
% \def\maru#1{{\ooalign{\hfil
%   \ifnum#1>999 \resizebox{.25\width}{\height}{#1}\else%
%   \ifnum#1>99 \resizebox{.33\width}{\height}{#1}\else%
%   \ifnum#1>9 \resizebox{.5\width}{\height}{#1}\else #1%
%   \fi\fi\fi%
% \/\hfil\crcr%
% \raise.167ex\hbox{\mathhexbox20D}}}}
%
\newcommand{\HypSpace}{\hskip-.8ex}
\newcommand{\RaiseHeight}{\raisebox{2.2ex}}
% \newcommand{\RaiseHeightLess}{\raisebox{1ex}}
%
% \newcommand{\fW}{\ensuremath{\mathfrak{W}}}
%
\newcommand{\ThreeColHyp}[1]{\RaiseHeight{\Bigg[}\HypSpace#1\HypSpace\RaiseHeight{\Bigg]}}
% \newcommand{\TwoColHyp}[1]{\RaiseHeightLess{\Big[}\HypSpace#1\HypSpace\RaiseHeightLess{\Big]}}
%
%
% %\newcommand{\maskref}[1]{\textsl{\textbf{[reference omitted for refereeing]}}}
% \newcommand{\maskref}[1]{#1}
%
% \newcommand{\DerivSize}{\small}
% \newcommand{\AppDerivSize}{\footnotesize}
%
% \renewcommand{\sem}{\ensuremath}


% \newcommand{\greekp}{{\color{green}π}}
% \newcommand{\greekr}{{\color{green}\textrho}}
% \newcommand{\greeks}{{\color{green}\textsigma}}
\newcommand{\ptfont}[1]{\texttt{#1}}                % what does ptfont do? Where is it defined?
                                % Question
%\newcommand{\ptfont}{\ttfamily}

%\newcommand{\grey}{\color{gray}}
\newcommand{\grey}[1]{\colorbox{mycolor}{#1}}
\definecolor{mycolor}{gray}{0.8}

\newcommand{\gap}{\longrule}
\newcommand{\gp}{\gap}
\newcommand{\vs}{\raisebox{.05em}{\ensuremath{\upharpoonright}}}
% \newcommand{\sub}[1]{\textsubscript[#1]}
\newcommand{\E}{\emph}
\newcommand{\B}{\textbf}
\newcommand{\f}{{\color{green}f}}  % Question what does f do? It does not have any output in the
                                % original PDF
%\newcommand{\Lemma}{{\color{pink}Lemma}}
\newcommand{\Lemma}{\ensuremath{\vdots\hskip.5cm\vdots}\noLine}

%\newcommand{\calP}{{\color{pink}calP}} % Sebastian
\newcommand{\calP}{\ensuremath{\mathcal{P}}}


\newcommand{\maru}[1]{\ooalign{\hfil#1\/\hfil\crcr
      \raise.05ex\hbox{\LARGE\mathhexbox20D}}}


%\newcommand{\sem}[2][M\!,g]{\mbox{$[\![ \mathrm{#2} ]\!]^{#1}$}}
\newcommand{\sem}{\ensuremath}

%
\newcommand{\trns}[1]{\textbf{#1}\xspace}

\newcommand{\bs}{{\textbackslash}}
\newcommand{\bsl}{{\bs}}


\newcommand{\fb}[1]{\textsubscript{#1}}

\newcommand{\syncat}[1]{\ensuremath{\mathrm{#1}}}
\newcommand{\term}[1]{\textit{#1}}
\newcommand{\LemmaAlt}{\ensuremath{\vdots\hskip.5cm\vdots}}

%\renewcommand{\O}{ø}

   %% hyphenation points for line breaks
%% Normally, automatic hyphenation in LaTeX is very good
%% If a word is mis-hyphenated, add it to this file
%%
%% add information to TeX file before \begin{document} with:
%% %% hyphenation points for line breaks
%% Normally, automatic hyphenation in LaTeX is very good
%% If a word is mis-hyphenated, add it to this file
%%
%% add information to TeX file before \begin{document} with:
%% %% hyphenation points for line breaks
%% Normally, automatic hyphenation in LaTeX is very good
%% If a word is mis-hyphenated, add it to this file
%%
%% add information to TeX file before \begin{document} with:
%% \include{localhyphenation}
\hyphenation{
    par-a-digm
}

\hyphenation{
    par-a-digm
}

\hyphenation{
    par-a-digm
}

   \boolfalse{bookcompile}
   \togglepaper[23]%%chapternumber
}{}

\begin{document}
\maketitle

\section{Introduction}
Mosetén is a Mosetenan language (cas, ISO 639—3) spoken in the tropical region of the lower foothills of the Bolivian Andes. The language family consists of just three closely related and mutually intelligible varieties. Mosetén of Covendo and Mosetén of Santa Ana, both spoken in the foothills of the La Paz Andes, are highly endangered and only have a few hundred speakers altogether. Chimane, spoken in the adjacent lowland Beni area, has a growing number of speakers, with various estimates of 5000—8000 speakers in total. Despite suggestions of relationships between Mosetenan and other South American languages, these have so far not been conclusively established and this small language family is still considered unrelated to other languages (cf. \citep{sakel2004grammar}). Despite the absence of genetic relationships, the Mosetenan varieties did not exist in isolation and have once been in close contact with a range of languages in the region, leading to loanwords and other contact phenomena, which are likely the reason for some of the superficial similarities with other language families. Historically, Mosetenan would have been in contact with indigenous languages such as Quechua, Uru-Chipaya and Tacanan. Since the middle of the 1950s, contact has mainly been with Spanish. This is not surprising, as Spanish is used in most official, and increasingly also personal, domains among many indigenous groups of Bolivia. In the Mosetén situation, Spanish influence has increased significantly in conjunction with better accessibility to the area, and subsequent heavy migration of indigenous farmers from the highlands in search of better living conditions.


\section{History and sociolinguistic profile of the language}
What we know about the history of the Mosetenan languages is largely what we can deduce from synchronic sources, such as variation between the varieties and speaker differences across generations. For example, Chimane has experienced less heavy language contact with Spanish. Dialectal differences aside, it can serve as a guide to the structures that have undergone changes in Mosetén due to contact with Spanish.

When I started to work on Mosetén in the 1990s, most Mosetén speakers would predominantly use Spanish in their day-to-day interactions, with Mosetén restricted to a few informal domains. For my grammar of the language \citep{sakel2004grammar}, I worked closely with older generations that used Mosetén as their primary language, preserving some of the grammatical structures that many young speakers had replaced with largely Spanish patterns. For example, older speakers would regularly use feminine pronouns when referring to groups of mixed-sex people (\ref{sakel_example_1}). Younger speakers generally preferred the use of the masculine in the same situation (\ref{sakel_example_2}) – modelled on the Spanish template (\ref{sakel_example_3}):

\ea \label{sakel_example_1}
\gll Mö'-in\\
     3F.\textsc{PL}\\
\glt `they, e.g. father and mother' (older Mosetén of Covendo speakers).
\z
\ea \label{sakel_example_2}
\gll Mi'-in\\
     3M.\textsc{PL}\\
\glt `they, e.g. father and mother' (younger Mosetén of Covendo speakers).
\z
\ea \label{sakel_example_3}
\gll Ellos\\
     3M.PL\\
\glt 'They, e.g. father and mother' (Spanish)
\z

\il{Spanish} %add "Spanish" to language index for this page
\il{Mosetén} %add "Mosetén" to language index for this page

\section{Language contact}
The above changes to gender agreement can be subtle and difficult to identify as potential Spanish influence without an in-depth analysis of the patterns of the language. This is because Mosetén lexical elements are used to model Spanish patterns, without the direct loan of Spanish elements \citep{matras2007grammatical}\citep{matras2007investigating}. The other type of loan, Matter borrowing of Spanish morphophonological elements is also attested in modern Mosetén and very common. These loans are often much more obvious, as they stand out as Spanish words. However, some of these loans have been adjusted phonologically, e.g. Spanish \textit{hasta} ‘until’ is pronounced \textit{ashta} in Mosetén. Matter loans can go hand in hand with borrowed patterns. For example, Mosetén has borrowed many Spanish function words, such as coordinators, subordinating conjunctions, markers of time and space, discourse markers and delimitation markers that are borrowed together with their respective syntactic patterns \citep{sakel2007language}\citep{sakel2007moseten}:

\ea \label{sakel_example_4}
\gll Its-näjä’ 	yi-sin’ 		\textbf{ke} 	jam-ra’ 	karij-tya-kha’.\\
     DEM\textsc{M}-FOC  say-1.PL.OBJ    that.E  NEG-IRR work-APPL-1PL.INCL.SBJ\\
\glt ‘This one (now) told us that we all wouldn’t be working.’
\z

(\ref{sakel_example_4}) shows a sentence structure modelled on Spanish, using the conjunction \textit{ke} (a direct Matter loan from Spanish \textit{que} ‘that’) between the two clauses. In the language spoken by elders, complement clauses can be expressed in a range of different ways, the most typical way being the addition of a clitic \textit{-dye’} to the verb of the subordinate clause:

\ea \label{sakel_example_5}
\gll Yäe  	ködye-ye  	\textbf{sob-a-k-dye’}  		\textbf{öi-yä’}  		\textbf{phen}.\\
     1SG	beg-1SG/2SG	visit-V-ANTIP-NMLZ	DEM.F-LOC	woman\\
\glt ‘I beg you to visit this woman.’
\z

\section{Data on the language}
Mosetén and Chimane are relatively well described, with grammars and further analyses of the varieties, most notably \citep{gill1999pedagogical}, who wrote various manuscripts on Chimane, and my own work on the grammar of the Mosetén and later Chimane, first published in the early 2000s \citep{sakel2004grammar}.

Yet, as for many other indigenous South American languages, we have very little historical information for Mosetén. The first acknowledgements of the language came from missionary sources, e.g. the Mosetenes (then referred to as Amo) were mentioned in 1588 \citep{metraux1942native}, with various missionaries more or less successfully settling in the area and subsequently noting down some information on the language. 

The first language data are presented by Andrés Herrero, a Franciscan missionary settling in the region in the early 1800s. Upon his return to Europe in 1834, he put together a prayer book on the language.

The Franciscan missionary Benigno Bibolotti stayed in the village of Covendo in 1857. His notes on the language were published and analysed by Rudolph Schuller, who published a basic grammatical description of the language based on Bibolotti’s original data as an 'introduction', alongside Bibolotti’s manuscript \citep{bibolotti1917moseteno, schuller1917introduction}. 

There are a number of other collections of information on the language, most notably a collection of data by a renowned Swedish adventurer, Erland von Nordenskiöld, who visited the region on an excursion in 1913, when he spent a short time in the Mosetén area and got a speaker of the language, Tomas Huasna, to write down three short stories for him, which are preserved in Nordenskiöld's diary held at the archive of the Etnografiska Museum in Göteborg, Sweden. Various publications by Nordenskiöld reference these stories \citep{nordenskiold1924forschungen}. These are the first native-speaker first hand language data we have of the language.

During my fieldwork on Mosetén in the 1990s and 2000s, I worked with Tomas Huasna’s grandson, the late Juan Huasna, who remembered Tomas as a modern, forward-looking man who had spent a great deal of time working with the local missionaries. The missionaries had taught him to read and write in Spanish and, to a certain degree, Mosetén, and he was helping with the translation of prayers and bible portions. 

\section{Lexical borrowing in the historical data}
A number of lexical loans are attested in the historical data. Loan words in Herrero's text are mainly of a religious nature, e.g: \textit{santo} ‘holy’, \textit{reino} ‘kingdom’, \textit{gracia} ‘grace’, \textit{salve} ‘hail’, \textit{virgen} ‘virgin’ and \textit{testimonio} ‘testimony’. Some of the borrowed elements are integrated into Mosetén structures, for example turning them into Mosetén verbs, which are obligatorily marked by verbal affixes: 
\textit{misa-arai} (attend.mass.E-verb, ‘to attend mass’), \textit{confes-arai} (confess.E-verb, ‘to confess’), \textit{comulga-arai} (commune.E-verb, ‘to commune’). Huasna uses the Spanish loan \textit{semana} ‘week’ in his text, capturing a western concept of time keeping. All loans are purely lexical, and there are no Matter loans of Spanish subordinators together with their structures in the historical data, despite their frequency in modern Mosetén. 

\section{Comparing the historical data with modern varieties of the language}
It is impossible to rule out any grammatical borrowing in the historical data. Pattern borrowing is often linked with a degree of bilingualism. It is unclear how much Mosetén Bibolotti and Herrero acquired. They were keenly working on the language, so may have had a certain level of command. Huasna is likely to have been fairly proficient in Spanish, working with the missionaries on a daily basis and assisting in their tasks. Some grammatical interference may have happened due to Mosetén speakers imitating the missionaries' imperfect learning of their language.

While not ruling out the possibility of changes in the language due to contact with Spanish or other Romance languages such as Italian (Bibolotti was a native speaker of Italian), the old language data is most likely able to give us an insight into a much earlier stage in the language contact journey. Thus, variation we see in modern varieties of Mosetén across speakers of different ages, environments and level of bilingualism with Spanish would be expected to be different in earlier stages of that contact journey, giving us an insight into possible changes due to language contact in modern Mosetén.

In \citep{sakel2007moseten} I identified two areas of grammar that had undergone Pattern changes due to the influence of Spanish in modern Mosetén: 1. changes in the use of gender: the unmarked gender changing from feminine to masculine and 2. changes in the word order within the NP: modifiers (esp. adjective) - head noun.

I will test to which degree my assumptions based on the comparison of synchronic data in the language are backed up by the historical data, as well as adding some information from Chimane.

\section{Gender agreement}
As shown in (\ref{sakel_example_1}) and (\ref{sakel_example_2}) above, there is a difference in how older and younger speakers of modern Mosetén attribute gender to mixed-sex groups: younger speakers typically model the Spanish pattern of using the masculine gender, while some older speakers with less frequent exposure to Spanish would use the feminine gender in the same situation. 

Herrero  presents a number of examples of feminine forms used in situations with male protagonists (head nouns underlined, agreement markers in bold):

\ea \label{sakel_example_6}
\gll \underline{Dios} 	\textbf{momo}	cogchi-cam 	eraise-\textbf{te} 	mi.\\
     God.E	only.F	heart-LOC	love-3M.OBJ	2SG\\
\glt ‘You love only God from your heart.’
\z

\ea \label{sakel_example_7}
\gll \underline{Dios} \underline{Mumu}, 	\underline{Dios} \underline{Aba-mu},	\underline{Dios} \underline{Espiritu} \underline{Santo}.\\
     God.E Father	God.E son-his	God.E holy.E spirit.E\\
     
     Chivin \textbf{munsi}, 		\textbf{yeret} 	\textbf{momo} 	\underline{Dios} 	ato.\\
    three	people.F	one.M	only.F	God	yet\\
\glt ‘The father, the son and the holy spirit - three people, but only one God.’
\z	

\ea \label{sakel_example_8}
\gll \underline{Jesu} \underline{Chisto} 	tim-\textbf{mo}.\\
     Jesus Christ	name-POSS.F\\
\glt ‘His name is Jesus Christ.’
\z

 \ea \label{sakel_example_9}
\gll YäeAchii-\textbf{ti} 	\textbf{\underline{munsi}} 		uñan 	arai 	inca-\textbf{Ø}-in?\\
     bad-POSS.M	people.F	where	IRR	go-M.SBJ-PL\\
\glt ‘Where do the bad people go?’
\z

In (\ref{sakel_example_6}) and (\ref{sakel_example_7}) \textit{Dios} ‘God’ appears with the form \textit{momo} ‘only’ and \textit{Jesu Chisto} ‘Jesus’ in (\ref{sakel_example_8}) appears with the related possessive pronoun \textit{mo}. In modern Mosetén, these forms are feminine and have the masculine equivalents \textit{mumu’} and \textit{mu’} (\textit{mimi’} and \textit{mi’} in some varieties of Mosetén). Are these representations of God and Jesus expressed as overtly feminine? God and Jesus are theoretical concepts, so the use of feminine as a generic gender may have been appropriate in this case, despite their depictions as male protagonists. Another possibility is that \textit{momo} and \textit{mo} are generic, underlying forms that could be used in both masculine and feminine environments. A third option is that Herrero made a mistake when noting these down. However, there appears some consistency in the use of feminine agreement in these cases, which makes it less likely to be a mistake. 
Note the use of a masculine cross-reference marker \textit{-te} ‘3rd person masculine object’ in (\ref{sakel_example_6}) and the masculine form of the numeral \textit{yeret} ‘one’ in (\ref{sakel_example_7}), which refers to God as a masculine entity. Thus, masculine gender agreement is used at the same time as the feminine forms. We see that masculine gender agreement is present elsewhere, for example in the cross-reference marking referring to a masculine subject (\ref{sakel_example_9}). 
Likewise, Bibolotti has examples of the use of \textit{momo} in environments where reference is to antecedents that are not exclusively feminine: 

\ea \label{sakel_example_10}
\gll Eñe-ra 	Cui 	tsuñ 	\textbf{momo}.\\
     like-IRR	self	we	only.F\\
\glt ‘just like ourselves’ (referring to the people)
\z

\ea \label{sakel_example_11}
\gll Dojit-\textbf{si} 	Aua-\textbf{mu}\\
     God-POSS.F	son-3M.POSS\\
\glt ‘God’s son’ (lit. ‘of God, his son’)
\z	

There various examples of the type given in (\ref{sakel_example_10}), where \textit{momo} is used with masculine or mixed-sex antecedents. (\ref{sakel_example_11}) shows an example of the use of feminine agreement with masculine antecedents. In this case, the possessive marker on the head \textit{Dojit} ‘God’ is in the feminine form, while the possessed entity \textit{Aua} ‘son’ appears with a masculine possessive marker.	
The texts written by Huasna are consistent with what we find in modern Mosetén, using masculine agreement forms of \textit{momo’/mumu’} with masculine heads:

 \ea \label{sakel_example_12}
\gll Pfai-tiiñ		tac-\textbf{mumu}			caca-tiiñ\\
     jump.on-VIO.M.SBJ	throw.to.ground-just.M	pick.up-VIO.M.SBJ\\
\glt ‘He (the jaguar) jumps on him violently, just throws him to the ground (and) picks him up.’
\z	

Indeed, we even find the use of masculine gender agreement when referring to various people of mixed (or unclear) gender, despite there being variation in modern Mosetén (cf examples \ref{sakel_example_1} and \ref{sakel_example_2} above):


 \ea \label{sakel_example_13}
\gll rre	\textbf{mu}-che	jicai-\textbf{Ø}-iñ\\
     all	up.there.M	go-M.SBJ-PL\\
\glt ‘they all went up there.’ (context: men, women, big children, small children)
\z	

In summary, the picture presented for gender agreement is somewhat complex. While Huasna appears to display a number of characteristics that are typical of modern speakers with heavy exposure to Spanish, Bibolotti and Herrero show examples of feminine forms used in masculine or mixed-sex environments. 
Provided these L2 speaker missionaries did not make a mistake, feminine forms are not just used as neutral gender forms in mixed-sex environments, but appear to also be also with some masculine antecedents – sometimes together with masculine agreement forms referring back to the same antecedents. This type of ‘mixed’ gender marking is not found in modern Mosetén. However, it exists – in part - in the closely related language Chimane, which has experienced considerably less contact with Spanish. In Chimane, \textit{momo’} ‘only, F’ generally used in NPs with masculine antecedents \citep{gill1999pedagogical}.

The likely explanation is that \textit{momo’} ‘only’ – a reduplicated form of the third person personal pronoun \textit{mo’} ‘she’ was originally used more generally across the language and could be applied to both masculine and feminine environments. Due to increasing pressure from Spanish, a masculine gender form \textit{mimi’/mumu’} appeared in analogy with \textit{momo’}, which is already present in the first-hand data presented by Huasna in 1913.

\section{Word order}
Another area of grammar that is often affected by language contact through Pattern changes without overt Matter borrowing is word order. Looking at the phrasal word order in modern Mosetén, we find variation: both orders N-ADJ and ADJ-N are accepted \cite[p.~103]{sakel2004grammar}. There is a tendency related to the animacy of the head, namely inanimate heads are typically preceded by an adjective ADJ-N , while animate heads are typically followed by the modifier N-ADJ, though the reverse order ADJ-N is possible as well. In Spanish, when the adjective describes a quality of the head, the order is typically N-ADJ, as opposed to ADJ-N to express a level of emphasis or appreciation of the head.
While there are not many examples of phrasal word order in Herrero, those that appear are ADJ-N (repeated from \ref{sakel_example_9}):

\ea \label{sakel_example_14}
\gll \textbf{Achii-ti} 	\textbf{munsi} 		uñan 	arai  	inca-in?\\
     Bad.POSS	people.F	where	IRR 	go.M.SBJ-PL\\
\glt ‘Where do bad people go?’
\z	

In the historical data, Bibolotti’s manuscript sets out instructions for other missionaries to understand the intricacies of the language. He translates Spanish phrases into Mosetén, at first giving a literal translation and then indicates the preferences by the speakers he worked with. In this way, ADJ-N word order is consistently ‘corrected’ in Bibolotti’s data (the relevant NPs are highlighted in bold): 

Literal:
\ea \label{sakel_example_15}
\gll Chinca  peaqui  \textbf{peacge}  \textbf{achis},  vori    Soyo    cañ cuisi   cotchi!\\
     that.who   speak    speech  bad	call   demon?    in  own heart\\
\glt ‘The one who speaks dirty words calls the devil in his heart.’
\z	

Corrected:
\ea \label{sakel_example_16}
\gll Chinca  peaqui  \textbf{achis}  \textbf{peacge},  vori    Soyo   cuisi   cotchi  cañ!\\
     that.who   speak    bad    speech	call   demon?  own heart   in\\
\glt ‘The one who speaks dirty words calls the devil in his heart.’
\z	

Literal:
\ea \label{sakel_example_17}
\gll Ges \textbf{soñi} \textbf{achitchit}, ere coi ueñege.\\
     for    man bad.bad all appear  dream\\
\glt ‘To the corrupt man, everything seems like an illusion.’
\z	

Corrected:
\ea \label{sakel_example_18}
\gll \textbf{Achitchi-ges} \textbf{soñi}, ere coi ueñege.\\
     bad.bad-for    man all appear  dream\\
\glt ‘To the corrupt man, everything seems like an illusion.’
\z	

Thus, in this case the order of \textit{achis} ‘bad.F’ and \textit{peacge} ‘word, story (F)’ is ‘corrected’ from the literal translation N-ADJ to ADJ-N. Likewise, the animate head noun \textit{soñi} ‘man’ and the adjective \textit{achitchit} ‘very bad’ are presented as following a preferred ADJ-N order. The element \textit{-ges} ‘for’ is a clitic in Mosetén, as opposed to a preposition in Spanish, as given in the literal translation.
While the literal translations in Bibolotti’s times did not seem to be acceptable to the speakers he was working with – or at least those speakers had a clear preference for the ADJ-N word order - in modern Mosetén both phrasal word orders are often acceptable and commonly used.	
Finally, Huasna has no clear examples of noun phrase word orders. In two cases, adjectives appear with nouns in the order N-ADJ, but they are divided by commas. Thus, it is unclear whether this is the word order N-ADJ, or whether Huasna added the adjective after the noun as a form of afterthought:

\ea \label{sakel_example_19}
\gll Oi 		\textbf{Pfeyacgej-iñ}, 		\textbf{Poroma-si}\\
     DEM.F	story-PL		old.POSS\\
\glt ‘these stories, the old ones’
\z	

\ea \label{sakel_example_20}
\gll jique	muñthi-iñ,	pfeñ-iñ,	\textbf{ñañathi-iñ},  \textbf{chi-dere-si-iñ},		\textbf{chi-chubo-si-si-iñ},\\
     	PST	man-PL	woman-PL	boy-PL	also-big-POSS.F-PL	also-carried-POSS.F-POSS.F-PL\\
\glt ‘and the men, the women, the boys, the big ones and also the ones carried (by their mothers)’
\z	
	
In summary, we may be seeing a loosening of the word order in Mosetén. While Bibolotti indicates a clear speaker preference for ADJ-N order, modern Mosetén allows a range of phrasal word orders. This looks to be closer to the Spanish pattern, while not being a carbon copy. This does not rule out language contact, as it is commonly attested in contact situations that structures resulting from language contact are not exact copies of the source language, but may undergo partial changes \citep{heine2005language}. Yet, both phrasal orders also exist in Chimane, which means that the loosening of the word order may either not be entirely due to language contact or Chimane may have undergone the same contact.


\section{Conclusion}
It can be difficult to attest language contact influence at the level of Pattern borrowing in grammar. We may be able get an insight into this by looking at synchronic data of speakers of different characteristics and varieties of a language or closely related languages with different language contact histories and levels of exposure to the contact language. Yet, historical data on a language can give us further insights into contact histories, being able to test a range of scenarios.

While historical data on a language can be helpful, we have to accept various insecurities: did the L2 speakers get it right? Did some of the L1 speakers already display considerable amounts of Spanish influence, e.g. Huasna, whose language seems much closer to some of the more progressive speakers of modern Mosetén?

The picture of Pattern borrowing becomes more complex by looking at the historical data, rather than supporting my original hypothesis of contact-induced changes modelled on synchronic language use. While overtly many modern Mosetén structures are modelled on Spanish patterns, often associated with lexical loans, the detailed analysis of gender agreement and noun phrase word order are only in part explicable as a result of language contact. 



\section*{Abbreviations}
Abbreviations are following the Leipzig glossing rules. Additional abbreviations:
\begin{tabularx}{.5\textwidth}{@{}lQ@{}}
1 & first person\\
2 & second person\\
3 & third person\\
ADJ & adjective\\
ANTIP & antipassive\\
APPL & applicative\\
DEM & demonstrative\\
E & Spanish loan\\
F & feminine\\
FOC & focus\\
INCL & inclusive\\
IRR & irrealis\\
LOC & locative\\
M & masculine\\
NEG & negation\\
NMLZ & nominalizer\\
OBJ & object\\
PL & plural\\
POSS & possessive\\
SBJ & subject\\
SG & singular\\
V & Verb marker\\
VIO & Marker for violence\\ 
\end{tabularx}%
\begin{tabularx}{.5\textwidth}{@{}lQ@{}}
 \ldots  & \\
 \ldots  & \\
\end{tabularx}

\sloppy
\printbibliography[heading=subbibliography,notkeyword=this]
\end{document}

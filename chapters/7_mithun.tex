\documentclass[output=paper,colorlinks,citecolor=brown
% ,hidelinks
% showindex
]{langscibook}

\author{Marianne Mithun\orcid{}\affiliation{University of California, Santa Barbara}}
\title{Just Where are the Universals?
Complexities of Place
}
\abstract{A major thrust of much theoretical linguistics has been the search for ‘universals’: innate, abstract, formal design principles governing all languages. Work by Dan Everett, based on long-term observation of speech in its cultural context, has raised questions about their universality and precise nature, and explored the power of culture in shaping language \citep{everett2012language,everett2015role,everett2018role}. One proposed universal is that a fundamental design feature is recursion via complex sentences. If we look only at translations of isolated sentences from English or another contact language, we stand to miss some intriguing complexities and perhaps some deeper understanding of forces that can shape language. Here such issues are explored in the examination of the expression of place in Mohawk. Though the language does contain complex sentences of the type ascribed to universals, an examination of unscripted speech in context shows that syntactic constructions specifying place are more complex and powerful than might be predicted purely from a principle of recursion.}

\graphicspath{ {./figures/} }
\IfFileExists{../localcommands.tex}{
   \addbibresource{../localbibliography.bib}
   \newcommand{\orcid}[1]{}

\usepackage{orcidlink}

\usepackage{tabularx,multicol}
\usepackage{url}
\urlstyle{same}


\usepackage{langsci-optional}
\usepackage{langsci-lgr}
\usepackage{langsci-gb4e}

% for texlive 2022
\usepackage{langsci-branding} 

% Müller


% \usepackage{biblatex-series-number-checks}

% \usepackage{eng-date}

% \usepackage{german}%% Das Buch ist nicht deutsch. Hör auf, solche Sachen zu laden


% \usepackage{tikz-dependency}
% \usepackage{tikz}
% \usepackage{tikz-qtree}

% \usepackage{hologo}

% 3_pullum.tex

% This does not work complains about recommanding epsilon
% We use a special adapted version, which is in the repro.
\usepackage{langsci-textipa}



% 8_levine

\usepackage{./styles/lg-macro2}
\usepackage{bm}
\usepackage{umoline}
\usepackage{pifont}
\usepackage{pstricks,pst-node,pst-tree}
\usepackage{ulem}
\usepackage{mathrsfs}
\usepackage{bussproofs}




%\usepackage{tikz,tikz-qtree}

%\usepackage{gb4e0}
%\noautomath

%\usepackage[letterpaper,margin=1.2in]{geometry}



% 14_kornai

%\usepackage{xypic} % seems not to be needed
% \usepackage[matrix,arrow]{xy}
%\usepackage{amsmath}
% \usepackage{subcaption}
% \usepackage{wrapfig}


   
\SetupAffiliations{output in groups = false,
                   orcid placement = after,
                   separator between two = {\bigskip\\},
                   separator between multiple = {\bigskip\\},
                   separator between final two = {\bigskip\\}
                   }

% ORCIDs in langsci-affiliations 
\definecolor{orcidlogocol}{cmyk}{0,0,0,1}
\RenewDocumentCommand{\LinkToORCIDinAffiliations}{ +m }
  {%
    \,\orcidlink{#1}%
  }


\makeatletter
\let\thetitle\@title
\let\theauthor\@author
\makeatother

\newcommand{\togglepaper}[1][0]{
   \bibliography{../localbibliography}
   \papernote{\scriptsize\normalfont
     \theauthor.
     \titleTemp.
     To appear in:
     E. Di Tor \& Herr Rausgeberin (ed.).
     Booktitle in localcommands.tex.
     Berlin: Language Science Press. [preliminary page numbering]
   }
   \pagenumbering{roman}
   \setcounter{chapter}{#1}
   \addtocounter{chapter}{-1}
}

\newbool{bookcompile}
\booltrue{bookcompile}
\newcommand{\bookorchapter}[2]{\ifbool{bookcompile}{#1}{#2}}


% Cite and cross-reference other chapters
\newcommand{\crossrefchaptert}[2][]{\citet*[#1]{chapters/#2}, Chapter~\ref{chap-#2} of this volume} 
\newcommand{\crossrefchapterp}[2][]{(\citealp*[#1]{chapters/#2}, Chapter~\ref{chap-#2} of this volume)}
\newcommand{\crossrefchapteralt}[2][]{\citealt*[#1]{chapters/#2}, Chapter~\ref{chap-#2} of this volume}
\newcommand{\crossrefchapteralp}[2][]{\citealp*[#1]{chapters/#2}, Chapter~\ref{chap-#2} of this volume}

\newcommand{\crossrefcitet}[2][]{\citet*[#1]{chapters/#2}} 
\newcommand{\crossrefcitep}[2][]{\citep*[#1]{chapters/#2}}
\newcommand{\crossrefcitealt}[2][]{\citealt*[#1]{chapters/#2}}
\newcommand{\crossrefcitealp}[2][]{\citealp*[#1]{chapters/#2}}


\newcommand{\sub}[1]{\textsubscript{\scriptsize\textrm{#1}}}

% Müller

\newcommand{\page}{}

\let\citew\citet

\def\underRevision{Revise and resubmit}

\let\textbfemph\emph

\newcommand{\todostefan}[1]{\todo[color=orange!80]{\footnotesize #1}\xspace}
\newcommand{\todosatz}[1]{\todo[color=red!40]{\footnotesize #1}\xspace}

\newcommand{\inlinetodostefan}[1]{\todo[color=green!40,inline]{\footnotesize #1}\xspace}

\newcommand{\inlinetodoopt}[1]{\todo[color=green!40,inline]{\footnotesize #1}\xspace}
\newcommand{\inlinetodoobl}[1]{\todo[color=red!40,inline]{\footnotesize #1}\xspace}

\newcommand{\itd}[1]{\inlinetodoobl{#1}}
\newcommand{\itdobl}[1]{\inlinetodoobl{#1}}
\newcommand{\itdopt}[1]{\inlinetodoopt{#1}}

\newcommand{\addpages}{\todostefan{add pages}}

%% % taken from https://tex.stackexchange.com/a/95079/18561
\newbox\usefulbox

\makeatletter
\def\getslant #1{\strip@pt\fontdimen1 #1}

\def\skoverline #1{\mathchoice
 {{\setbox\usefulbox=\hbox{$\m@th\displaystyle #1$}%
    \dimen@ \getslant\the\textfont\symletters \ht\usefulbox
    \divide\dimen@ \tw@ 
    \kern\dimen@ 
    \overline{\kern-\dimen@ \box\usefulbox\kern\dimen@ }\kern-\dimen@ }}
 {{\setbox\usefulbox=\hbox{$\m@th\textstyle #1$}%
    \dimen@ \getslant\the\textfont\symletters \ht\usefulbox
    \divide\dimen@ \tw@ 
    \kern\dimen@ 
    \overline{\kern-\dimen@ \box\usefulbox\kern\dimen@ }\kern-\dimen@ }}
 {{\setbox\usefulbox=\hbox{$\m@th\scriptstyle #1$}%
    \dimen@ \getslant\the\scriptfont\symletters \ht\usefulbox
    \divide\dimen@ \tw@ 
    \kern\dimen@ 
    \overline{\kern-\dimen@ \box\usefulbox\kern\dimen@ }\kern-\dimen@ }}
 {{\setbox\usefulbox=\hbox{$\m@th\scriptscriptstyle #1$}%
    \dimen@ \getslant\the\scriptscriptfont\symletters \ht\usefulbox
    \divide\dimen@ \tw@ 
    \kern\dimen@ 
    \overline{\kern-\dimen@ \box\usefulbox\kern\dimen@ }\kern-\dimen@ }}%
 {}}
\makeatother

% 1_intro.tex

% For the block quote:

\usepackage[most]{tcolorbox}
\definecolor{linequote}{RGB}{224,215,188}
\definecolor{backquote}{RGB}{249,245,233}
\newtcolorbox{myquote}[1][]{%
    enhanced, breakable, 
    size=minimal,
    frame hidden, boxrule=0pt,
    sharp corners,
    colback=backquote,
    #1
}

% 2_gibson.tex


% Example(s) Environments
% 12pt, No new-lines after example number is printed

\newcounter{examplectr}
\newcounter{fnexamplectr}

% Note: don't use subexamples in footnotes.

% This line is to overcome a bug in cmu-art style: it prints counter
% values to the aux file using \theaux... rather than using \the...
\def\theauxexamplectr{\theexamplectr}

\newcounter{subexamplectr}
\def\theauxsubexamplectr{\thesubexamplectr}
\def\theauxfnexamplectr{\thefnexamplectr}

\renewcommand{\theexamplectr}{\arabic{examplectr}}
% This command causes example numbers to appear without following periods

\renewcommand{\thefnexamplectr}{\roman{fnexamplectr}}
% This command causes example numbers to appear without following periods

\renewcommand{\thesubexamplectr}{\theexamplectr\alph{subexamplectr}}
% This command gives the number of an example and subexample as e.g. 1a, 2b

\newlength{\wdth}
\newcommand{\strike}[1]{\settowidth{\wdth}{#1}\rlap{\rule[.5ex]{\wdth}{1pt}}#1}

\newcommand{\exref}[1]{(\ref{#1})}
% This command puts reference numbers with parentheses
% surrounding them 

% The environment ``examples'' gives a list of examples, one on each line,
% numbered with a lower case alphabetic character
\newenvironment{examples}%
   { \vspace{-\baselineskip}
     \begin{list}%
     \textrm{\alph{subexamplectr}.}%
     {\usecounter{subexamplectr}
     \setlength{\topsep}{-\parskip}
     \setlength{\itemsep}{-2pt}
     \setlength{\leftmargin}{0.5in}
     \setlength{\rightmargin}{0in} } }%
   { \end{list}}

% The environment ``myexample'' outputs an arabic counter ``examplectr''
% surrounded by parentheses.
\newenvironment{myexample}
   { \vspace{20pt}
     \noindent
     \begin{minipage}{\textwidth}    % minipage environment disallows
                 % breaks across pages

     \refstepcounter{examplectr}     % step the counter and cause this
                 % section to be referenced by the
                 % counter ``examplectr''
     (\arabic{examplectr})}%
   { \vspace{20pt}
     \end{minipage}}

\newenvironment{myfnexample}
   { \vspace{2pt}
     \noindent
     \begin{minipage}{\textwidth}    % minipage environment disallows
                 % breaks across pages

     \refstepcounter{fnexamplectr}     % step the counter and cause this
                 % section to be referenced by the
                 % counter ``examplectr''
     (\roman{fnexamplectr})}%
   { \vspace{2pt}c
     \end{minipage}}
    
\newcommand*\circled[1]{\tikz[baseline=(char.base)]{
            \node[shape=circle,draw,inner sep=2pt] (char) {#1};}}



% 3_pullum.tex


% %%%  GKP:  I put in these pointless commands to kill off a bug elsewhere
% %%%        that tries to \newcommand \it (etc.) as \itshape (etc.), but
% %%%        fails because they haven't been defined.
% \newcommand{\it}{\relax}
% \newcommand{\bf}{\relax}
% \newcommand{\sc}{\relax}
% \newcommand{\rm}{\relax}

%%% GKP:  Two additional commands that I need
\newcommand{\data}[1]{\textit{#1}}
\newcommand{\blank}{\rule{1.2em}{0.5pt}}



% 8_levine

% ???
%\renewcommand{\emph}{\textit}
%\renewcommand{\em}{\it}

% not used \newcommand{\cites}[1]{\citeauthor{#1}'s~\citeyearpar{#1}}

% \renewcommand{\SetInfLen}{\setpremisesend{0pt}\setpremisesspace{10pt}\setnamespace{0pt}}

\newcommand{\pt}[1]{\ensuremath{\mathsf{#1}}}
\newcommand{\ptv}[1]{\ensuremath{\textsf{\textsl{#1}}}}

%\newcommand{\sv}[1]{\ensuremath{\bm{\mathcal{#1}}}}
\newcommand{\sv}[1]{\ensuremath{\mathcal{#1}}}

\newcommand{\sX}{\sv{X}}
\newcommand{\sF}{\sv{F}}
\newcommand{\sG}{\sv{G}}
%
% \renewcommand{\lex}{\SF}
% \renewcommand{\syncat}[1]{\ensuremath{\mathrm{#1}}}
% \newcommand{\syncatVar}[1]{\ensuremath{\mathit{#1}}}
%
% \newcommand{\RuleName}[1]{\textrm{#1}}
%
% \newcommand{\SemTyp}{\textsf{Sem}}
%
% \newcommand{\something}{\vdots\,\,\,\,\,\,\vdots}
%
% \newcommand{\pb}{\phantom{[}}
%
% \renewcommand{\E}{\ensuremath{\bm{\epsilon}}\xspace}
%
% \newcommand{\greeka}{\upalpha}
% \newcommand{\greekb}{\upbeta}
% \newcommand{\greekd}{\updelta}
\newcommand{\greekp}{\upvarphi}
\newcommand{\greekr}{\uprho}
\newcommand{\greeks}{\upsigma}
% \newcommand{\greekt}{\uptau}
% \newcommand{\greeko}{\upomega}
% \newcommand{\greekz}{\upzeta}
%
% % Do not do this!!!!!
% % \renewcommand{\labelenumi}{(\roman{enumi})}
%
% \newcommand{\upa}[2]{\ensuremath{\syncat{#1}|\syncat{#2}}}
% \newcommand{\dna}[2]{\upa{#2}{#1}}
%

%
% \newcommand{\Lemma}{\ensuremath{\vdots\hskip.5cm\vdots}\noLine}
%
% \newcommand{\la}{\ensuremath{\langle}}
% \newcommand{\ra}{\ensuremath{\rangle}}
%
% \renewcommand{\I}{\iota}
%
% \renewcommand{\sem}{\ensuremath}
%
% \newcommand{\LemmaShort}{\ensuremath{\vdots\hskip.2cm\vdots\hskip.2cm\vdots}\noLine}
% \newcommand{\LemmaShortAlt}{\ensuremath{\vdots\hskip.2cm\vdots}}
%
%
% \newcommand{\NoSem}{%
% \renewcommand{\LexEnt}[3]{##1; \syncat{##3}}
% \renewcommand{\LexEntTwoLine}[3]{\renewcommand{\arraystretch}{.8}%
% \begin{array}[b]{l} ##1;  \\ \syncat{##3} \end{array}}
% \renewcommand{\LexEntThreeLine}[3]{\renewcommand{\arraystretch}{.8}%
% \begin{array}[b]{l} ##1; \\ \syncat{##3} \end{array}}}
%
%
% \newcommand{\NoSemVar}{%
% \renewcommand{\LexEnt}[3]{##1; \syncat{##3}}
% \renewcommand{\LexEntTwoLine}[3]{\renewcommand{\arraystretch}{.8}%
% \begin{array}{l} ##1;  \\ \syncat{##3} \end{array}}
% \renewcommand{\LexEntThreeLine}[3]{\renewcommand{\arraystretch}{.8}%
% \begin{array}{l} ##1; \\ \syncat{##3} \end{array}}}
%
% \newcommand{\vs}{\raisebox{.05em}{\ensuremath{\upharpoonright}}}
%
% \newcommand{\AXX}[1]{\raisebox{-7mm}{\ensuremath{#1}}}
%
% \newcommand{\hypml}[2]{\left[\!\!#1\!\!\right]^{#2}}
%
% \newcommand{\alt}[2]{$\left\{\begin{array}{c}
% \hskip-.7ex\textrm{#1}\hskip-.7ex \\
% \hskip-.7ex\textrm{#2}\hskip-.7ex
%         \end{array}
% \right\}$}
%
% \newcommand{\altalt}[2]{\{#1/#2\}}
%
%
%
% \newcommand{\altt}[3]{$\left\{\begin{array}{c}
% \hskip-.7ex\textrm{#1}\hskip-.7ex \\
% \hskip-.7ex\textrm{#2}\hskip-.7ex \\
% \hskip-.7ex\textrm{#3}\hskip-.7ex
% \end{array}
% \right\}$}
%
% \newcommand{\alttt}[4]{$\left\{\begin{array}{c}
% \hskip-.7ex\textrm{#1}\hskip-.7ex \\
% \hskip-.7ex\textrm{#2}\hskip-.7ex \\
% \hskip-.7ex\textrm{#3}\hskip-.7ex \\
% \hskip-.7ex\textrm{#4}\hskip-.7ex
% \end{array}
% \right\}$}
%
%
% %%%%for bussproof
%
% \def\defaultHypSeparation{\hskip0.1in}
% \def\ScoreOverhang{0pt}
%
%
% %%\newcommand{\MultiLine}[1]{\renewcommand{\arraystretch}{.8}%
% %%\ensuremath{\begin{array}{l} #1 \end{array}}}
%
\newcommand{\MultiLine}[1]{\renewcommand{\arraystretch}{.8}%
\ensuremath{\begin{array}[b]{l} #1 \end{array}}}

%
%
% \newcommand{\MultiLineMod}[1]{%
% \ensuremath{\begin{array}[t]{l} #1 \end{array}}}
%
%
% %%%%%\AFourMargin
% %%\JLSubmissionMargin
%
% %%\setlength\topmargin{-1cm}
% %%\setlength\textheight{23cm}
% %%%%%\setlength\textwidth{13.5cm}
%
% %\setstretch{1.2}
%
% % not used \newcommand{\hyp}[2]{[ #]^{#2}}
%
\newcommand{\LexEnt}[3]{#1; \ensuremath{#2}; \syncat{#3}}
%
% \newcommand{\LexEntTwoLine}[3]{\renewcommand{\arraystretch}{.8}%
% \begin{array}[b]{l} #1; \\ \ensuremath{#2};  \syncat{#3} \end{array}}
%
% \newcommand{\LexEntThreeLine}[3]{\renewcommand{\arraystretch}{.8}%
% \begin{array}[b]{l} #1; \\ \ensuremath{#2}; \\ \syncat{#3} \end{array}}
%
%
% \newcommand{\LexEntFiveLine}[5]{\renewcommand{\arraystretch}{.8}%
% \begin{array}{l} #1 \\ #2; \\ \ensuremath{#3} \\ \ensuremath{#4}; \\ \syncat{#5} \end{array}}
%
%
% \newcommand{\LexEntFourLine}[4]{\renewcommand{\arraystretch}{.8}%
% \begin{array}{l} \pt{#1} \\ \pt{#2}; \\ \syncat{#4} \end{array}}
%
% \newcommand{\ManySomething}{\renewcommand{\arraystretch}{.8}%
% \raisebox{-3mm}{\begin{array}[b]{c} \vdots \,\,\,\,\,\, \vdots \\
% \vdots \,\,\,\,\,\, \vdots \end{array}}}
%
%
% \newcommand{\lemma}[1]{\renewcommand{\arraystretch}{.8}%
% \begin{array}[b]{c} \vdots \,\,\,\,\,\, \vdots \\ #1 \end{array}}
%
% \newcommand{\lemmarev}[1]{\renewcommand{\arraystretch}{.8}%
% \begin{array}[b]{c} #1 \\ \vdots \,\,\,\,\,\, \vdots \end{array}}
%
% \newcommand{\p}{\ensuremath{\upvarphi}}
%
% \newcommand{\Not}{\leavevmode\llap{\textbf{\smc{NOT:}} }}
%
% \newcommand{\Conj}{\fs{\bsp{\mathit{X}}{\mathit{X}}}{\mathit{X}}}
% \newcommand{\ConjY}{\fs{\bsp{\mathit{Y}}{\mathit{Y}}}{\mathit{Y}}}
% \newcommand{\sameLE}{\dna{(\upa{(\upa{S}{\mathit{X}})}{NP})}{(\upa{S}{\mathit{X}})}}
%
% \newcommand{\derivcenter}[2][1.1]{%
% \SetInfLen
% \attop{\vskip3ex
% \resizebox{#1\linewidth}{!}{\hskip-#1in
% #2}}}
%
% \newcommand{\derivcenterAlt}[2][.98]{%
% \SetInfLen
% \attop{\vskip3ex
% \resizebox{#1\linewidth}{!}{\hskip-#1in \hskip.5in
% #2}}\vspace{.5ex}}
%
% \renewcommand{\O}{\circ}  Do not recommand!
\newcommand{\BobsO}{\circ}
%
% \newcommand{\derivcenterMod}[2][1.1]{%
% \renewcommand{\LexEntThreeLine}[3]{\renewcommand{\arraystretch}{.8}%
% \raisebox{.4ex}{\ensuremath{\begin{array}{l} ##1; \\ \ensuremath{##2}; \\ \syncat{##3} \end{array}}}}
% \SetInfLen
% \attop{\vskip3ex
% \resizebox{#1\linewidth}{!}{\hskip-#1in
% #2}}}
%
%
% \newcommand{\shortarrow}{\xspace\hskip-1.2ex\scalebox{.5}[1]{\ensuremath{\bm{\rightarrow}}}\hskip-.5ex\xspace}
%
% \newcommand{\SemInt}[1]{\mbox{$[\![ \textrm{#1} ]\!]$}}
%
% \def\maru#1{{\ooalign{\hfil
%   \ifnum#1>999 \resizebox{.25\width}{\height}{#1}\else%
%   \ifnum#1>99 \resizebox{.33\width}{\height}{#1}\else%
%   \ifnum#1>9 \resizebox{.5\width}{\height}{#1}\else #1%
%   \fi\fi\fi%
% \/\hfil\crcr%
% \raise.167ex\hbox{\mathhexbox20D}}}}
%
\newcommand{\HypSpace}{\hskip-.8ex}
\newcommand{\RaiseHeight}{\raisebox{2.2ex}}
% \newcommand{\RaiseHeightLess}{\raisebox{1ex}}
%
% \newcommand{\fW}{\ensuremath{\mathfrak{W}}}
%
\newcommand{\ThreeColHyp}[1]{\RaiseHeight{\Bigg[}\HypSpace#1\HypSpace\RaiseHeight{\Bigg]}}
% \newcommand{\TwoColHyp}[1]{\RaiseHeightLess{\Big[}\HypSpace#1\HypSpace\RaiseHeightLess{\Big]}}
%
%
% %\newcommand{\maskref}[1]{\textsl{\textbf{[reference omitted for refereeing]}}}
% \newcommand{\maskref}[1]{#1}
%
% \newcommand{\DerivSize}{\small}
% \newcommand{\AppDerivSize}{\footnotesize}
%
% \renewcommand{\sem}{\ensuremath}


% \newcommand{\greekp}{{\color{green}π}}
% \newcommand{\greekr}{{\color{green}\textrho}}
% \newcommand{\greeks}{{\color{green}\textsigma}}
\newcommand{\ptfont}[1]{\texttt{#1}}                % what does ptfont do? Where is it defined?
                                % Question
%\newcommand{\ptfont}{\ttfamily}

%\newcommand{\grey}{\color{gray}}
\newcommand{\grey}[1]{\colorbox{mycolor}{#1}}
\definecolor{mycolor}{gray}{0.8}

\newcommand{\gap}{\longrule}
\newcommand{\gp}{\gap}
\newcommand{\vs}{\raisebox{.05em}{\ensuremath{\upharpoonright}}}
% \newcommand{\sub}[1]{\textsubscript[#1]}
\newcommand{\E}{\emph}
\newcommand{\B}{\textbf}
\newcommand{\f}{{\color{green}f}}  % Question what does f do? It does not have any output in the
                                % original PDF
%\newcommand{\Lemma}{{\color{pink}Lemma}}
\newcommand{\Lemma}{\ensuremath{\vdots\hskip.5cm\vdots}\noLine}

%\newcommand{\calP}{{\color{pink}calP}} % Sebastian
\newcommand{\calP}{\ensuremath{\mathcal{P}}}


\newcommand{\maru}[1]{\ooalign{\hfil#1\/\hfil\crcr
      \raise.05ex\hbox{\LARGE\mathhexbox20D}}}


%\newcommand{\sem}[2][M\!,g]{\mbox{$[\![ \mathrm{#2} ]\!]^{#1}$}}
\newcommand{\sem}{\ensuremath}

%
\newcommand{\trns}[1]{\textbf{#1}\xspace}

\newcommand{\bs}{{\textbackslash}}
\newcommand{\bsl}{{\bs}}


\newcommand{\fb}[1]{\textsubscript{#1}}

\newcommand{\syncat}[1]{\ensuremath{\mathrm{#1}}}
\newcommand{\term}[1]{\textit{#1}}
\newcommand{\LemmaAlt}{\ensuremath{\vdots\hskip.5cm\vdots}}

%\renewcommand{\O}{ø}

   %% hyphenation points for line breaks
%% Normally, automatic hyphenation in LaTeX is very good
%% If a word is mis-hyphenated, add it to this file
%%
%% add information to TeX file before \begin{document} with:
%% %% hyphenation points for line breaks
%% Normally, automatic hyphenation in LaTeX is very good
%% If a word is mis-hyphenated, add it to this file
%%
%% add information to TeX file before \begin{document} with:
%% %% hyphenation points for line breaks
%% Normally, automatic hyphenation in LaTeX is very good
%% If a word is mis-hyphenated, add it to this file
%%
%% add information to TeX file before \begin{document} with:
%% \include{localhyphenation}
\hyphenation{
    par-a-digm
}

\hyphenation{
    par-a-digm
}

\hyphenation{
    par-a-digm
}

   \boolfalse{bookcompile}
   \togglepaper[23]%%chapternumber
}{}

   
\begin{document}

\maketitle

\itdopt{redo examples, there are dangling single quotes}

\section{Place}

Specification of place is accomplished with various formal devices in languages, most commonly demonstratives, adverbs, case-marked nouns, adpositional phrases, and adverbial clauses. An examination of spontaneous speech in Mohawk, however, a language of the Iroquoian family spoken in northeastern North America in Quebec, New York State, and Ontario, shows that there may be more, and that things might be more interesting.

Mohawk is polysynthetic and head-marking. There are three lexical categories defined in terms of their morphological structure: particles, nouns, and verbs. Location and direction can be specified with all three. Since the language is head-marking, there is no nominal case and no adpositional phrases. Locative and directional relations are understood from verbal semantics and verbal morphology. In clauses with verbs meaning such things as `live' or `sit', a term designating a place is likely to be in a locative relation. In clauses with verbs meaning `go' or `insert', a term designating a place is likely to indicate a source or goal. A cislocative prefix on verbs indicates location at an unmarked location or direction toward a reference point: `hither'. A translocative prefix marks a situation at a distant location of direction away from a reference point: `thither'. Place can also be specified by adverbial clauses. If, however, we observe what speakers actually do as they are choosing what to say and how to say it, we see that there are more intricate constructions used for designating place. 
\section{Basic grammatical structure: Managing the flow of information}
 Both morphological and syntactic constructions in Mohawk provide speakers with choices for how they shape the flow of information. Significant new referents are often introduced in a word or phrase, then backgrounded in subsequent speech as morphological elements of verbs. The pattern can be seen with noun incorporation, a kind of noun-verb compounding. An example is in (1). Some friends were traveling to another community, a trip which involves crossing the Canada-U.S. border. The border was introduced here in the phrase \emph{tsi karistì:seron}, literally `place metal is dragged', coined from the time the border was established by surveyors. \vspace{5mm}

(1) Separate word (Watshenní:ne' Sawyer, speaker)

\begin{tabular}{ll}
     Oh & neniá:wen’ne’\\
     oh & n-en-iaw-en’n-e’\\
     how & PRT-FUT-N-happen-PFV \\
     \multicolumn{2}{l}{What's going to happen }
\end{tabular}


 \vspace{5mm}
\bigskip

\begin{tabular}{lllll}
    ki: & nó:nen & ieniákwawe’\\
    ki: & n=onen & i-en-iakwa-w-e’\\
    this & ART=when & TRANSLOCATIVE-FUT-1EXCL.PL.AGT-arrive-PFV \\
    when we arrive
\end{tabular}

\bigskip

\begin{tabular}{lllll}
    tsi & karistì:seron? \\
    tsi & ka-rist-i’ser-on? \\
    place & N-metal-drag-STATIVE \\
    place & metal is dragged \\
    at the border?’	\\
\end{tabular}

\bigskip

The party crossed without incident, however, and felt relieved. At this point the term for border was simply incorporated into the verb `cross', since it was already part of the scene.\bigskip

(2) Incorporated noun (Watshenní:ne' Sawyer, speaker)

\begin{tabular}{lllll} 

    \emph{Nek tsi} & \emph{ó:nen} \\
    nek tsi & onen \\
    but & now \\
\end{tabular}

\vspace{5mm}
\begin{tabular}{ll} 
\emph{ia'tiakwaristí:ia'ke'}&{[}. . .{]}\\
i-a'-t-iakwa-rist-í:ia'k-e' \\
\end{tabular}

\begin{tabular}{l} 
\textsc{pfv}\textsc{translocative-fact-1excl.pl.agt}-metal-cross-\textsc{pfv}\\
\end{tabular}

\begin{tabular}{ll} 
there we border crossed\\
`But after we crossed the border&{[}we felt a little better again{]}.'
\end{tabular}\bigskip

There is no syntactically-based constituent order. After various orienting and connective markers, constituents occur essentially in descending order of newsworthiness at the moment. When the friends were returning home, they had to cross back over the border.
\bigskip

(3) Constituent order (Watshenní:ne' Sawyer, speaker)

\begin{tabular}{l}
\emph{Karíhton.}\\

`Police.'\\
\end{tabular}\bigskip

\begin{tabular}{lllllll}
\emph{Ó:nen}&\emph{sénha'} & \emph{é:so'} & \emph{rá:ti} & \emph{karíhton}&\emph{tho}&\emph{thó:nete',}\\
now & more & much & of them & police & there & there they stand\\
\multicolumn{7}{l}{There were more police there}
\\

\end{tabular}\bigskip


\begin{tabular}{l}
\emph{ahskwákta'.}\\

near bridge place\\

near the bridge.'\\
\end{tabular}\bigskip

\begin{tabular}{lll}

\emph{Akwé:kon}&\emph{káhonre'}&\emph{ratíhawe'.}\\

all& gun & they carry\\
\multicolumn{3}{l}{`They had guns.'}\\
\end{tabular}\bigskip

\begin{tabular}{lllll}
\emph{Akwé:kon} & \emph{ne:} & \emph{enhóntken'se'} & \emph{ne} &\emph{sà:sere.}\\

all&it is& they will inspect&the&your car\\
\multicolumn{3}{l}{`They were going to inspect the car.'}
\end{tabular}\bigskip

By the second line, the police had already been introduced. The most important information was that there were more than before, so `many more' preceded the term for the police. The location by the bridge was incidental. The point of the fourth line was the guns; the verb `carry' simply served to bring them onto the scene The point of the fifth line was the inspection; the car was already part of the scene. 
Similar patterns can be seen in focus constructions, especially clear in questions and answers. A mother and daughter were preparing to go out to a meeting. The verb `it is not cold' was the main point of the question. The particle `outdoors' accordingly occurred after the verb.

\bigskip` 

(4) Question and answer (Grace Curotte, Audrey Curotte, speakers)\\

\begin{tabular}{lllll}
GC & \emph{Iáh} & \emph{ken} & \emph{teiowísto} & \emph{nátste?}\\

& not & \textsc{q} & is it cold & outdoors\\
\multicolumn{4}{l}{}
`Isn\textquotesingle t it cold outside?'
\end{tabular}

\bigskip`
\begin{tabular}{ll}
AC &   \emph{En:, iowísto.}\\
& \multicolumn{1}{l}{`Yes, it is cold.'}


\end{tabular}\bigskip`

On another occasion, a speaker noted that the church bells were loud. Here the particle `outside' occurred at the beginning of her statement; it was more important to her message than the sitting.\bigskip`

(6) Word order (Doris White, speaker)\\
\begin{tabular}{lll}
\emph{Átste} & \emph{ki'} & \emph{tkitskó:tahkwe' . . .}\\

outside & in fact & there I was sitting\\
\multicolumn{3}{l}{`In fact I was sitting outside.'}

\end{tabular}\bigskip`

\section{Prosodic structure}

Mohawk speakers, like those of other languages, tend to produce speech in spurts, one intonation unit or prosodic phrase at a time, each conveying one significant new idea or focus of consciousness. The speaker in (7) remarked, `I stop by the old folks' home sometimes.' Each line of transcription below and henceforth represents a single intonation unit, characterized by a coherent pitch contour. The punctuation reflects the prosody, with commas for a non-terminal contour, and periods for a terminal contour.\bigskip`

(7) Intonation units (Leo Diabo, speaker)\\

\begin{tabular}{llll}
\emph{Né:} & \emph{ki:} & \emph{ni'} & \emph{kí:kén:,}\\
it is & this & myself &this\\
\multicolumn{4}{l}{`I myself,'}

\end{tabular}\bigskip

\begin{tabular}{ll}
\emph{kenh} & \emph{nekwá:ti,}\\
there & the side\\
\multicolumn{2}{l}{`over that way,'}
\end{tabular}\bigskip

\begin{tabular}{ll}
\emph{shionsakkwátho'} & \emph{ki:kén:,}\\
when I stop back by over there & this\\
\multicolumn{2}{l}{`I stop by there,'}
\end{tabular}\bigskip


\begin{tabular}{ll}
\emph{thati'teròn:ton'} & \emph{ki:kén: um,}\\

there they reside variously & this\\
\multicolumn{2}{l}{`they live there,'}
\end{tabular}\bigskip

\begin{tabular}{lllll}
\emph{ratikstenhokòn:'a} & \emph{tho} & \emph{nekwá:} &\emph{ia'kkwá:tho'} & \emph{ostòn:ha,}\\

they are old variously & there & side & there I stop by & a bit\\
\multicolumn{5}{l}{`I stop by the old folks place a bit,'}
\end{tabular}\bigskip

\begin{tabular}{lllll}
\emph{sewatié:rens.}\\
`sometimes.'
\end{tabular}\bigskip

The next to the last intonation unit contains more words than some of the others: the action `I stop by there' was not new information. 
Constructions involving demonstratives are exploited pervasively by speakers to manage the flow of information over intonation units and sentences.

\section{Structuring the flow within sentences}

Cataphoric and anaphoric demonstrative constructions permit speakers to package one idea at a time prosodically, ordering them according to their significance at the moment, while retaining their coherence. Cataphoric demonstratives in one intonation unit can serve as placeholders, signaling that further elaboration is to follow. The speaker in (8) was explaining that her mother was going to go out and leave her in charge of watching her younger brother. In the first intonation unit here she noted that she(the mother) made him a pallet (a single idea conveyed in a single word with incorporated noun), and in the second, she supplied the location.
\bigskip

(8) Cataphoric demonstrative (Sadie Smoke Peters, speaker)\\

\begin{tabular}{llll}
\emph{Thó} & \emph{ki'} & \emph{wahonwéntskaron'se'} & \emph{ki:kén:,}\\

there & just & she made him a pallet & this\\
\end{tabular}\bigskip

\begin{tabular}{ll}
\emph{tsi} & \emph{iotékha'.}\\

place & it burns\\
\multicolumn{2}{l}{`She made him a pallet by the fire.'}
\end{tabular}\bigskip

A similar construction is in (9). Friends were discussing a funeral, noting that the deceased was not left at the mortuary.\\

(9) Cataphoric demonstrative (Josephine Kaieríthon Horne, speaker)\\

\begin{tabular}{lll}
\emph{Thó} & \emph{ki' iá:ken'} &\emph{iahonwaia'ténhawe',}\\

there & just they say & away they bodily took him\\
\multicolumn{3}{l}{`They just took him}
\end{tabular}\bigskip

\begin{tabular}{lll}
\emph{tsi-\/-} & \emph{tsi} & \emph{thonónhsote', . . .}\\

place-\/- & place & there his house stands\\
\multicolumn{3}{l}{to his {[}father's{]} house.'}
\end{tabular}\bigskip

The speaker in (10) was relating that a man left his house in the forest and was walking to town. The stopping was the significant event in the sequence at this point, and it occurred first. The demonstrative \emph{tho}, however, signaled that specification of the location was to follow.\bigskip


(10) Cataphoric demonstrative (Annette Kaia'titáhkhe' Jacobs,

speaker)\\

\begin{tabular}{llll}
\emph{Tánon'} & \emph{thó} & \emph{ia'thá:ta'ne'} &\emph{ thi:kén:,}\\

and & there & over there he stopped & that one
\end{tabular}\bigskip

\begin{tabular}{l}
\emph{ononhsatokenhthì:ke,}\\

holy house place
\end{tabular}\bigskip

\begin{tabular}{ll}
\emph{ísi'} & \emph{na'oháhati.}\\

yonder & it is on the other side of the road\\
\multicolumn{2}{l}{`And he stopped across the road from the church.'}
\end{tabular}\bigskip

Demonstratives are also used anaphorically to link an intonation unit to a preceding unit specifying place. The speaker in (11) opened his statement by setting the scene along the river.\bigskip


(11) Anaphoric demonstrative (Joe Awenhráthon Deer, speaker)\\

\begin{tabular}{lll}
\emph{Kí:ken} & \emph{atsa'któntie'} & \emph{wáhi',}\\

this & along the river & you know,
\end{tabular}\bigskip

\begin{tabular}{ll}
\emph{thó}  & \emph{thonathéhtaien',}\\

there & there they garden have,
\end{tabular}\bigskip

\begin{tabular}{ll}
\emph{ónhka'k,} & \emph{takwáh.}\\

just somebody & or other\\
\multicolumn{2}{l}{`Somebody has a garden along the river.'}
\end{tabular}\bigskip

A mother cleaning her son's bedroom found something dubious under her son's bed. She later recounted that when asked about it, the son responded that his teacher had told him to put the object someplace dark and warm. Here the location was the main point of his comment, so it occurred at the beginning of his statement. The demonstrative in the following intonation unit linked it to the preceding.\bigskip


(12) Anaphoric demonstrative (Marie Kahentorehtha' Cross, speaker)\\

\begin{tabular}{llll}
\emph{Rakhró:ri} & \emph{se'} & \emph{tsi nón:} & \emph{tetiò:karas}\\

he told me & just & place & it is dark
\end{tabular}\bigskip

\begin{tabular}{lllll}
\emph{tánon'} & \emph{wahèn:ron'} & \emph{tsi nón:} & \emph{ne} & \emph{io'taríhen,}\\
and & he said & place & the & it is warm
\end{tabular}\bigskip

\begin{tabular}{llll}
\emph{thó} &  \emph{ki'} & \emph{nón:} & \emph{nénhsien'.}\\
there & in fact & place & there you will lay it\\
\multicolumn{4}{l}{`He told me to put it where it's dark and warm.'}
\end{tabular}\bigskip

\section{Distributing information across sentences}

Demonstrative constructions are also used pervasively to give prominence to important ideas in separate sentences rather than subordinate clauses, while retaining coherence. The speaker cited earlier in (8) continued her story about her babysitting adventures noting that her mother laid her younger brother near the fireplace in front of the fire screen. She introduced the fire screen, a fairly complex term, in one sentence, then her mother's action in another, tying it to the preceding with the demonstrative \emph{tho} `there.\bigskip

(13) Discourse coherence (Sadie Smoke Peters, speaker)\\

\begin{tabular}{llllll}

\emph{Nek tsi} & \emph{ohén:ton} & \emph{ki:} & \emph{ne'} & \emph{kà:niote'} & \emph{ne:,}\\

but & area in front & this & it is & it stands & it is
\end{tabular}\bigskip

\begin{tabular}{llllll}
\emph{tóhsa'} & \emph{ki:} & \emph{sótsi} & \emph{taon-} & \emph{taonré:ni'} & \emph{kí:ken,}\\

not & this &  too & not would-&  not would it spread & this
\end{tabular}\bigskip

\begin{tabular}{lllll}
\emph{katsiénha'.}\\

ember.
\end{tabular}\bigskip

\begin{tabular}{lll}
\emph{Thó} & \emph{wahonwaia'kión:nite'} & \emph{kí:ken, . . .}\\

there & she bodily laid him & this\\
\multicolumn{3}{l}{`She laid him in front of the firescreen.'}
\end{tabular}\bigskip

The speaker in (14) was describing a pow wow ground, where people set up tables to sell jewelry. She introduced the tables in a separate sentence, as a separate element of her description of the area, not simply a location for the jewelry. Coherence was established by the demonstrative \emph{tho} `there'.\bigskip

(14) Discourse coherence (Annette Kaia'titáhkhe' Jacobs, speaker)\\

\begin{tabular}{lllll}
\emph{Tánon'} & \emph{shes'} & \emph{ò:ni'} &\emph{watekhwahra'tsheró:ton'} & \emph{um,}\\

and & then & also & it was tables standing here and there
\end{tabular}\bigskip

\begin{tabular}{lllll}
\emph{ata'èn:rakon.}\\

fence interior.
\end{tabular}\bigskip

\begin{tabular}{llllll}
\emph{Tánon'} & \emph{thó} & \emph{shes} & \emph{non:}& \emph{kahrónnion'} & \emph{ki:kén:,}\\

And & there  & then &  place & it is set up here and there & this
\end{tabular}\bigskip

\begin{tabular}{ll}
\emph{tsi nahò:ten'} &  \emph{rotihsa'ánion'.}\\
what & they have made variously\\
\multicolumn{2}{l}{`And people set out things they had made on tables in the yard.'}
\end{tabular}\bigskip

The speaker in (15) was describing the house she grew up in, which had attached quarters for her grandmother. She introduced the addition in one sentence. She then mentioned that that arrangement was customary in those days. Then, in a separate sentence, she said that her grandmother lived there, with a demonstrative link to the preceding.\bigskip

(15) Discourse coherence (Watshenní:ne' Sawyer, speaker)\\

\begin{tabular}{lllll}

\emph{Kenh} & \emph{neká:} & \emph{iononhsanontá:kon.}\\

there & side & the house had an addition
\end{tabular}\bigskip

\begin{tabular}{lllll}
\emph{Akwé: shens} & \emph{tho} & \emph{niiohtòn:ne'} & \emph{tsi náhe'.}\\

all customarily & there & it was so  & back then.
\end{tabular}\bigskip

\begin{tabular}{lllll}
\emph{Thó} & \emph{iè:teron'} &  \emph{ne,}\\

there & she lived & the
\end{tabular}\bigskip

\begin{tabular}{lllll}
\emph{ne:'ne} & \emph{Tóta.}\\
the one & Grandma.
\end{tabular}\bigskip


The same speaker described earlier times when people would go out to an island on the other side of a bridge. Here the introduction of the island was complex, presented in one sentence, with an anaphoric demonstrative. The going was presented in a second, linked to the first with another anaphoric demonstrative.\bigskip

(16) Discourse coherence (Watshenní:ne' Sawyer, speaker)\\

\begin{tabular}{llllll}
\emph{Né:} & \emph{kí:} & \emph{tsi} & \emph{tiowè:note'} & \emph{tho} & \emph{nón:,}\\

it is  & this & place & there it island stands & there & place
\end{tabular}\bigskip

\begin{tabular}{lllll}
\emph{tho} & \emph{nón:} & \emph{ieiorhárhon} & \emph{ki:kén:,}\\

there & place & over there it is moored & this
\end{tabular}\bigskip

\begin{tabular}{lllll}
\emph{áhskwa'.}\\

bridge
\end{tabular}\bigskip

\begin{tabular}{lllll}
\emph{Thó } & \emph{ki'} & \emph{ni'} & \emph{tsi} & \emph{ ieiakwéhtha'.}\\

there & in fact &  we  & place & over there we go\\
\multicolumn{5}{l}{`We would go to the island connected by a bridge.'}
\end{tabular}\bigskip

The speaker in (17) first established a location, then in a second sentence introduced chickens with an anaphoric demonstrative linking this to the previous sentence.\bigskip


(17) Discourse coherence (Joe Awenhráthon Deer, speaker)\\

\begin{tabular}{lllll}
\emph{Kawinehthà:ke} & \emph{ohnà:ken} & \emph{énska} & \emph{shé:} &\emph{kanónhsote'.}\\

Kawinehtha's place & behind & one & still it house stands
\end{tabular}\bigskip

\begin{tabular}{lllll}
\emph{Tho} & \emph{se'} & \emph{non:} & \emph{konti'terontónhkwe'} & \emph{ki:kén:,}\\

there & then & place & they used to live & these
\end{tabular}\bigskip

\begin{tabular}{lllll}
\emph{kítkit.}\\

chicken.\\

`The chickens used to live in the house behind Kawinéhtha's.'
\end{tabular}\bigskip


\section{Conclusion}

Over the past several decades Dan Everett has provided thought-provoking ideas on the nature of language universals, questions of innateness, and the role of culture in shaping language. The role of culture appears to have shaped the Mohawk language in striking ways. There are long, well-documented traditions among the Iroquois of appreciation and cultivation of linguistic virtuosity in all genres, from formal oratory, through traditional tales, anecdotes, jokes, and snappy comebacks. Mohawk people often comment on the skill of particular speakers and visibly delight in it. This cultural value is reflected pervasively in masterful attention to shaping the flow of information.

Looking at what speakers actually do on a daily basis, in what they choose to say and how they choose to frame it, rather than stopping at translations of isolated sentences from a contact language, promises to tell us more and more about the deeper and cognitive and cultural forces shaping languages.

\section*{Abbreviations}

\textsc{agt = grammatical agent, art = article, excl = exclusive, fut =
future, n = neuter, pat = grammatical patient, pfv = perfective, prt =
partitive}

\printbibliography[heading=subbibliography,notkeyword=this]


\end{document}

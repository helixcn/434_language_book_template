\documentclass[output=paper,colorlinks,citecolor=brown]{langscibook}
\IfFileExists{../localcommands.tex}{
   \addbibresource{../localbibliography.bib}
   \newcommand{\orcid}[1]{}

\usepackage{orcidlink}

\usepackage{tabularx,multicol}
\usepackage{url}
\urlstyle{same}


\usepackage{langsci-optional}
\usepackage{langsci-lgr}
\usepackage{langsci-gb4e}

% for texlive 2022
\usepackage{langsci-branding} 

% Müller


% \usepackage{biblatex-series-number-checks}

% \usepackage{eng-date}

% \usepackage{german}%% Das Buch ist nicht deutsch. Hör auf, solche Sachen zu laden


% \usepackage{tikz-dependency}
% \usepackage{tikz}
% \usepackage{tikz-qtree}

% \usepackage{hologo}

% 3_pullum.tex

% This does not work complains about recommanding epsilon
% We use a special adapted version, which is in the repro.
\usepackage{langsci-textipa}



% 8_levine

\usepackage{./styles/lg-macro2}
\usepackage{bm}
\usepackage{umoline}
\usepackage{pifont}
\usepackage{pstricks,pst-node,pst-tree}
\usepackage{ulem}
\usepackage{mathrsfs}
\usepackage{bussproofs}




%\usepackage{tikz,tikz-qtree}

%\usepackage{gb4e0}
%\noautomath

%\usepackage[letterpaper,margin=1.2in]{geometry}



% 14_kornai

%\usepackage{xypic} % seems not to be needed
% \usepackage[matrix,arrow]{xy}
%\usepackage{amsmath}
% \usepackage{subcaption}
% \usepackage{wrapfig}


   
\SetupAffiliations{output in groups = false,
                   orcid placement = after,
                   separator between two = {\bigskip\\},
                   separator between multiple = {\bigskip\\},
                   separator between final two = {\bigskip\\}
                   }

% ORCIDs in langsci-affiliations 
\definecolor{orcidlogocol}{cmyk}{0,0,0,1}
\RenewDocumentCommand{\LinkToORCIDinAffiliations}{ +m }
  {%
    \,\orcidlink{#1}%
  }


\makeatletter
\let\thetitle\@title
\let\theauthor\@author
\makeatother

\newcommand{\togglepaper}[1][0]{
   \bibliography{../localbibliography}
   \papernote{\scriptsize\normalfont
     \theauthor.
     \titleTemp.
     To appear in:
     E. Di Tor \& Herr Rausgeberin (ed.).
     Booktitle in localcommands.tex.
     Berlin: Language Science Press. [preliminary page numbering]
   }
   \pagenumbering{roman}
   \setcounter{chapter}{#1}
   \addtocounter{chapter}{-1}
}

\newbool{bookcompile}
\booltrue{bookcompile}
\newcommand{\bookorchapter}[2]{\ifbool{bookcompile}{#1}{#2}}


% Cite and cross-reference other chapters
\newcommand{\crossrefchaptert}[2][]{\citet*[#1]{chapters/#2}, Chapter~\ref{chap-#2} of this volume} 
\newcommand{\crossrefchapterp}[2][]{(\citealp*[#1]{chapters/#2}, Chapter~\ref{chap-#2} of this volume)}
\newcommand{\crossrefchapteralt}[2][]{\citealt*[#1]{chapters/#2}, Chapter~\ref{chap-#2} of this volume}
\newcommand{\crossrefchapteralp}[2][]{\citealp*[#1]{chapters/#2}, Chapter~\ref{chap-#2} of this volume}

\newcommand{\crossrefcitet}[2][]{\citet*[#1]{chapters/#2}} 
\newcommand{\crossrefcitep}[2][]{\citep*[#1]{chapters/#2}}
\newcommand{\crossrefcitealt}[2][]{\citealt*[#1]{chapters/#2}}
\newcommand{\crossrefcitealp}[2][]{\citealp*[#1]{chapters/#2}}


\newcommand{\sub}[1]{\textsubscript{\scriptsize\textrm{#1}}}

% Müller

\newcommand{\page}{}

\let\citew\citet

\def\underRevision{Revise and resubmit}

\let\textbfemph\emph

\newcommand{\todostefan}[1]{\todo[color=orange!80]{\footnotesize #1}\xspace}
\newcommand{\todosatz}[1]{\todo[color=red!40]{\footnotesize #1}\xspace}

\newcommand{\inlinetodostefan}[1]{\todo[color=green!40,inline]{\footnotesize #1}\xspace}

\newcommand{\inlinetodoopt}[1]{\todo[color=green!40,inline]{\footnotesize #1}\xspace}
\newcommand{\inlinetodoobl}[1]{\todo[color=red!40,inline]{\footnotesize #1}\xspace}

\newcommand{\itd}[1]{\inlinetodoobl{#1}}
\newcommand{\itdobl}[1]{\inlinetodoobl{#1}}
\newcommand{\itdopt}[1]{\inlinetodoopt{#1}}

\newcommand{\addpages}{\todostefan{add pages}}

%% % taken from https://tex.stackexchange.com/a/95079/18561
\newbox\usefulbox

\makeatletter
\def\getslant #1{\strip@pt\fontdimen1 #1}

\def\skoverline #1{\mathchoice
 {{\setbox\usefulbox=\hbox{$\m@th\displaystyle #1$}%
    \dimen@ \getslant\the\textfont\symletters \ht\usefulbox
    \divide\dimen@ \tw@ 
    \kern\dimen@ 
    \overline{\kern-\dimen@ \box\usefulbox\kern\dimen@ }\kern-\dimen@ }}
 {{\setbox\usefulbox=\hbox{$\m@th\textstyle #1$}%
    \dimen@ \getslant\the\textfont\symletters \ht\usefulbox
    \divide\dimen@ \tw@ 
    \kern\dimen@ 
    \overline{\kern-\dimen@ \box\usefulbox\kern\dimen@ }\kern-\dimen@ }}
 {{\setbox\usefulbox=\hbox{$\m@th\scriptstyle #1$}%
    \dimen@ \getslant\the\scriptfont\symletters \ht\usefulbox
    \divide\dimen@ \tw@ 
    \kern\dimen@ 
    \overline{\kern-\dimen@ \box\usefulbox\kern\dimen@ }\kern-\dimen@ }}
 {{\setbox\usefulbox=\hbox{$\m@th\scriptscriptstyle #1$}%
    \dimen@ \getslant\the\scriptscriptfont\symletters \ht\usefulbox
    \divide\dimen@ \tw@ 
    \kern\dimen@ 
    \overline{\kern-\dimen@ \box\usefulbox\kern\dimen@ }\kern-\dimen@ }}%
 {}}
\makeatother

% 1_intro.tex

% For the block quote:

\usepackage[most]{tcolorbox}
\definecolor{linequote}{RGB}{224,215,188}
\definecolor{backquote}{RGB}{249,245,233}
\newtcolorbox{myquote}[1][]{%
    enhanced, breakable, 
    size=minimal,
    frame hidden, boxrule=0pt,
    sharp corners,
    colback=backquote,
    #1
}

% 2_gibson.tex


% Example(s) Environments
% 12pt, No new-lines after example number is printed

\newcounter{examplectr}
\newcounter{fnexamplectr}

% Note: don't use subexamples in footnotes.

% This line is to overcome a bug in cmu-art style: it prints counter
% values to the aux file using \theaux... rather than using \the...
\def\theauxexamplectr{\theexamplectr}

\newcounter{subexamplectr}
\def\theauxsubexamplectr{\thesubexamplectr}
\def\theauxfnexamplectr{\thefnexamplectr}

\renewcommand{\theexamplectr}{\arabic{examplectr}}
% This command causes example numbers to appear without following periods

\renewcommand{\thefnexamplectr}{\roman{fnexamplectr}}
% This command causes example numbers to appear without following periods

\renewcommand{\thesubexamplectr}{\theexamplectr\alph{subexamplectr}}
% This command gives the number of an example and subexample as e.g. 1a, 2b

\newlength{\wdth}
\newcommand{\strike}[1]{\settowidth{\wdth}{#1}\rlap{\rule[.5ex]{\wdth}{1pt}}#1}

\newcommand{\exref}[1]{(\ref{#1})}
% This command puts reference numbers with parentheses
% surrounding them 

% The environment ``examples'' gives a list of examples, one on each line,
% numbered with a lower case alphabetic character
\newenvironment{examples}%
   { \vspace{-\baselineskip}
     \begin{list}%
     \textrm{\alph{subexamplectr}.}%
     {\usecounter{subexamplectr}
     \setlength{\topsep}{-\parskip}
     \setlength{\itemsep}{-2pt}
     \setlength{\leftmargin}{0.5in}
     \setlength{\rightmargin}{0in} } }%
   { \end{list}}

% The environment ``myexample'' outputs an arabic counter ``examplectr''
% surrounded by parentheses.
\newenvironment{myexample}
   { \vspace{20pt}
     \noindent
     \begin{minipage}{\textwidth}    % minipage environment disallows
                 % breaks across pages

     \refstepcounter{examplectr}     % step the counter and cause this
                 % section to be referenced by the
                 % counter ``examplectr''
     (\arabic{examplectr})}%
   { \vspace{20pt}
     \end{minipage}}

\newenvironment{myfnexample}
   { \vspace{2pt}
     \noindent
     \begin{minipage}{\textwidth}    % minipage environment disallows
                 % breaks across pages

     \refstepcounter{fnexamplectr}     % step the counter and cause this
                 % section to be referenced by the
                 % counter ``examplectr''
     (\roman{fnexamplectr})}%
   { \vspace{2pt}c
     \end{minipage}}
    
\newcommand*\circled[1]{\tikz[baseline=(char.base)]{
            \node[shape=circle,draw,inner sep=2pt] (char) {#1};}}



% 3_pullum.tex


% %%%  GKP:  I put in these pointless commands to kill off a bug elsewhere
% %%%        that tries to \newcommand \it (etc.) as \itshape (etc.), but
% %%%        fails because they haven't been defined.
% \newcommand{\it}{\relax}
% \newcommand{\bf}{\relax}
% \newcommand{\sc}{\relax}
% \newcommand{\rm}{\relax}

%%% GKP:  Two additional commands that I need
\newcommand{\data}[1]{\textit{#1}}
\newcommand{\blank}{\rule{1.2em}{0.5pt}}



% 8_levine

% ???
%\renewcommand{\emph}{\textit}
%\renewcommand{\em}{\it}

% not used \newcommand{\cites}[1]{\citeauthor{#1}'s~\citeyearpar{#1}}

% \renewcommand{\SetInfLen}{\setpremisesend{0pt}\setpremisesspace{10pt}\setnamespace{0pt}}

\newcommand{\pt}[1]{\ensuremath{\mathsf{#1}}}
\newcommand{\ptv}[1]{\ensuremath{\textsf{\textsl{#1}}}}

%\newcommand{\sv}[1]{\ensuremath{\bm{\mathcal{#1}}}}
\newcommand{\sv}[1]{\ensuremath{\mathcal{#1}}}

\newcommand{\sX}{\sv{X}}
\newcommand{\sF}{\sv{F}}
\newcommand{\sG}{\sv{G}}
%
% \renewcommand{\lex}{\SF}
% \renewcommand{\syncat}[1]{\ensuremath{\mathrm{#1}}}
% \newcommand{\syncatVar}[1]{\ensuremath{\mathit{#1}}}
%
% \newcommand{\RuleName}[1]{\textrm{#1}}
%
% \newcommand{\SemTyp}{\textsf{Sem}}
%
% \newcommand{\something}{\vdots\,\,\,\,\,\,\vdots}
%
% \newcommand{\pb}{\phantom{[}}
%
% \renewcommand{\E}{\ensuremath{\bm{\epsilon}}\xspace}
%
% \newcommand{\greeka}{\upalpha}
% \newcommand{\greekb}{\upbeta}
% \newcommand{\greekd}{\updelta}
\newcommand{\greekp}{\upvarphi}
\newcommand{\greekr}{\uprho}
\newcommand{\greeks}{\upsigma}
% \newcommand{\greekt}{\uptau}
% \newcommand{\greeko}{\upomega}
% \newcommand{\greekz}{\upzeta}
%
% % Do not do this!!!!!
% % \renewcommand{\labelenumi}{(\roman{enumi})}
%
% \newcommand{\upa}[2]{\ensuremath{\syncat{#1}|\syncat{#2}}}
% \newcommand{\dna}[2]{\upa{#2}{#1}}
%

%
% \newcommand{\Lemma}{\ensuremath{\vdots\hskip.5cm\vdots}\noLine}
%
% \newcommand{\la}{\ensuremath{\langle}}
% \newcommand{\ra}{\ensuremath{\rangle}}
%
% \renewcommand{\I}{\iota}
%
% \renewcommand{\sem}{\ensuremath}
%
% \newcommand{\LemmaShort}{\ensuremath{\vdots\hskip.2cm\vdots\hskip.2cm\vdots}\noLine}
% \newcommand{\LemmaShortAlt}{\ensuremath{\vdots\hskip.2cm\vdots}}
%
%
% \newcommand{\NoSem}{%
% \renewcommand{\LexEnt}[3]{##1; \syncat{##3}}
% \renewcommand{\LexEntTwoLine}[3]{\renewcommand{\arraystretch}{.8}%
% \begin{array}[b]{l} ##1;  \\ \syncat{##3} \end{array}}
% \renewcommand{\LexEntThreeLine}[3]{\renewcommand{\arraystretch}{.8}%
% \begin{array}[b]{l} ##1; \\ \syncat{##3} \end{array}}}
%
%
% \newcommand{\NoSemVar}{%
% \renewcommand{\LexEnt}[3]{##1; \syncat{##3}}
% \renewcommand{\LexEntTwoLine}[3]{\renewcommand{\arraystretch}{.8}%
% \begin{array}{l} ##1;  \\ \syncat{##3} \end{array}}
% \renewcommand{\LexEntThreeLine}[3]{\renewcommand{\arraystretch}{.8}%
% \begin{array}{l} ##1; \\ \syncat{##3} \end{array}}}
%
% \newcommand{\vs}{\raisebox{.05em}{\ensuremath{\upharpoonright}}}
%
% \newcommand{\AXX}[1]{\raisebox{-7mm}{\ensuremath{#1}}}
%
% \newcommand{\hypml}[2]{\left[\!\!#1\!\!\right]^{#2}}
%
% \newcommand{\alt}[2]{$\left\{\begin{array}{c}
% \hskip-.7ex\textrm{#1}\hskip-.7ex \\
% \hskip-.7ex\textrm{#2}\hskip-.7ex
%         \end{array}
% \right\}$}
%
% \newcommand{\altalt}[2]{\{#1/#2\}}
%
%
%
% \newcommand{\altt}[3]{$\left\{\begin{array}{c}
% \hskip-.7ex\textrm{#1}\hskip-.7ex \\
% \hskip-.7ex\textrm{#2}\hskip-.7ex \\
% \hskip-.7ex\textrm{#3}\hskip-.7ex
% \end{array}
% \right\}$}
%
% \newcommand{\alttt}[4]{$\left\{\begin{array}{c}
% \hskip-.7ex\textrm{#1}\hskip-.7ex \\
% \hskip-.7ex\textrm{#2}\hskip-.7ex \\
% \hskip-.7ex\textrm{#3}\hskip-.7ex \\
% \hskip-.7ex\textrm{#4}\hskip-.7ex
% \end{array}
% \right\}$}
%
%
% %%%%for bussproof
%
% \def\defaultHypSeparation{\hskip0.1in}
% \def\ScoreOverhang{0pt}
%
%
% %%\newcommand{\MultiLine}[1]{\renewcommand{\arraystretch}{.8}%
% %%\ensuremath{\begin{array}{l} #1 \end{array}}}
%
\newcommand{\MultiLine}[1]{\renewcommand{\arraystretch}{.8}%
\ensuremath{\begin{array}[b]{l} #1 \end{array}}}

%
%
% \newcommand{\MultiLineMod}[1]{%
% \ensuremath{\begin{array}[t]{l} #1 \end{array}}}
%
%
% %%%%%\AFourMargin
% %%\JLSubmissionMargin
%
% %%\setlength\topmargin{-1cm}
% %%\setlength\textheight{23cm}
% %%%%%\setlength\textwidth{13.5cm}
%
% %\setstretch{1.2}
%
% % not used \newcommand{\hyp}[2]{[ #]^{#2}}
%
\newcommand{\LexEnt}[3]{#1; \ensuremath{#2}; \syncat{#3}}
%
% \newcommand{\LexEntTwoLine}[3]{\renewcommand{\arraystretch}{.8}%
% \begin{array}[b]{l} #1; \\ \ensuremath{#2};  \syncat{#3} \end{array}}
%
% \newcommand{\LexEntThreeLine}[3]{\renewcommand{\arraystretch}{.8}%
% \begin{array}[b]{l} #1; \\ \ensuremath{#2}; \\ \syncat{#3} \end{array}}
%
%
% \newcommand{\LexEntFiveLine}[5]{\renewcommand{\arraystretch}{.8}%
% \begin{array}{l} #1 \\ #2; \\ \ensuremath{#3} \\ \ensuremath{#4}; \\ \syncat{#5} \end{array}}
%
%
% \newcommand{\LexEntFourLine}[4]{\renewcommand{\arraystretch}{.8}%
% \begin{array}{l} \pt{#1} \\ \pt{#2}; \\ \syncat{#4} \end{array}}
%
% \newcommand{\ManySomething}{\renewcommand{\arraystretch}{.8}%
% \raisebox{-3mm}{\begin{array}[b]{c} \vdots \,\,\,\,\,\, \vdots \\
% \vdots \,\,\,\,\,\, \vdots \end{array}}}
%
%
% \newcommand{\lemma}[1]{\renewcommand{\arraystretch}{.8}%
% \begin{array}[b]{c} \vdots \,\,\,\,\,\, \vdots \\ #1 \end{array}}
%
% \newcommand{\lemmarev}[1]{\renewcommand{\arraystretch}{.8}%
% \begin{array}[b]{c} #1 \\ \vdots \,\,\,\,\,\, \vdots \end{array}}
%
% \newcommand{\p}{\ensuremath{\upvarphi}}
%
% \newcommand{\Not}{\leavevmode\llap{\textbf{\smc{NOT:}} }}
%
% \newcommand{\Conj}{\fs{\bsp{\mathit{X}}{\mathit{X}}}{\mathit{X}}}
% \newcommand{\ConjY}{\fs{\bsp{\mathit{Y}}{\mathit{Y}}}{\mathit{Y}}}
% \newcommand{\sameLE}{\dna{(\upa{(\upa{S}{\mathit{X}})}{NP})}{(\upa{S}{\mathit{X}})}}
%
% \newcommand{\derivcenter}[2][1.1]{%
% \SetInfLen
% \attop{\vskip3ex
% \resizebox{#1\linewidth}{!}{\hskip-#1in
% #2}}}
%
% \newcommand{\derivcenterAlt}[2][.98]{%
% \SetInfLen
% \attop{\vskip3ex
% \resizebox{#1\linewidth}{!}{\hskip-#1in \hskip.5in
% #2}}\vspace{.5ex}}
%
% \renewcommand{\O}{\circ}  Do not recommand!
\newcommand{\BobsO}{\circ}
%
% \newcommand{\derivcenterMod}[2][1.1]{%
% \renewcommand{\LexEntThreeLine}[3]{\renewcommand{\arraystretch}{.8}%
% \raisebox{.4ex}{\ensuremath{\begin{array}{l} ##1; \\ \ensuremath{##2}; \\ \syncat{##3} \end{array}}}}
% \SetInfLen
% \attop{\vskip3ex
% \resizebox{#1\linewidth}{!}{\hskip-#1in
% #2}}}
%
%
% \newcommand{\shortarrow}{\xspace\hskip-1.2ex\scalebox{.5}[1]{\ensuremath{\bm{\rightarrow}}}\hskip-.5ex\xspace}
%
% \newcommand{\SemInt}[1]{\mbox{$[\![ \textrm{#1} ]\!]$}}
%
% \def\maru#1{{\ooalign{\hfil
%   \ifnum#1>999 \resizebox{.25\width}{\height}{#1}\else%
%   \ifnum#1>99 \resizebox{.33\width}{\height}{#1}\else%
%   \ifnum#1>9 \resizebox{.5\width}{\height}{#1}\else #1%
%   \fi\fi\fi%
% \/\hfil\crcr%
% \raise.167ex\hbox{\mathhexbox20D}}}}
%
\newcommand{\HypSpace}{\hskip-.8ex}
\newcommand{\RaiseHeight}{\raisebox{2.2ex}}
% \newcommand{\RaiseHeightLess}{\raisebox{1ex}}
%
% \newcommand{\fW}{\ensuremath{\mathfrak{W}}}
%
\newcommand{\ThreeColHyp}[1]{\RaiseHeight{\Bigg[}\HypSpace#1\HypSpace\RaiseHeight{\Bigg]}}
% \newcommand{\TwoColHyp}[1]{\RaiseHeightLess{\Big[}\HypSpace#1\HypSpace\RaiseHeightLess{\Big]}}
%
%
% %\newcommand{\maskref}[1]{\textsl{\textbf{[reference omitted for refereeing]}}}
% \newcommand{\maskref}[1]{#1}
%
% \newcommand{\DerivSize}{\small}
% \newcommand{\AppDerivSize}{\footnotesize}
%
% \renewcommand{\sem}{\ensuremath}


% \newcommand{\greekp}{{\color{green}π}}
% \newcommand{\greekr}{{\color{green}\textrho}}
% \newcommand{\greeks}{{\color{green}\textsigma}}
\newcommand{\ptfont}[1]{\texttt{#1}}                % what does ptfont do? Where is it defined?
                                % Question
%\newcommand{\ptfont}{\ttfamily}

%\newcommand{\grey}{\color{gray}}
\newcommand{\grey}[1]{\colorbox{mycolor}{#1}}
\definecolor{mycolor}{gray}{0.8}

\newcommand{\gap}{\longrule}
\newcommand{\gp}{\gap}
\newcommand{\vs}{\raisebox{.05em}{\ensuremath{\upharpoonright}}}
% \newcommand{\sub}[1]{\textsubscript[#1]}
\newcommand{\E}{\emph}
\newcommand{\B}{\textbf}
\newcommand{\f}{{\color{green}f}}  % Question what does f do? It does not have any output in the
                                % original PDF
%\newcommand{\Lemma}{{\color{pink}Lemma}}
\newcommand{\Lemma}{\ensuremath{\vdots\hskip.5cm\vdots}\noLine}

%\newcommand{\calP}{{\color{pink}calP}} % Sebastian
\newcommand{\calP}{\ensuremath{\mathcal{P}}}


\newcommand{\maru}[1]{\ooalign{\hfil#1\/\hfil\crcr
      \raise.05ex\hbox{\LARGE\mathhexbox20D}}}


%\newcommand{\sem}[2][M\!,g]{\mbox{$[\![ \mathrm{#2} ]\!]^{#1}$}}
\newcommand{\sem}{\ensuremath}

%
\newcommand{\trns}[1]{\textbf{#1}\xspace}

\newcommand{\bs}{{\textbackslash}}
\newcommand{\bsl}{{\bs}}


\newcommand{\fb}[1]{\textsubscript{#1}}

\newcommand{\syncat}[1]{\ensuremath{\mathrm{#1}}}
\newcommand{\term}[1]{\textit{#1}}
\newcommand{\LemmaAlt}{\ensuremath{\vdots\hskip.5cm\vdots}}

%\renewcommand{\O}{ø}

   \togglepaper[23]%%chaptern
   %% hyphenation points for line breaks
%% Normally, automatic hyphenation in LaTeX is very good
%% If a word is mis-hyphenated, add it to this file
%%
%% add information to TeX file before \begin{document} with:
%% %% hyphenation points for line breaks
%% Normally, automatic hyphenation in LaTeX is very good
%% If a word is mis-hyphenated, add it to this file
%%
%% add information to TeX file before \begin{document} with:
%% %% hyphenation points for line breaks
%% Normally, automatic hyphenation in LaTeX is very good
%% If a word is mis-hyphenated, add it to this file
%%
%% add information to TeX file before \begin{document} with:
%% \include{localhyphenation}
\hyphenation{
    par-a-digm
}

\hyphenation{
    par-a-digm
}

\hyphenation{
    par-a-digm
}

   \boolfalse{bookcompile}umber
}{}

\title{A Peircean turn in linguistics: syntactic-semantic composition as logical inference}


\author{Robert Levine\affiliation{Ohio State University}}

\abstract{In the late 19th century Charles Sanders Peirce proposed what can be seen as a model of
  natural language in which the combinatoric affinity of lexical items—which he characterizes as
  their respective valence—drove the composition of sentences. In this paper I argue that Peirce’s
  conception of valence as the basis of linguistic composition,  ncorporated into a logic of types
  in which valence is interpreted as implication, finds its formal realization in a species of
  categorial grammar. I further show the power of this conception in capturing a complex interaction
  of filler-gap connectivity with ellipsis which has been claimed to be one of the strongest pieces
  of evidence for covert structure analyses treatment of ellipsis patterns. The type-logical
  treatment of this supposed pattern of extraction from ellipsis sites undercuts such claims, and
  reinforces Joachim Lambek’s invocation of Peirce as perhaps the earliest intellectual ancestor of
  modern type-logical approaches to natural language architecture.}

\begin{document}
\shorttitlerunninghead{A Peircean turn in linguistics}
\maketitle
\providecommand{\SetInfLen}{\relax}



\itdopt{proper quotation marks, can we avoid bold? Section 2 has a long beginning and a very short
  subsection 2.1 at the end.}

\section{Peirce and valence}

The work of Charles Sanders Peirce---a long-time research focus of Dan
Everett, whom this festschrift honors---spans a range of interests in,
and major contributions to, a range of mathematical and scientific
domains that may well be unique in the history of human
accomplishent. Pearce's work is widely recognized as seminal in
mathematics, logic, the philosophy of knowledge, chemistry, astronomy
and many other fields, but it is not generally recognized that he was
the source of an analytic concept, \textsl{valence}, which has become
a foundational tool in linguistic theory. But until quite recently,
Peirce has not been credited with his identification of the
combinatorial potential of lexical items as one of the key drivers of
linguistic form, or for being the first scholar to use the term
`valence', which he borrowed from chemistry as a close analogue of
this linguistic concept. It is still common to see linguistic
applications of valence as having been `founded in 1959 by Lucien
Tesnière' \citet{hollein2022}. There were, as noted in \citet{AdamP-Peirce}, a
number of earlier invocations of the metaphor referencing the
electrons needed by atoms of an element in order to attain a stable
state. But the earliest such appeal to this metaphor, as \citet{askedal91}
and Przepi\'{o}rkowski document, was \citet{peirce1897}, where the word
\textit{gives} is explicitly identified as having the same number of
`unsaturated bonds' as the nitrogen atom, which combines with three
hydrogen atoms to form the ammonia molecule NH\fb{\,\,\textrm{3}}, and
Przepi{\'o}rkowski considers it likely that Tesnière and the
others who introduced valence into the parlance of grammarians were
all influenced by Peirce's original invocation of the
concept. Przepi{\'o}rkowski notes (155) that `four linguists working
in four different countries independently came up with the valency
metaphor', within the space of a single decade, and suggests that the
common source for their exposure to Peirce's metaphor was not Peirce
himself, but Roman Jacobson, who was probably the earliest
grammarian of the modern era to recognize the depth of Peirce's
insights on natural language, particularly \citet{peirce1897}, and actively
promoted Peirce's work in conversations and international gatherings,
such as the 1948 International Congress of Linguists in Paris, among
other venues.

It is generally conceded among those who have studied Peirce's work as
it bears on natural languages that his perspective was primarily
rooted in their semiotic capabilities, as systems of signs. But as
Nöth observes, for Peirce, ``the key to syntactic structure is the
predicate and its valency'' \citeyearpar[7]{nothPeirce}. Peirce
seems to have regarded the valence of sentences in both a syntactic
and semantic way: on the one hand, the places in which names can
appear (whose occupants he called `subjects') and on the other, as the
parts of propositions which the predicate sets into the relationship
that the predicate denotes, and which point to particular
individuals---the referents of the names themselves.

There are a few aspects to this conception of syntax which deserve
some amplification, because they bear directly on what I believe
amounts to a specific development of Peirce's ideas. Peirce clearly
did not adopt the widespread contemporary view that syntactic
categories are to be regarded as projections of lexical categories;
that e.g. NPs are in effect just nouns with various other encrusted
bits---adjectives, determiners and so on that are attached to the Ns
that are the `head' of the NP. Rather, his perspective appears to have been
based much more on the conceptual burden of the items corresponding to
the parts of the proposition conveyed by the sentence. But in
principle, if there were a one-to-one relationship between the way in
which syntactic valence is satisfied and the way in which semantic
meaning is assembled, then the conceptual construal of valence and the
syntactic combinatorics of language would essentially mirror each
other.

Contemporary phrase structure approaches, of course, do not adhere to
the analytic program such a unified view of syntax and semantics
imposes. Typically, we find a set of lexical (and, in certain
approaches, morphological) elements that represent the lowest tier of
syntactic objects, corresponding to the terminal nodes in phrase
structure trees, and more complex objects that these element compose
into, which satisfy some set of criteria---typically based heavily on
distributional possibilities, displaceability chief among them. These
elements combine by rules which license hierarchical structures that
represent the syntactic form of a sentence as the record of all the
combinatorial steps that had to apply to derive that sentence.  But
there is a alternative approach available, one in which lexical items
are regarded as inhabitants of different \textsc{types}, representing what
is in effect the combinatorial `destiny' of the words inhabiting that
type, and in which the mode of synactic composition and the mode of
semantic composition are at a more abstract level the same
operation. Such a theoretical architecture represents, in my view, one
possible way in which contemporary formal linguistics reflects a
Peircian turn, although one perhaps rather different from what Peirce
himself had in mind.


In a sense, it seems a bit of truism to describe any particular
framework as `valence-based'; virtually all major theories or
`programs' utilize some notion of valence as a central feature in
licensing sentences. But it is not often appreciated how much mileage
is possible by driving an approach in which the combinatoric
possibilities of individual words can determines quite complex
patterns and effects, including arbitrarily non-local dependencies and
interactions amongst such dependencies. In the following sections I
outline a framework based on this architecture---as first envisaged
and articulated by Peirce---and show how it allows us to formulate
alternatives to standard phrase structure analyses that do not require
us to posit elaborate machinery altering the hierarchical arrangement
of structures that have already been formed, but nonetheless capture a
particularly intricate relationship between long-distance dependencies
and ellipsis strictly on the basis of lexical argument structure.



\section{Argument structure and labeled deductive systems\label{sec:argstruc}}

Theories of syntactic structure of the sort alluded to in the final
paragraph of the preceding section belong to a family of frameworks
that represent different versions of type-logical categorial grammar.
The essential premise shared by the frameworks is that the rules of
syntactic composition are stated as a deductive calculus formally
equivalent to at least the implicational fragment of one or another
standard truth conditional logic, with inference from valid type(s) to
valid type in place of inference from true premise(s) to true
premise. In the framework described below, each linguistic sign
comprises a phonological and semantic annotation which is said to
\textsl{label} the sign's syntactic type. The compositional rules of
the grammar are homologous to the implicational subsystem of
substructural intuitionistic propositional logic (SIPL), i.e.,
IPL lacking rules of permutation, contraction or weakening, with implication
corresponding to types of the form \syncat{\textit{Y}/\textit{X}}, \syncat{\textit{X}\bsl{}\textit{Y}} and \syncat{\textit{Y}\vs \textit{X}} The
first of these can be thought of as something like, `give me a sign of
type \textit{X} on my right and you'll get back a sign of type  \textit{Y}', and the
second is the same with `left' in place of `right'. The third is a bit
more complex: it tells you that, if there is a sign of type it can be
realized in a certain designated position `within' the sign typed
\textit{Y}\vs \textit{X}. We refer to inhabitants of slashed types as
\textsl{functional} terms, in view of their semantics, as we discuss below.

What are the syntactic types that can instantiate \textit{Y} and \textit{X}? For our
purposes, we can posit three atomic types, which are in a
one-to-one relationship with basic semantic types:\footnote{Here and
in what follows, I used the standard angled bracket notation
$\langle\tau_1,\tau_2\rangle$ to indicate a function from some object
of semantic type $\tau_1$ to an object of semantic type $\tau_2$.}

\begin{exe}
 \ex\label{types}
\begin{tabular}{clc}
\textsf{Type} & \textsf{Semantic object} & \textsf{Semantic type} \\
 & & \\
S & proposition & $t$\\
NP & referring expression & $e$ \\
N  & property & $\langle e,t\rangle$ \\
\end{tabular}
\end{exe}
Clauses then correspond to propositions, and NPs to Peirce's
`subjects', so that in (\ref{example}a), for example, we would assign \textit{give}
the type \syncat{(NP\bsl{}S)/NP/NP}:\footnote{We will also take PP to be a basic type,
although here matters are a bit more complex: typically, inhabitants of
the type PP have the same semantic type as those typed NP.}


\begin{exe}
 \ex\label{example}
  \begin{xlist}
 \ex\label{}
    John showed Mary the manuscript.
 \ex\label{}
    \LexEnt{\pt{ gave}}{\sem{ \trns{give} }}{\syncat{(NP\bsl{}S)/NP/NP }}
  \end{xlist}
\end{exe}
The rules under which (\ref{example}b) composes with its argument terms to
yield the sentence in (\ref{example}a) are, as noted, the elimination and
introduction rules for implication of intuitionistic propositional
logic, where implication takes the three forms noted earlier. There
are a number of different formats for logical rules; the system I
introduce here belongs to a subfamily of type-logical frameworks which
uses the Natural Deduction conventions. In the Prawitz notion followed
below, the ordinary IPL rule would take the form in (\ref{prawitz}b):

\begin{exe}
 \ex\label{prawitz}
\begin{tabular}{cccc}
{a.
\AxiomC{$\phi\supset\psi$}
\AxiomC{$\phi$}
\RightLabel{\mbox{\tiny $\scriptstyle \supset\,$}\tiny Elim}
\BinaryInfC{$\psi$}
\DisplayProof
} & & &
{b.
\AxiomC{$\phi$}
\noLine
\UnaryInfC{$\vdots$}
\noLine
\UnaryInfC{$\psi$}
\RightLabel{\mbox{\tiny $\scriptstyle \supset\,$}\tiny Intro}
\UnaryInfC{$\phi \supset \psi$}
\DisplayProof
}
\end{tabular}
\end{exe}
(\ref{prawitz}a) is nothing other than the ancient principle of Modus
Ponens, where there is an antecedent ($\phi$) and a consequent
($\psi$), such that the truth of $\phi$ is a guarantor of the truth of
$\psi$. (\ref{prawitz}b) is the slightly less transparent rule of
hypothetical reasoning: if, in some context of established results
introducing an hypothesis $phi$ allows us to deduce $\psi$, then in
that same context, we know that the implication $\phi\supset\psi$
follows.\footnote{Intuitionistic implication differs from classical
implication in that Peirce's
Law---$((\phi\supset\psi)\supset\phi)\supset\phi$---holds for the
latter but not the former, since on intuitionistic assumptions there
is no way to deduce the consequent $\phi$ from the antecedent
$(\phi\supset\psi)\supset\phi$. This is as it should be so far as our
type logic is concerned, since translation of Peirce's Law into type
logic results in a generally false prediction about argument
structure.} In a nutshell, if we assume a certain premise that allows
us to deduce a certain result, we know that, mutatis mutandis, if that
premise were true, the result would then follow.

But transating these rules into the type-logical domain requires a
good deal more than just inference rules for types. Linguistic signs do not just inhabit types;
they also carry phonological and semantic information. Unlike the
propositions that combine under intuitionistic rules of inference, the
word(sequence)s that are the corresponding type-logical objects are
ordered linearly in sentences---a property we take to be a prosodic,
not syntactic fact, reflecting our partial adoption of the
tectogrammatical/phenogrammatical distinction advanced in
\citet{Curry1961}. Similarly, syntactic composition and inference are
exactly mirrored in the semantic combinatorics, as will become evident
from the full statement of the type-logical rules of inference given
in (\ref{rules}), corresponding to (\ref{prawitz}), assumed throughout this
paper. In (\ref{rules}) and hereafter, I take a sign to be a tripartite
object with a prosodic sector, a semantic sector and a type value,
presented in that order.

\begin{exe}
 \ex\label{rules}
\oneline{%
\begin{tabular}{ccc}
\term{Connective} & \term{Introduction} & \term{Elimination} \\
/ &
{
\AxiomC{\Lemma}
\UnaryInfC{\Lemma}
\AxiomC{\ensuremath{[\LexEnt{\pt{φ}}{\sem{ x }}{\syncat{\textit{A}}}]^1\,\,}}
\UnaryInfC{\LemmaAlt}
\AxiomC{\Lemma}
\UnaryInfC{\Lemma}
\TrinaryInfC{
  \LexEnt{
  \pt{\ptv{b} \BobsO \pt{φ} }}
  {\sem{ \sF}}
  {\syncat{\textit{B}}}
}
\RightLabel{\scalebox{.8}{/I$^n$}}
\UnaryInfC{\LexEnt{\pt{\ptv{b}}}{\sem{ λx. \sF}}{\syncat{\textit{B}/\textit{A}}}}
\DisplayProof}
&
{
\AxiomC{
  \LexEnt{\pt{\ptv{b}}}
  {\sem{ \calP}}
  {\syncat{\textit{B}/\textit{A}}}
}
\AxiomC{\LexEnt{\pt{\ptv{a}}}{\sem{ \alpha}}{\syncat{\textit{A}}}}
\RightLabel{\scalebox{.8}{\ensuremath{/}E}}
\BinaryInfC{\LexEnt{\pt{\ptv{b} \BobsO \ptv{a}}}{\sem{ \calP(\alpha)}}{\syncat{\textit{B}}}}
\DisplayProof}\\
& & \\[-1ex]
\bsl &
{
\AxiomC{\Lemma}
\UnaryInfC{\Lemma}
\AxiomC{\ensuremath{[\LexEnt{\pt{φ}}{\sem{ x }}{\syncat{\textit{A}}}]^1\,\,}}
\UnaryInfC{\LemmaAlt}
\AxiomC{\Lemma}
\UnaryInfC{\Lemma}
\TrinaryInfC{\LexEnt{\pt{\pt{φ} \BobsO \ptv{b}}}{\sem{ \sF}}{\syncat{\textit{B}}}}
\RightLabel{\scalebox{.8}{\bsl I$^n$}}x
\UnaryInfC{\LexEnt{\pt{\ptv{b}}}{\sem{ λx. \sF}}{\syncat{\textit{A}\bsl{}\textit{B}}}}
\DisplayProof}
&
{
\AxiomC{\LexEnt{\pt{\ptv{a}}}{\sem{ \alpha}}{\syncat{\textit{A}}}}
\AxiomC{\LexEnt{\pt{\ptv{b}}}{\sem{ \calP}}{\syncat{\textit{A}\bsl{}\textit{B}}}}
\RightLabel{\scalebox{.8}{\bsl E}}
\BinaryInfC{\LexEnt{\pt{\ptv{a} \BobsO \ptv{b}}}{\sem{ \calP(\alpha)}}{\syncat{\textit{B}}}}
\DisplayProof}\\
 & & \\
\ensuremath{\vs} &
{
\AxiomC{\Lemma}
\UnaryInfC{\Lemma}
\AxiomC{\ensuremath{[\LexEnt{\pt{φ}}{\sem{ x }}{\syncat{\textit{A}}}]^1\,\,}}
\UnaryInfC{\LemmaAlt}
\AxiomC{\Lemma}
\UnaryInfC{\Lemma}
\TrinaryInfC{\LexEnt{\pt{\ptv{b}}}{\sem{ \sF}}{\syncat{\textit{B}}}}
\RightLabel{\scalebox{.8}{\ensuremath{\vs}$^n$}}
\UnaryInfC{\LexEnt{\pt{\ensuremath{\lambda} \ensuremath{\greekp}.  \ptv{b}}}{\sem{ λx. \sF}}{\syncat{\textit{B}\vs \textit{A}}}}
\DisplayProof}
&
{
\AxiomC{\LexEnt{\pt{\ptv{b}}}{\sem{ \calP}}{\syncat{\textit{B}\vs \textit{A}}}}
\AxiomC{\LexEnt{\pt{\ptv{a}}}{\sem{ \alpha}}{\syncat{\textit{A}}}}
\RightLabel{\scalebox{.8}{\ensuremath{\vs}E}}
\BinaryInfC{\LexEnt{\pt{\ptv{b(a)}}}{\sem{ \calP(\alpha)}}{\syncat{\textit{B}}}}
\DisplayProof}
\end{tabular}}
\end{exe}
In (\ref{rules}), the vertical lines surrounding the variable and its
composition into the proof denote the proof history subsequent to the
introduction of the variable. \ptv{a,b} are metavariables over
strings---lexical items or sequences of lexical items---while \pt{ p-
} is a variable, supplied not by the lexicon, or as a stand-in for
some actual string whose value is irrelevant in the context of the
rule.  Rather, they are part of the logic itself, representing, in
effect.a space in a prosodic or semantic expression that could be
occupied by any term of the same type as the variable. Each variable
sign is introduced with a specific index, and each application of an
Introduction rule is keyed to the index of the variable which is
removed in the introduction of the directional slashes or \L-bound in
\ensuremath{\vs} introduction.  The Elimination rules shown are, again, different
avatars of (\ref{prawitz}a): a slashed term seeks a term of the antecedent
type to give us a consequent type, and the result of composing the
slashed term with the antecedent term is necessarily a term of the
consequent type. One can see these inference rules as inversions of
ordinary context-free PS rules; for example, taking VP to be an
abbreviation for \syncat{NP\bsl{}S}---a clause modulo an NP term on its left
edge---we have S \ensuremath{ \rightarrow } NP \syncat{NP\bsl{}S} on the one hand and a deduction\bigskip

\AxiomC{NP}
\AxiomC{NP$\bsl$S}
\RightLabel{\scalebox{.8}{\bsl E}}
\BinaryInfC{S}
\DisplayProof
\bigskip

\noindent on the other.\footnote{However, as noted below, this view leaves
the nature of a type-logical proof open to a foundational misinterpretation.} The prosodic and semantic sectors combine in
lockstep with the type composition: the prosody of directionally
slashed types---$\!\!\!$ \syncat{\textit{Y}/\textit{X}} and \syncat{\textit{X}\bsl{}\textit{Y}}---reflects the
direction of the slash: the former precedes the prosody of its type X
argument, the latter follows it\footnote{I defer discussion of
vertically slashed terms till we get to the Introduction rules}. The
semantics, however, does not reflect the direction of the slash: for
all functional types, the semantic term is a function which takes the
denotation of the syntactic argument as its own arguent.

With this much in hand, we can now provide a complete proof of an
English sentence that illustrates the ways in which type, prosody and
semantics collaborate to derive the sentence in (\ref{sent}) as, in effect, a
theorem. We start with a lexicon, as in (\ref{lexicon}):

\begin{exe}
 \ex\label{sent}
  John sent those documents to the committee over the weekend.
\end{exe}
\begin{exe}
 \ex\label{lexicon}
\begin{tabular}{lll}
\LexEnt{\pt{john}}{\sem{ \trns{j} }}{\syncat{NP}} & &
\LexEnt{\pt{sent}}{\sem{ send}}{\syncat{VP/PP\fb{to}/NP}} \\
\LexEnt{\pt{those}}{\sem{ \iota}}{\syncat{NP/N\fb{pl}}} & &
\LexEnt{\pt{documents}}{\sem{ \trns{docs} }}{\syncat{N\fb{pl}}} \\
\LexEnt{\pt{to}}{\sem{ λx.x}}{\syncat{PP\fb{to}/NP}} & &
\LexEnt{\pt{the}}{\sem{ \iota}}{\syncat{NP/N}} \\
\LexEnt{\pt{committee}}{\sem{ comm}}{\syncat{N}} & &
\LexEnt{\pt{over}}{\sem{ \trns{over} }}{\syncat{(VP\bsl{}VP)/NP}} \\
\LexEnt{\pt{weekend}}{\sem{ \trns{wknd} }}{\syncat{N}} & &
\end{tabular}
\end{exe}
Lexical entries are axioms of the type logic (though other axioms are
possible, including axioms which incur some kind of penalty, and
license proofs whose output is not fully acceptable, allowing us to
incorporate a range of gradience effects into the framework). A few
comments on (\ref{lexicon}) are in order:  PP\fb{to} is
a subtype of PP, derived via the the unique prepositional type PP\fb{to}/NP, whose
semantic interpretation is an identity function, yielding
a denotation identical to that of its argument. Ns have subtypes N\fb{sg}
and N\fb{pl}, with some determiners targeting one or the other. Finally,
\textit{over}, despite its standard identification as a preposition, is in
type-logical terms a function composing with an NP to yield a function
which applies a temporal semantics to a property, corresponding to a
restriction of the event instantiating that property.

The rules of the logic apply to the lexical axioms to yield the proof
in (\ref{sentProof}):

\begin{exe}
 \ex\label{sentProof}
%\hspace{-8ex}
\oneline{
\AxiomC{\MultiLine{\ptfont sent; \\ \sem{ \trns{send} }; \\\syncat{  VP/PP\fb{to}/NP}}}
\AxiomC{\MultiLine{\ptfont the; \\ \sem{ \iota}; \\\syncat{  NP/N}}}
\AxiomC{\MultiLine{\ptfont documents; \\ \sem{ \trns{docs} }; \\\syncat{  N}}}
\RightLabel{\scalebox{.8}{\ensuremath{/}E}}
\BinaryInfC{\MultiLine{\ptfont the \BobsO documents; \\ \sem{ \iota(\trns{docs})}; \\\syncat{  NP}}}
\RightLabel{\scalebox{.8}{\ensuremath{/}E}}
\BinaryInfC{\MultiLine{\ptfont sent \BobsO the \BobsO documents; \\ \sem{ \trns{send}(\iota(\trns{docs}))}; \\\syncat{  VP/PP\fb{to}}}}
\AxiomC{\MultiLine{\ptfont the; \\ \sem{ \iota}; \\\syncat{  NP/N}}}
\AxiomC{\MultiLine{\ptfont committee; \\ \sem{ \trns{comm} }; \\\syncat{  N}}}
\RightLabel{\scalebox{.8}{\ensuremath{/}E}}
\BinaryInfC{\MultiLine{\ptfont the \BobsO committee; \\ \sem{ \iota(\trns{comm})}; \\\syncat{  NP}}}
\AxiomC{\MultiLine{\ptfont to; \\ \sem{ λx.x}; \\\syncat{  PP\fb{to}/NP}}}
\RightLabel{\scalebox{.8}{\ensuremath{/}E}}
\BinaryInfC{\MultiLine{\ptfont to \BobsO the \BobsO committee; \\ \sem{ \iota(\trns{comm})}; \\\syncat{  PP\fb{to}}}}
\RightLabel{\scalebox{.8}{\ensuremath{/}E}}
\BinaryInfC{\MultiLine{\ptfont sent \BobsO the \BobsO documents \BobsO to \BobsO the \BobsO committee; \\ \sem{ \trns{send}(\iota(\trns{docs}))(\iota(\trns{comm}))}; \syncat{ VP}}}
\AxiomC{\MultiLine{\ptfont  over; \\ \sem{ \trns{over} }; \\\syncat{  VP\bsl{}VP/NP}}}
\AxiomC{\MultiLine{\ptfont  the; \\ \sem{ \iota}; \\\syncat{  NP/N}}}
\AxiomC{\MultiLine{\ptfont  weekend; \\ \sem{ \trns{wknd} }; \\\syncat{  N}}}
\RightLabel{\scalebox{.8}{\ensuremath{/}E}}
\BinaryInfC{\MultiLine{\ptfont the \BobsO weekend; \\ \sem{ \iota(\trns{wknd})}; \syncat{ NP}}}
\RightLabel{\scalebox{.8}{\ensuremath{/}E}}
\BinaryInfC{\MultiLine{\ptfont over \BobsO the \BobsO weekend; \\ \sem{ \trns{over}(\iota(\trns{wknd}))}; \\\syncat{  VP\bsl{}VP}}}
\RightLabel{\scalebox{.8}{\bsl E}}
\BinaryInfC{\MultiLine{\ptfont sent \BobsO the \BobsO documents \BobsO to \BobsO the \BobsO    {}\\ \ptfont   \hspace{3ex}committee \BobsO over \BobsO the \BobsO weekend; \\   \sem{ \trns{over}(\iota(\trns{wknd}))(\trns{send}(\iota(\trns{docs}))(\iota(\trns{comm})))}; \\\syncat{  VP}}}
\AxiomC{\MultiLine{\ptfont john; \\ \sem{ \trns{j} }; \\\syncat{  NP}}}
\RightLabel{\scalebox{.8}{\bsl E}}
\BinaryInfC{\MultiLine{\ptfont john \BobsO sent \BobsO the \BobsO documents \BobsO to \BobsO the \BobsO    {}\\ \ptfont   \hspace{3ex}committee \BobsO over \BobsO the \BobsO weekend; \\   \sem{ \trns{over}(\iota(\trns{wknd}))(\trns{send}(\iota(\trns{docs}))(\iota(\trns{comm})))}; \\\syncat{  S}}}
\DisplayProof
}
\end{exe}
This proof can be seen as a realization of Peirce's emphasis on
argument structure, and its satisfaction, as the `engine' of syntactic
combinatorics. As noted at the beginning of this paper, the types
associated with strings---either in the lexicon or via composition in
the course of the proof---do not reflect the standard parts of speech
inherited from the classical grammarians, but rather their
combinatorial affinities, determined in part by the nature of their
contribution to the formation of the proposition conveyed by a
declarative sentence, or of the more complex semantical object denoted
by questions, and so on. Proofs proceed purely on the basis of logical
inference driven by type specifications, with semantic composition
mirroring the composition steps determined by the inference
rules given in (\ref{rules}), and the rules themselves reflecting standard
truth-conditional deductive systems.

On the other hand, one might take the view that such a proof is
nothing more than a recasting of a standard hierarchical
representation licensed by context-free phrase structure rules. This
would be however a fundamental error: note that, in contrast to the
hierarchical realization of phrase-structure rules in branching tree
representations of constituent syntax, the proof steps in (\ref{sentProof})
have no representational status so far as the structure of the
sentence is concerned. Indeed, strictly speaking there \textit{is} no
such structure: what we have in (\ref{sentProof}) is a demonstration that
the closure of the axioms of the system---the English lexicon---under
the inference rules of the logic allow a valid inference of a prosodic
string
\pt{john \BobsO sent \BobsO the \BobsO documents \BobsO to \BobsO the \BobsO committee \BobsO over \BobsO the \BobsO weekend}
which signifies a proposition
\sem{ \trns{over}(\iota(\trns{wknd}))(\trns{send}(\iota(\trns{docs}))(\iota(\trns{comm}))) } and that the
linguistic expression which has those prosodic and semantic values is
a sentence. The steps involved in the proof have no representational
status, any more than, given a set of premises $\Gamma$, the steps in
the proof of \sem{ \Gamma  \ensuremath{\vdash\xspace } \phi } in some stardard logic have any
bearing on the content of $\phi$.\footnote{For example, there are any
number of ways to prove that \sem{  \ensuremath{\vdash\xspace } \phi\supset(\neg\phi\supset\psi) }
in classical logic, but the content of of the implication is
altogether independent of proof narrative.}

The difference between the logical composition of terms in (\ref{sentProof})
a tree representation of (\ref{sent}) under a set of phrase structure rules
becomes far more stark when we turn from the elimination rules, which
are the only ones in play in (\ref{sentProof}), to the introduction rules
shown in (\ref{rules}). There is nothing in phrase structure grammmar which
corresponds to the introduction rules, and here the advantages of the
proof-theoretic framework become apparent. So-called non-constituent
coordination patterns such as Right Node Raising in (\ref{RNR}) and
Dependent Cluster Coordination in (\ref{DCC}) are pointed examples:

\begin{exe}
 \ex\label{NCC}
  \begin{xlist}
 \ex\label{RNR}
    John bought, and Bill baked, the pizza margherita.
 \ex\label{DCC}
    John sent that message to Bill on Thursday and Mary on
  \end{xlist}
  Saturday.
\end{exe}
Both of the patterns in (\ref{NCC}) are essentially embarrassments to
frameworks based on phrase structure configurations, requiring either
transformational grammar's complex arrangements of structure-altering
operations, including movement and{\slash}or deletion (along with the purely
stipulative constraints on the linear output of these operations
required to get the facts right), or essentially stipulative
constructional templates, as in later developments of HPSG (for
detailed critques of these approaches, see \citet{levine11},
\citet{kubota-levine-coord}, \citet{kubotalevineBook}). For proof-theoretic
approaches, on the other hands, where valence satisfaction is driven
by the inference rules of standard logics, the data in (\ref{NCC}) are
almost trivial to obtain with the correct semantics, once we've
generalized the system based only on the elimination rules to the
introduction rules that are their logical duals, as in all Natural
Deduction systems. For example, we have the following straightforward
application of the / Elimination rule:

\begin{exe}
 \ex\label{rnrSubproof}
\AxiomC{\LexEnt{\pt{bought}}{\sem{ \trns{buy} }}{\syncat{VP/NP}}}
\AxiomC{\LexEnt{\pt{φ}}{\sem{ x}}{\syncat{NP}}}
\BinaryInfC{\LexEnt{\pt{bought \BobsO p}}{\sem{ \trns{buy}(x)}}{\syncat{VP}}}
\AxiomC{\LexEnt{\pt{john}}{\sem{ \trns{j} }}{\syncat{NP}}}
\BinaryInfC{\LexEnt{\pt{john \BobsO bought \BobsO p}}{\sem{ \trns{buy}(x)(\trns{j})}}{\syncat{S}}}
\UnaryInfC{\LexEnt{\pt{john \BobsO bought}}{\sem{ λx. \trns{buy}(x)(\trns{j})}}{\syncat{S/NP}}}
\DisplayProof
\end{exe}
The / Elimination rule allows us to obtain what in stardard phrase
structure appoaches would be characterized as a partial constituent
(although, in the framework adopted here, it is no more
`partial' than VPs, i.e., signs inhabiting the type \syncat{NP\bsl{}S}). A
completely parallel proof will derive the sign in (\ref{baked}):

\begin{exe}
 \ex\label{baked}
  \LexEnt{\pt{bill \BobsO baked}}{\sem{ λu. \trns{bake}(u)(\trns{b})}}{\syncat{S/NP}}
\end{exe}
Application of the standard generalized conjunction operator
\sem{  \ensuremath{ \sqcap\xspace }  } \trns{introduced} \trns{in} \citet{partee-rooth1983a}, which we take to be the
denotation of \textit{and}, with the type (\syncat{\textit{X}\bsl{}\textit{X}})/\textit{X}, will then lead to the
inference in (\ref{rnrSign}):

\begin{exe}
 \ex\label{rnrSign}
  \LexEnt{\pt{john \BobsO bought \BobsO and \BobsO bill \BobsO baked}}{\sem{ λw. \trns{buy}(w)(\trns{j}) /\f \trns{bake}(w)(\trns{b})}}{\syncat{S/NP}}
\end{exe}
\itdopt{St.: Question what does f do? It does not have any output in the original PDF}
The final step in the proof will then be (\ref{rnrFinal}):

\begin{exe}
 \ex\label{rnrFinal}
\AxiomC{\MultiLine{\ptfont john \BobsO bought \BobsO and \BobsO    {}\\ \ptfont   \hspace{3ex} bill \BobsO baked; \\   \sem{ λw. \trns{buy}(w)(\trns{j}) \ensuremath{ \wedge\xspace } \trns{bake}(w)(\trns{b})}; \\\syncat{  S/NP}}}
\AxiomC{\Lemma}
\UnaryInfC{\MultiLine{\ptfont the \BobsO pizza \BobsO margarita; \\ \sem{ \iota(\trns{pzzmarg})}; \syncat{ NP}}}
\BinaryInfC{\MultiLine{\ptfont john \BobsO bought \BobsO and \BobsO    {}\\ \ptfont   \hspace{3ex}bill \BobsO baked; \\   \sem{  \trns{buy}(\iota(\trns{pzzmarg}))(\trns{j}) \ensuremath{ \wedge\xspace } \trns{bake}(\iota(\trns{pzzmarg}))(\trns{b})}; \\\syncat{  S}}}
\DisplayProof
\end{exe}
(\ref{DCC}) can be similarly derived via a somewhat tedious but
straightforward subproof that yields \textit{Bill on Thursday} and \textit{Mary on
Friday} as inhabitants of the type \syncat{(PP\fb{to}/NP)\bsl{}(NP\bsl{}(PTV\bsl{}VP))}, where PTV
is an abbreviation for the type VP/PP\fb{to}/NP. The conjunction of the two
is therefore also of this same type, so that \textit{Bill on Thursday and Mary on
Saturday} combines to its left first with a PP\fb{to}/NP sign (i.e., \textit{to}),
then an NP (\textit{the message}), then a PTV sign (\textit{sent}), and finally VP,
i.e., \syncat{NP\bsl{}S}, which picks up \textit{John} to give us (\ref{DCC}).

In a nutshell, in both of the patterns exhibited in (\ref{NCC}), the
interplay of the elimination and introduction rules allows us to
compose each of the conjoined `nonconstituents' into an S as arguments
of a variable, with all other components of the S realized as
variables, and then eliminate the variable terms by successive
applications of the relevant introduction rules. The result is that
the apparent nonconstituent prosodic elements are assigned a type,
with a corresponding semantics corresponding to the applicaton of
abstraction operators at each elimination step. They are thus, in our
terms, full constituents, now with the status of functional terms, and
can then be conjoined. The resulting conjunction, possibly with a
rather elaborate valence as in the case of (\ref{DCC}), then composes with
its arguments to form the coordination. No structural operations, or
indeed any structures at all are involved; the proofs given do
nothing more than verify the association of the prosody of the
specific conjunctions with a certain valence, or argument structure,
and a corresponding semantics. This kind of operation is often
characterized as type-raising, but in the deductive system embodied in
(\ref{rules}), it is simply a by-product of the logic of implication
elimination and introduction.

This leaves the rule for \ensuremath{\vs} introduction to be considered. \ensuremath{\vs}
introduction differs from directional slash introduction in one
foundational respect: rather than simply removing \pt{ \ensuremath{\greekp} } from the
prosodic string, the variable becomes bound by an abstraction
operator. This makes the resulting prosodic object a function, not a
string, and when the prosody of a sign typed \syncat{\textit{Y}\vs \textit{X}} composes with
the prosody of a type $X$ sign, the former takes the latter as an
argument (as aptly illustrated by (\ref{cleopatra}) below).  It is worth
noting that the introduction rules repesent a formal expression of
Peirce's observation, quoted in \citet[8]{nothPeirce}: `in the proposition
{`'}Anthony gave a ring to Cleopatra'', Cleopatra is as much a subject of what is
meant and expressed as is the ring or Anthony. A proposition, then,
has one predicate and any number of subjects.' The significant insight
here---that a sentence expressing a proposition can be composed as the
ascription of some property to \textsl{any} of the argument
terms---corresponds exactly to the possibility of deriving a predicate
by composing a predicate with one variable term, with constants for
all the other arguments, and then abstracting on that
variable.

\begin{exe}
 \ex\label{cleopatra}
\oneline{%
\AxiomC{\LexEnt{\pt{gave}}{\sem{ \trns{give} }}{\syncat{VP/PP\fb{to}/NP}}}
\AxiomC{\LexEnt{\pt{φ}}{\sem{ x}}{\syncat{NP}}}
\BinaryInfC{\LexEnt{\pt{gave \BobsO \ensuremath{\greekp}}}{\sem{ \trns{give}(x)}}{\syncat{VP/PP\fb{to}}}}
\AxiomC{\Lemma}
\UnaryInfC{\MultiLine{\ptfont to \BobsO cleopatra; \\ \sem{ \trns{cleop} }; \syncat{ PP\fb{to}}}}
\BinaryInfC{\LexEnt{\pt{gave \BobsO \ensuremath{\greekp} \BobsO to \BobsO cleopatra}}{\sem{ \trns{give}(x)(\trns{cleop})}}{\syncat{VP}}}
\AxiomC{\LexEnt{\pt{antony}}{\sem{ \trns{ant} }}{\syncat{NP}}}
\BinaryInfC{\LexEnt{\pt{antony \BobsO gave \BobsO \ensuremath{\greekp} \BobsO to \BobsO cleopatra}}{\sem{ \trns{give}(x)(\trns{cleop})(\trns{ant})}}{\syncat{S}}}
\UnaryInfC{\LexEnt{\pt{\ensuremath{\lambda} \ensuremath{\greekp}.antony \BobsO gave \BobsO \ensuremath{\greekp} \BobsO to \BobsO cleopatra}}{\sem{ λx. \trns{give}(x)(\trns{cleop})(\trns{ant})}}{\syncat{S\vs NP}}}
\DisplayProof}
\end{exe}
The predicate in the final line of (\ref{cleopatra}) ascribes a property to
some object; that object is in the set of things given to Cleopatra to
Antony, and the proposition in the passage Noth quotes from Peirce is
decomposable into \L-terms along these lines just as much as it is the
composition of \textit{Antony} with the denotation of the VP \textit{gave a ring to
Cleopatra}. In this sense, \ensuremath{\vs} Introduction is the logical warrant for
the Peirce's view that \textit{Anthony gave a ring to Cleopatra} is `about'
any (or all) of its `subject', not just the NP which carries the
grammatical function `subject'.

But the empirical problem which led to \ensuremath{\vs} introduction (under a different
notation) in \citet{oehrle1994} was rather different, and took the form of
the question, how can we capture the fact that quantified expressions
such as \textit{every student}, \textit{some book}, \textit{most journals} and so on have
the same syntactic distribution as NPs, i.e., names and definite
descriptions, while corresponding to radically different semantic
objects? And, related to this question, is a second: how do
quantifiers interact syntactically with the sentences they appear in
such that they take scope over subportions of the semantic
interpretation of those sentences? Various solutions have been
proposed, e.g., the machinery introduced by \citet{montague1973a}, whereby all quantified expressions and
names denote property sets, i.e., are functors on the properties
denoted by the VPs that take them as syntactic subject, which require
the use of meaning postulates; post-SpellOut
movement operations (`Quantifier Raising', originating in
\citet{may_r1985a}) in the most recent incarnations of transformational
grammar; Cooper's \citeyearpar{cooper75,Cooper83}
storage mechanism, adopted in \citet{pollardsag94}, and may others. In
some cases the solutions involve formal devices that seem to be
purpose-built for the description of quantifier's syntactic and
semantic behavior, with little use outside the specific problem they
were designed for, e.g., quantifier storage and retrieval; in others,
there is is no connection to an actual model-theoretically accessible
semantic denotation, as is the case with `Quantifier Raising' in
transformational grammar and the \citet{pollardsag94} proposal; and still
others are problematic in both respects.

Oehrle's 1994 breakthrough, in contrast, is conceptually simple, of
extremely broad application to problems of the syntax-semantics
interface, and yields a directly interpretable expression in
higher-order logic that is model-theoretically interpretable in a
straightforward way. But this last point needs to be amplified: the
basic approach is itself compatible with a wide range of explicit
semantic frameworks, including proof-theoretic approaches that do not
appeal to any model. Oehrle's key innovation was the application of a
higher-order logic in the prosodic sector, with a corresponding type
hierarchy, allowing the semantics and the prosody to operate
independently of each other so that quantified expressions, and scopal
operators, generally, can in effect take the syntactic contexts in
which they appear as their own arguments. The following simple example
is representative of the setup generally. We have

\begin{exe}
 \ex\label{someone}
  John gave someone that book.
\end{exe}
We take the quantified expression \textit{someone} to be a functor that
intersects a property argument with the set of people, and returns a
truth value of 0 if the intersection is $\varnothing$, and 1
otherwise. A proof along the lines of (\ref{cleopatra}) will directly yield the
sign in (\ref{gaveTerm}):

\begin{exe}
 \ex\label{gaveTerm}
  \LexEnt{\pt{ \ensuremath{\lambda} \ensuremath{\greekp}. john \BobsO gave \BobsO \ensuremath{\greekp} \BobsO that \BobsO book}}{\sem{ λx. \trns{gave}(x)(\iota(\trns{book}))}}{\syncat{S\vs NP }}
\end{exe}
The semantics here is just what we need: the characteristic function
of the set of entities who received some discoure-prominent (and in
some sense pragmatically distal) member of the set of books from
John. \textit{Someone} intersects this set with that of people and, based on
the model, returns a value of 0 or 1. But in that case, the pronunciation
\pt{someone} cannot itself be the prosody of \textit{someone}, since in that
case it would be an argument of the prosody in (\ref{gaveTerm}) despite
\textit{someone}'s semantics taking the latter's
interpretation as its argument. Prosody and semantics would thuis be at
irreconcilable cross-purposes.

Oehrle's ingenious solution to this seeming contradiction takes the
prosody of \textit{someone} to be, not \pt{someone}, but a function that
applies the prosody of its \syncat{S\vs NP} argument to \pt{someone}. Since
\pt{\ensuremath{\lambda} \ensuremath{\greekp}. john \BobsO gave \BobsO \ensuremath{\greekp} \BobsO that \BobsO book} is a string-to-string
function, \textit{someone} is given a prosody which applies to such functions
and positions them to take a string argument \pt{someone} to the
pronunciation of (\ref{someone}). The lexical entry for \textit{someone} is then

\begin{exe}
 \ex\label{someoneEntry}
  \LexEnt{\pt{ \ensuremath{\lambda} \ensuremath{\greeks}. \ensuremath{\greeks}(someone)}}{\sem{ λP. \trns{\raisebox{1.5ex}{\ensuremath{\rotatebox{180}{\textbf{E}}}}}(\trns{person})(P)}}{\syncat{S\vs (S\vs NP) }}
\end{exe}
and we have the simple proof in (\ref{someoneProof}):

\begin{exe}
 \ex\label{someoneProof}
\oneline{%
\AxiomC{\Lemma}
\UnaryInfC{\MultiLine{\ptfont λ\ensuremath{\greekp}. john \BobsO gave \BobsO \ensuremath{\greekp} \BobsO that \BobsO book; \\ \sem{ λx. \trns{gave}(x)(\iota(\trns{book}))}; \syncat{ S\vs NP}}}
\AxiomC{\LexEnt{\pt{\ensuremath{\lambda} \ensuremath{\greeks}. \ensuremath{\greeks}(someone)}}{\sem{ λP. \trns{\raisebox{1.5ex}{\ensuremath{\rotatebox{180}{\textbf{E}}}}}(\trns{person})(P)}}{\syncat{S\vs (S\vs NP)}}}
\BinaryInfC{\MultiLine{\ptfont λ\ensuremath{\greeks} [\ensuremath{\greeks}(someone)](λ\ensuremath{\greekp}. john \BobsO gave \BobsO \ensuremath{\greekp} \BobsO that \BobsO book); \\ \sem{ λP[ \trns{\raisebox{1.5ex}{\ensuremath{\rotatebox{180}{\textbf{E}}}}}(\trns{person})(P)](λx. \trns{gave}(x)(\iota(\trns{book})))}; \syncat{ S}}}
\dottedLine
\RightLabel{\mbox{\tiny{$\beta$}-conversion}}
\UnaryInfC{\LexEnt{\pt{\ensuremath{\lambda} \ensuremath{\greekp} [john \BobsO gave \BobsO \ensuremath{\greekp} \BobsO that \BobsO book](someone)}}{\sem{ \trns{\raisebox{1.5ex}{\ensuremath{\rotatebox{180}{\textbf{E}}}}}(\trns{person})(λx. \trns{gave}(x)(\iota(\trns{book})))}}{\syncat{S}}}
\dottedLine
\RightLabel{\mbox{\tiny{$\beta$}-conversion}}
\UnaryInfC{\LexEnt{\pt{john \BobsO gave \BobsO someone \BobsO that \BobsO book)}}{\sem{ \trns{\raisebox{1.5ex}{\ensuremath{\rotatebox{180}{\textbf{E}}}}}(\trns{person})(λx. \trns{gave}(x)(\iota(\trns{book})))}}{\syncat{S}}}
\DisplayProof}
\end{exe}
The quantified expression takes scope over the context in which it
appears---its continuation, in Barker's terms \citeyearpar{Barker2002,Barker2004} (see also \citealt{barkershan2015}). If two quantified expressions are
introduced into a single proof, the first one introduced into the
proof will scope over the material included into the proof up to that
point, and will then be part of the context which the second one
scopes over when the latter is in turn added in the proof. A different
proof, in which the two are introduced in the opposite order, will
yield the opposite scoping. No special mechanism or operation is
therefor required to obtain multiple scopings under the inference
rules in (\ref{rules}) (see \citealt[Section~2.3]{kubotalevineBook} for
details).

Oehrle's solution to the parallelism of NP and quantifed expression
distibutions plays on the independent but linked relationship of
prosody and semantics in type-logical grammar—a relationship made
possible by the \ensuremath{\vs} connnective. Quantified expressions parallel NPs
precisely because they are in a sense parasitic on NP variables: they
only appear in parts of the string where such variables can appear,
undergo abstraction and ultimately replacement by the string element
in the prosody of quantifiers. At the same time, their syntax targets
sentences which are `missing' NPs, in the sense that some argument
position in the semantics is occupied by a \L-bound variable. These
characteristics of \ensuremath{\vs} play an essential role across a wide range of
phenomena, one of which is considered in detail in \sectref{sec-extraction} as
a dramatic illustration of the way effects which require recourse to
operations on phrase structure in other approaches can be reduced to
mappings between valence values in type-logical grammar, with no need
to posit syntactic configuation.


\subsection{Interlude}

At this point, it's important that we take a step back from the
technical details covered in this section in order to get a more
global picture of the strategy embodied in an approach based on
(\ref{rules}). The explicit correspondence between the implicational syntax
and the operations of abstraction and function application in the
semantics and prosody via independent type hierarchies with their own
respective \L-calculi, guarantee a fully compositional derivation of
signs, with the syntactic types guiding the composition on the basis
of the familiar logic of Modus Ponens and hypothetical reasoning. The
critical point here is that not only obviously local dependencies
involving argument structure, but arbitrarily long-distance
effects---in particular, the interpretation of quantifier scope---are
reducible to the satisfaction of argument requirements; in effect, in
the proof-theoretic architecture of type-logical frameworks, valence
satisfaction is the source of all observed grammatical regularities,
as well as constructional idiosyncrasies.\footnote{For
a demonstration of how these eccentries can be elegantly accounted
for, see \citet{kubota-levine2022a}}. In the case of scopal operators, such
as generalized quantifiers---as well as symmetrical predicates such as
\textit{same}, \textit{similar}, \textit{different}, and various other varieties, the
relationship between semantics and syntax is immediate and
transparent: quantified expressions scope over the denotations of
their syntactic arguments, in exactly the same way that modal
auxiliaries and raising verbs scope over their VP arguments. In all
cases, truth-conditional meaning is composed in accord with the
valence of predicates and operators.

What about genuinely long-distance dependencies, of the sort
exemplified by topicalization, \textit{wh} displacement, \textit{tough}
constructions and many others? These are standardly treated by
machinery which `localizes' the dependency, but in neither
derivational nor monostratal phrase-structure frameworks is the same
mechanism employed for this localization as for garden-variety valence
satisfaction. The point of the following analysis is to demonstrate
the degree to which a proof-theoretic approach in which valence satisfaction,
rather than syntactic configuration, yields the extraction dependency
can capture the relevant phenomena in a simple and transparent fashion.

\section{``Extraction'' from ellipsis sites: what you don't see is what you don't get}
\label{sec-extraction}

\subsection{The empirical problem}

There is a sizable contemporary cross-linguistic literature on
ellipsis, generally understood to refer to a varied range of phenomena
in which semantic content from one part of a discourse context is
part of the interpretation supplied by other(typically, but
not necessarily, following)  material, despite the absence of any
overt phonology and syntax corresponding to that interpretation. We
find, for example, patterns such as the following:

\begin{exe}
 \ex\label{ellipsis}
  \begin{xlist}
 \ex\label{VPE}
    John likes pizza, but Bill doesn't $\varnothing$\bsl\xspace\
    `John likes pizza, but Bill doesn't like pizza'. \\
    (VP/Post-auxiliary ellipsis)
 \ex\label{pseudo}
    John eats way more junk food than he does $\varnothing$ real food.\\
    `John eats way more junk food than he eats real food. \\
    (Pseudogapping)
 \ex\label{sluice}
    John was arguing with someone, but I don't know who
  $\varnothing$\bsl\xspace\
    `John was arguing with someone, but I don't know who John was
  arguing with.' \\
    (Sluicing)
 \ex\label{shortans}
   \textsf{Q:} Who was John talking to? \\
   \textsf{A:} $\varnothing$ Someone from his department.\\
   `John was talking to someone from his department'\\
   (Fragment answers)
  \end{xlist}
\end{exe}
There are a number of other subspecies of ellipsis, but those in
(\ref{ellipsis}) have had the lion's share of attention from theorists, most
of whom appear to favor some version of the basic analytic line taken
that originates in \citet{kuno81} and has been most influentially developed
in \citet{merchant01} and subsequent work, whereby the interpretive glosses
in (\ref{ellipsis}) are, in essence, the syntactic sources of the examples
themselves. (\ref{VPE}) on this approach arises from a series of processes
that can be graphically summarized as something very much like
(\ref{VPdeletion}):

\begin{exe}
 \ex\label{VPdeletion}
  John likes pizza, but Bill doesn't [\sub{VP} \sout{like pizza}]
\end{exe}
Pseudo-gapping, as in (\ref{pseudo}), is the result of a movement to the
left or right of the post-auxiliary `remnant' followed by the VP
deletion process suggested in (\ref{pseudodeletion}):

\begin{exe}
 \ex\label{pseudodeletion}
  John eats way more junk food than he does [\sub{VP} [\sub{VP} \sout{eat} {\gp}\xspace ] real food]
\end{exe}
and so on. Most of the arguments in favor of this approach are
necessarily indirect, based on patterns of acceptability judgments which seem to
mirror judgments of corresponding non-ellipsed data; in \citet{kubotalevineBook}, a
detailed examination of what appear to be the most persuasive of these
arguments strongly suggests that they are in fact quite fragile on
both empirical and methodological grounds. The central difficulty with
such arguments is their pivotal assumption  that
the phenomena in  ellipsis and corresponding non-ellipsed example
which evoke parallel judgments of acceptability---e.g., island effects,
restriction on anaphora, etc.---are themselves syntactic in
nature. Building this assumption into an argument that parallel
judgments of ellpsis and correspnding non-ellipsed data reflects the
need to posit covert phrase structure which is deleted in the course
of derivations thus appears to be a textbook instance of begging the question.

Defenders of the view that what you don't see in ellipsis was never
there in the first place still have their work cut out for them, of
course; it is necessary to construct plausibility arguments for the
premises that (i) the putatively syntactic effects alluded to have
non-syntactic origins and (ii) that the parallels between ellipsis and
non-ellipsed examples can originate in the extragrammatical sources
adduced in establishing (i). Examples such as the following are
particularly challenging insofar as (i) is concerned;


\begin{exe}
 \ex\label{extrac-from-VPE}
  \begin{xlist}
 \ex\label{}
    I know what John ate for lunch,  but I don't know what Bill did.
 \ex\label{}
    I'm acutely aware of what I can do and what I
  can't. (\citet[735]{mahoney2004}).
 \ex\label{}
    John is certain \textsl{he} would buy \textsl{this} kind of sports car, but I have no idea what kind \textsl{I} would.
  \end{xlist}
\end{exe}
Although examples of the sort displayed in (\ref{extrac-from-VPE}) are not
easy to discover in corpora, they can be found with a bit of
persistence, though the third example is unattested (but has been
checked with multiple informants, the great majority of whom found it
altogether unproblematic with the right prosody (though the latter
varied somewhat from speaker to speaker). But there is one species of this class of
\textit{wh}-extractions, so called antecedent-contained deletion,  which is
quite common. Data parallel to (\ref{antecDel}) can readily be found in
Google search results, for example.

\begin{exe}
 \ex\label{antecDel}
  \begin{xlist}
 \ex\label{}
    I hate feeling like everyone knows something I don't {\gp}\xspace . \\\url{https://twitter.com/therealkimj/status/1640857002896396288}
 \ex\label{}
    And perhaps they would nod with understanding at what a senior
  once told me: `'Everyone knows something that I don't {\gp}\xspace . I keep
  asking until I find out what that is.''\\
  \url{https://www.footstepstoinspire.org/post/}\\
  \url{sportistheperfecthuntinggroundforsexualabuse}
 \ex\label{}
    However, 4 months ago i said something which i shouldn`t have. \\
    \url{https://tinybuddha.com/blog/how-to-mend-a-broken-friendship-}\\
    \url{even-if-youre-not-on-speaking-terms/}
  \end{xlist}
\end{exe}
Dozens of such instances of the construction can be found in Google
searches, and there is a very substantial literature on them. Versions
of the sort shown in (\ref{extrac-from-VPE}) are less well-studied, but
there \textsl{have} \textsl{has} been a certain amount of research devoted to them (see, for example,
\citet{schuyler2002} and references there).

The problem for (i) is that whereas there is now a deep body of
results constituting compelling evidence against the structural
origins of island effects (for recent overviews of the relevant
literature, see, e.g., \citet{chaves-putnam2020}, \citet{kubotalevineBook}, \citet{liu-et-al2022}),
most frameworks take filler-gap connectivity itself as irreducibly
syntactic in nature. And while there are deep consequences that follow
from rejecting movement operations as the source of extraction, this
theoretical position does not, on its own, give us any particular help
in explaining what the \textit{wh} word is doing in (\ref{extrac-from-VPE}). In
GPSG and its descendent HPSG, for example, a feature carrying
information about the syntactic and semantic content of a \textit{wh}
constituent must be carried through the structure to the point where a
category matching that content satisfies the valence requirements of a
selecting head. In the case of (\ref{extrac-from-VPE}), the default analysis
in these frameworks would license a connectivity linkage of this sort
which would be `cashed out', as it were, by either an empty category
corresponding to a a valent of some transitive verb or, as in later
work in HPSG assumed, in a reduction in the valence of such a verb
(e.g., per the analyses of extraction patterns in \citet{NoordBouma94} and
\citet{BoumaMaloufea01}). And the entire `point' of VP ellipsis is that no such
verb is present. Unsurprisingly, advocates of analyses based on
covert-structure solutions to the problems posed by ellipsis seem to
have been taken such examples as prima facie evidence for the presence
of covert structure.  Thus \citet{johnson2001} takes examples such as
(\ref{extrac-from-VPE}) to show that `the ellipsis site seems to have
internal parts', while Elbourne agrees that `things seem especially
difficult for [approaches to ellipsis] according to which there is
nothing whatsoever in ellipsis sites' \citep[216]{Elbourne2008}.  So far as I
am aware, there has to date been no account of the pattern exhibited
in (\ref{extrac-from-VPE}) in any work in the monostratal phrase-structure
tradition that offers an explicit counteranalysis to the
movement-and-deletion analysis assumed by transformationalists.

But such an alternative is readily available. It rests however on a
particular approach to extraction connectivity and assumes a specific
analysis of VP ellipsis, both of which differ considerably from
standard positions common to both transformational and monostratal
frameworks. In the following section, I first outline a commonly
assumed type-logical treatment of filler/gap linkage, and in the next
section, recapitulate the treatment of VP ellipsis, and its
generalization to pseudogapping, proposed in
\citet{kubota-levine-pseudo}. This background sets the stage for my account
of (\ref{extrac-from-VPE}).\footnote{For a rather different, though
ultimately related approach to a solution in a framework belonging to
a distinct class of categorial grammar frameworks, see \citet{jacobson1992}.}



\citet{muskens03} outlines a treatment of unbounded \textit{wh} dependencies,
readily extendable to topicalization, which differs radically from
previous analyses of extraction within both phrase-structure-based
approaches and categorial grammar. In terms of \textit{wh} relatives,
Muskens' proposal takes the form of the lexical sign in (\ref{whrel}):

\begin{exe}
 \ex\label{whrel}
  \LexEnt{\pt{\ensuremath{\lambda} \ensuremath{\greeks}. which \BobsO \ensuremath{\greeks}(\E)}}{\sem{ λP λQ λw. P(w) \ensuremath{ \wedge\xspace } Q(w)}}{\syncat{(N\bsl{}N)\vs (S\vs NP)}}
\end{exe}
Unpacking this operator a bit, we can see that its argument structure
seeks a clause missing an NP, and its denotation is predicated of
some entity, while the prosodic functor corresponding to its phonology
applies to a string of length zero. To derive (\ref{book}), then, we start
with the subproof in (\ref{bookProof1}):

\begin{exe}
 \ex\label{book}
  the book which John lost yesterday
\end{exe}
\begin{exe}
 \ex\label{bookProof1}
\oneline{%
\AxiomC{\LexEnt{\pt{lost}}{\sem{ λx λy \trns{lose}(x)(y)}}{\syncat{(NP\bsl{}S)/NP}}}
\AxiomC{\ensuremath{[\LexEnt{\pt{\ensuremath{\greekp_0}}}{\sem{ u}}{\syncat{NP}}]^1\,\,}}
\BinaryInfC{\LexEnt{\pt{lost \BobsO \ensuremath{\greekp_0}}}{\sem{ λy \trns{lose}(u)(y)}}{\syncat{NP\bsl{}S}}}
\AxiomC{\LexEnt{\pt{yesterday}}{\sem{ λP λv.yst(P)(v) }}{\syncat{(NP\bsl{}S)\bsl{}(NP\bsl{}S)}}}
\BinaryInfC{\MultiLine{\ptfont lost \BobsO \ensuremath{\greekp_0} \BobsO yesterday; \\ \sem{ λv. \trns{yst}(\trns{lose}(u))(v)}; \syncat{ NP\bsl{}S}}}
\AxiomC{\MultiLine{\ptfont john; \\ \sem{ \trns{j} }; \\\syncat{  NP}}}
\BinaryInfC{\LexEnt{\pt{john \BobsO lost \BobsO \ensuremath{\greekp_0}}}{\sem{ \trns{yst}(\trns{lose}(u))(\trns{j})}}{\syncat{S}}}
\LeftLabel{①\ensuremath{\rightarrow}}
\UnaryInfC{\LexEnt{\pt{\ensuremath{\lambda} \ensuremath{\greekp_0}. john \BobsO lost \BobsO \ensuremath{\greekp_0} \BobsO yesterday}}{\sem{ λu.  \trns{yst}(\trns{lose}(u))(\trns{j})}}{\syncat{S\vs NP}}}
\DisplayProof}
\end{exe}
The operator in (\ref{whrel}) takes arguments of this type and returns a
function which picks up an N on the left, while $\beta$-converting a
zero-length string into the position occupied by \pt{ \ensuremath{\greekp_0} } in the last
proof line in (\ref{bookProof1}), giving us \textit{which John lost yesterday}:

\begin{exe}
 \ex\label{bookProof2}
\oneline{%
\AxiomC{\MultiLine{\ptfont λ\ensuremath{\greeks}. which \BobsO \ensuremath{\greeks}(\E); \\ \sem{ λP λQ λw. P(w) \ensuremath{ \wedge\xspace } Q(w)}; \\\syncat{  (N\bsl{}N)\vs (S\vs NP)}}}
\AxiomC{\MultiLine{\ptfont λ\ensuremath{\greekp_0}. john \BobsO lost \BobsO \ensuremath{\greekp_0} \BobsO yesterday; \\ \sem{ λu  \trns{yst}(\trns{lose}(u))(\trns{j})}; \\\syncat{  S\vs NP}}}
\BinaryInfC{\MultiLine{\ptfont λ\ensuremath{\greeks} [which \BobsO \ensuremath{\greeks}(\E)](λ\ensuremath{\greekp_0}. john \BobsO lost \BobsO \ensuremath{\greekp_0} \BobsO yesterday); \\ \sem{ λP [ λQ λx. P(w) \ensuremath{ \wedge\xspace } Q(w)](λu  \trns{yst}(\trns{lose}(u))(\trns{j}))}; \\\syncat{  N\bsl{}N}}}
\dottedLine
\UnaryInfC{\LexEnt{\pt{which \BobsO \ensuremath{\lambda} \ensuremath{\greekp_0} [john \BobsO lost \BobsO \ensuremath{\greekp_0} \BobsO yesterday](\E)}}{\sem{ λQ. [ λw.  \trns{yst}(\trns{lose}(w))(\trns{j}) \ensuremath{ \wedge\xspace } Q(w)]}}{\syncat{N\bsl{}N}}}
\dottedLine
\UnaryInfC{\LexEnt{\pt{which \BobsO john \BobsO lost \BobsO \E \BobsO yesterday}}{\sem{ λQ [ λw.  \trns{yst}(\trns{lose}(w))(\trns{j}) \ensuremath{ \wedge\xspace } Q(w)]}}{\syncat{N\bsl{}N}}}
\DisplayProof}
\end{exe}

The final part of the proof supplies an N argument to the functional
term in the last proof line in (\ref{bookProof2}):

\begin{exe}
 \ex\label{bookProof3}
\oneline{%
\AxiomC{\LexEnt{\pt{book}}{\sem{ \trns{book} }}{\syncat{N}}}
\AxiomC{\LexEnt{\pt{which \BobsO john \BobsO lost \BobsO \E \BobsO yesterday}}{\sem{ λQ [ λw.  \trns{yst}(\trns{lose}(w))(\trns{j}) \ensuremath{ \wedge\xspace } Q(w)]}}{\syncat{N\bsl{}N}}}
\BinaryInfC{\LexEnt{\pt{book \BobsO  which \BobsO john \BobsO lost \BobsO \E \BobsO yesterday}}{\sem{  λw.  \trns{yst}(\trns{lose}(w))(\trns{j}) \ensuremath{ \wedge\xspace } \trns{book}(w)]}}{\syncat{N}}}
\DisplayProof}
\end{exe}
We thus obtain \textit{book which John lost}, denoting the set of things
which have the properties of being books and being objects that John
lost.

The critical point for us is what happens at \ding{172} in
(\ref{bookProof1}). Application of \ensuremath{\vs} Introduction abstracts on the variable
terms superscripted as 1---an operation completely indifferent to
the length of the string in which the variable \pt{ \ensuremath{\greekp_0} }
appears. Exactly the same step would take us from the expression above
the proof line in (\ref{length}) to the sign below the line:

\begin{exe}
 \ex\label{length}
\AxiomC{\MultiLine{\ptfont mary \BobsO thinks \BobsO bill \BobsO remembers \BobsO ann \BobsO saying \BobsO john \BobsO lost \BobsO \ensuremath{\greekp}; \\ \sem{ \trns{think}(\trns{remember}(\trns{saying}(\trns{yest}(\trns{lost}(u))(\trns{j}))(\trns{a})))(\trns{b})))(\trns{m})}; \\\syncat{  S}}}
\UnaryInfC{\MultiLine{\ptfont λ\ensuremath{\greekp}. mary \BobsO thinks \BobsO bill \BobsO remembers \BobsO ann \BobsO saying \BobsO john \BobsO lost \BobsO \ensuremath{\greekp}; \\ \sem{ \trns{think}(\trns{remember}(\trns{saying}(\trns{yest}(\trns{lost}(u))(\trns{j}))(\trns{a})))(\trns{b})))(\trns{m})}; \\\syncat{  S}}}
\DisplayProof
\end{exe}
Essentially the same proof storyline in (\ref{bookProof1})--(\ref{bookProof3})
will give us the \textit{book which Mary thinks Bill remembers Ann saying
John lost}. There is no local registration of the information linking
the filler to the gap---nothing analagous to cyclic \textit{wh} movement, no
SLASH feature shared between vertically adjacent nodes in a
phrase-structure tree that gets realized at the tail end of the
chain. Properly speaking, there isn't even anything that can be
properly identified as a gap `site'. We have a prosodic component of
the sign with no marker corresponding to some missing substring, since
the model theory for the prosodic calculus interprets \ptv{a \BobsO b \BobsO \E{}} as
\ptv {a o b}; nor is there any representation in the semantics or the
syntactic type of something we would want to call a `gap'. In a way,
this treatment of extraction is an echo of the view in extraction in
the earliest phase of transformational grammar, when \textit{wh} movement
shifted a constituent to the left over an unconstrained variable. The
appearance of \citet{ross67} resulted in the almost universal rejection of
this view, but the most recent research on the island effects that
Ross first documented, as noted earlier, overwhelmingly support a view
of such effects which takes them to be epiphenomena of functional
factors. Clearly, the nonlocal view of syntactic connectivity has an
empirical claim on a second act.\footnote{This is not to say, of
course, that Muskens' operator is completely unproblematic. For one
thing, it has an obvious failure in its coverage, since obviously
there's no way that (\ref{whrel}) as given accounts for pied-piping. A
second problem is that the linearity of the type logic shared by
Muskens' \L-grammar and our own HTLG, inter alia, makes it difficult
to derive multiple extractions linked to a single filler. In \citet{kubotalevineBook},
we offer solutions for both problems, and are currently generalizing
our proposal for pied-piping to take into account the interaction of
the latter with a variety of coordination possibilities, an aspect of
the pied-piping problem that does not appear to have been previously
addressed. But for our purposes, the approach exemplified in (\ref{whrel})
is completely serviceable.}

\subsubsection{VP ellipsis and pseudogapping}\label{subsubsec:VPE}

The correct explanation for data such as (\ref{extrac-from-VPE}) obviously
depends on an empirically sound analysis of VP ellipsis in the first
place. The standard transformational approaches following \citet{kuno1981}
face severe empirical challenges and serious conceptual problems,
detailed in \citet{kubota-levine-pseudo}. These problems are avoided in the
proof-theoretic solution proposed there, whose central premise that VP
ellipsis itself is the expression of a kind of `zero derivation'
whereby signs typed VP/VP are mapped to the type VP, whose denotation
is the application of the modal/aspectual operator of the input sign
to some salient property retrieved from the discourse context or,
under certain conditions, inferred exophorically, per
\citet{MillerPullum13}. This approach is implemented via the operator in
(\ref{ellipseOp1}), where \$ is a variable over sequences of arguments,
following notation introduced in \citet{steedman2000a})\footnote{Because the
prosodic term is a function, the main connective in the type
description is \ensuremath{\vs} rather than /.}:

\begin{exe}
 \ex\label{ellipseOp1}
  \textbf{VP ellipsis operator} \\
  \LexEnt{\pt{ \ensuremath{\lambda} \ensuremath{\greekp}. \ensuremath{\greekp} }}{\sem{ λ\ensuremath{\mathscr{F}}. \ensuremath{\mathscr{F}}(P')}}{\syncat{(NP\bsl{}S)\vs ((NP\bsl{}S)\vs (NP\bsl{}S) }}

  ---where $P'$ is a free variable whose value is
  resolved anaphorically
 \ex\label{Acondition}
  Anaphora resolution condition on the VP ellipsis/pseudogapping
  operator:
  \begin{enumerate}
   \item
    if there is a syntactic  constituent
    with category VP in the antecedent clause matching the
    syntactic  category of the missing verb in the target clause,
    then the value of $P$ is identified with the denotation of that constituent;
   \item
    if there is no such syntactic  constituent, then the value of $P$  is anaphorically
    identified with some salient property in the discourse that is not
    inconsistent with the syntactic  category VP.
  \end{enumerate}
\end{exe}
An example of simple VP ellipsis, illustrating how the ellipsis operator
in (\ref{ellipseOp1}) works, is given in (\ref{simpleVPE})

\begin{exe}
 \ex\label{simpleVPE}
  \begin{xlist}
 \ex\label{simpleVPEex}
    Mary should call Ann, but Bill shouldn't.
  \end{xlist}
\end{exe}
\begin{exe}
 \ex\label{simpleVPEproof}
\begin{tabular}{ll}
 \scalebox{.48}{
\AxiomC{\MultiLine{\ptfont should ; \\ \sem{ λQ λy. \ensuremath{\square} Q(y)}; \\\syncat{   (NP\bsl{}S)/(NP\bsl{}S)}}}
\AxiomC{\MultiLine{\ptfont call; \\ \sem{ \trns{call} }; \\\syncat{  (NP\bsl{}S)/NP}}}
\AxiomC{\MultiLine{\ptfont ann; \\ \sem{ \trns{a} }; \\\syncat{  NP}}}
\LeftLabel{①\ensuremath{\rightarrow}}
\BinaryInfC{\MultiLine{\ptfont call \BobsO ann; \\ \sem{ \trns{call}(\trns{a})}; \\\syncat{  NP\bsl{}S}}}
\RightLabel{\scalebox{.8}{\ensuremath{/}E}}
\BinaryInfC{\MultiLine{\ptfont should \BobsO call \BobsO ann ; \\ \sem{ λy. \ensuremath{\square} \trns{call}(\trns{a})(y)}; \\\syncat{  NP\bsl{}S}}}
\AxiomC{\LexEnt{\pt{mary}}{\sem{ \trns{m} }}{\syncat{NP}}}
\RightLabel{\scalebox{.8}{\bsl E}}
\BinaryInfC{\MultiLine{\ptfont mary \BobsO should \BobsO call \BobsO ann ; \\ \sem{  \ensuremath{\square} \trns{call}(\trns{a})(\trns{m})}; \\\syncat{  S}}}
\DisplayProof}
 &
 \scalebox{.48}{
\AxiomC{\MultiLine{\ptfont bill ; \\ \sem{ \trns{b} }; \\\syncat{  NP}}}
\RightLabel{per \ding{172}}
\AxiomC{\grey{\MultiLine{\ptfont λ\ensuremath{\greekp}. \ensuremath{\greekp} ; \\ \sem{ λ\ensuremath{\mathscr{F}}. \ensuremath{\mathscr{F}}(\trns{call}(\trns{a}))}; \\\syncat{  (NP\bsl{}S)\vs (NP\bsl{}S/NP\bsl{}S)}}}}
\AxiomC{\MultiLine{\ptfont  should't ; \\ \sem{ λP λy. \neg \ensuremath{\square}\, P(y)}; \\\syncat{  (NP\bsl{}S)/(NP\bsl{}S)}}}
\RightLabel{\scalebox{.8}{\ensuremath{\vs}E}}
\BinaryInfC{\LexEnt{\pt{should't }}{\sem{ λy. \neg \ensuremath{\square}\, \trns{call}(y)()}}{\syncat{ NP\bsl{}S}}}
\RightLabel{\scalebox{.8}{\bsl E}}
\BinaryInfC{\LexEnt{\pt{bill \BobsO should't }}{\sem{  \neg \ensuremath{\square}\, \trns{call}(\trns{a})(\trns{b})}}{\syncat{S}}}
\DisplayProof}
\end{tabular}
\end{exe}

\noindent At the grayed-in proof line, the free variable $P$ is instantiated as
the prominent contextually available property \sem{ \trns{call}(\trns{a}) }.

More complex cases, e.g. those involving sloppy identity (\textit{John thinks
he deserves a promotion, and Bill does too}) and scopal operators
(\textit{John read every book before Bill did}) fall out altogether
straightforwardly on this approach, as shown in
\citet[236--238]{kubota-levine-pseudo}. But for our purposes, what is relevant
is the fact that the operator in (\ref{ellipseOp1}) applies to a functional
term taking a complete \syncat{NP\bsl{}S} to a complete \syncat{NP\bsl{}S}---which we can
abbreviate as VP/VP---and returns a complete VP. But suppose
we generalize the operator so that it applies to a functional term
taking \textsl{partial} VP to a partial VP, and returns a partial VP?
This seems perhaps like a question completely orthogonal to the
phenomena we're looking at, because auxiliaries are, in
non-transformational frameworks generally, taken to apply to VPs and
return VPs, period. But it is a strict theorem of our proof theory
that every VP/VP type has a prosodically identical counterpart which
applies to VP/NP objects and returns a VP/NP object---i.e., maps a
transitive verb to a transitive verb. This is nothing more than a
conversion into type logic of one consequence of the transitivity of
implication in standard logics, and is simply demonstrated as follows
(where $\bigcirc$ is a variable over arbitrary operators):

\begin{exe}
 \ex\label{GeachRule}
\textbf{Geach Theorem}:\\
\AxiomC{\ensuremath{[\LexEnt{\pt{\ensuremath{\greekp_2}}}{\sem{ R}}{\syncat{VP/NP}}]^2\,\,}}
\AxiomC{\ensuremath{[\LexEnt{\pt{\ensuremath{\greekp_1}}}{\sem{ u}}{\syncat{NP}}]^1\,\,}}
\RightLabel{\scalebox{.8}{\ensuremath{/}E}}
\BinaryInfC{\LexEnt{\pt{\ensuremath{\greekp_2} \BobsO \ensuremath{\greekp_1}}}{\sem{  R(u)}}{\syncat{VP}}}
\AxiomC{\grey{\LexEnt{\pt{\ptv{a}}}{\sem{ λT. λz.\bigcirc T(z) }}{\syncat{VP/VP}}}}
\RightLabel{\scalebox{.8}{\ensuremath{/}E}}
\BinaryInfC{\LexEnt{\pt{\ptv{a} \BobsO \ensuremath{\greekp_2} \BobsO \ensuremath{\greekp_1}}}{\sem{ λz. \bigcirc \,R(u)(z))}}{\syncat{VP}}}
\RightLabel{\scalebox{.8}{\ensuremath{/}I\ensuremath{^1}}}
\UnaryInfC{\LexEnt{\pt{\ptv{a} \BobsO \ensuremath{\greekp_2}}}{\sem{ λu λz. \bigcirc \,R(u)(z)}}{\syncat{VP/NP}}}
\RightLabel{\scalebox{.8}{\ensuremath{/}I\ensuremath{^2}}}
\UnaryInfC{\grey{\LexEnt{\pt{\ptv{a}}}{\sem{ λR λu λz. \bigcirc \,R(u)(z)}}{\syncat{(VP/NP)/(VP/NP)}}}}
\DisplayProof
\end{exe}
The point is that VP/VP \sem{  \ensuremath{\vdash\xspace } } (VP/NP)/(VP/NP), and  a completely parallel
entailment can be proven between VP/VP and terms of type VP/NP/NP,
VP/PP/NP, etc. In general, then, for any auxiliary, we have an entailment
VP/VP\sem{ ,  \ensuremath{\vdash\xspace } \, } VP\$/VP\$. It follows that if we generalize the VP ellipsis
operator to the type VP\$\ensuremath{\vs}(VP\$/VP\$), we derive an
operator that yields a form of the auxiliary as a transitive verb, a
ditransitive verb and so on. And such an operator enables us to extend the coverage of
the VP ellipsis rule to the pseudogapping phenomenon illustrated in
(\ref{pseudo}) above; to evade the complexities of the comparative
semantics, I use the somewhat less natural (though still typically
acceptable) \textit{but}-conjunction in (\ref{conjPseudo}):

\begin{exe}
 \ex\label{conjPseudo}
  For some reasons, John will read \textsc{essays} but he won't \textsc{novels}.
\end{exe}
Generalizing the VP ellipsis operator to the form in (\ref{generalized})
would have the effect of taking \textit{won't}, typed
(VP/NP)\ensuremath{\vs}((VP/NP)/(VP/NP)) via application of the Geach entailment so that if we set \$ = NP, application of this
generalized operator to the Geached form of an auxiliary is an
auxliary typed VP/NP, i.e., a transitive verb. This revised operator
can be stated as in (\ref{generalized}):

\begin{exe}
 \ex\label{generalized}
  \textbf{Generalized ellipsis operator} \\
  \pt{\ensuremath{\lambda} \ensuremath{\greekp}. \ensuremath{\greekp} }; \sem{ λ\ensuremath{\mathscr{F}}. \ensuremath{\mathscr{F}}(P) }; VP\$\ensuremath{\vs}(VP\$/VP\$)

  ---where $P$ is a free variable whose value is
  resolved anaphorically
 \ex\label{Acondition}
  Anaphora resolution condition on the VP ellipsis/pseudogapping
  operator:
  \begin{enumerate}
   \item
    if there is a syntactic  constituent
    with category VP\$ in the antecedent clause matching the
    syntactic  category of the missing verb in the target clause,
    then the value of $P$ is identified with the denotation of that constituent;
   \item
    if there is no such syntactic  constituent, then the value of $P$  is anaphorically
    identified with some salient property in the discourse that is not
    inconsistent with the syntactic  category VP\$.
  \end{enumerate}
\end{exe}
We can now derive (\ref{conjPseudo}) directly:

\begin{exe}
 \ex\label{conjPseudo}~\\
\hspace{-10ex}
\begin{tabular}{ll}
{\scalebox{.45}{
\AxiomC{\MultiLine{\ptfont will; \\ \sem{ λP λy.\textrm{\textbf{W}}\scalebox{.75}{\textrm{\textbf{ILL}}}\, P(y)}; \\\syncat{  VP/VP}}}
\AxiomC{\grey{\MultiLine{\ptfont read; \\ \sem{ \trns{read} }; \\\syncat{  VP/NP}}}}
\AxiomC{  essays}
\RightLabel{\scalebox{.8}{\ensuremath{/}E}}
\BinaryInfC{\MultiLine{\ptfont read \BobsO essays; \\ \sem{ \trns{read}(\trns{essays})}; \\\syncat{  VP}}}
\RightLabel{\scalebox{.8}{\ensuremath{/}E}}
\BinaryInfC{\MultiLine{\ptfont will \BobsO read \BobsO essays; \\ \sem{  λy.\textrm{\textbf{W}}\scalebox{.75}{\textrm{\textbf{ILL}}}\, \trns{read}(\trns{essays})(y)}; \\\syncat{  VP}}}
\AxiomC{\MultiLine{\ptfont john; \\ \sem{ \trns{j} }; \\\syncat{  NP}}}
\RightLabel{\scalebox{.8}{\ensuremath{/}E}}
\BinaryInfC{\MultiLine{\ptfont john \BobsO will \BobsO read \BobsO essays; \\ \sem{ \textrm{\textbf{W}}\scalebox{.75}{\textrm{\textbf{ILL}}}\, \trns{read}(\trns{essays})(\trns{j})}; \\\syncat{  S}}}
\DisplayProof
}}
&
\hspace{-6ex}
{\scalebox{.45}{
\AxiomC{\LexEnt{\pt{won't}}{\sem{ λQ [L u λv.\neg\,\textrm{\textbf{W}}\scalebox{.75}{\textrm{\textbf{ILL}}} Q(u)(v)](P')}}{\syncat{VP/NP}}}
\dottedLine
\UnaryInfC{\LexEnt{\pt{won't}}{\sem{ λu λv.\neg\,\textrm{\textbf{W}}\scalebox{.75}{\textrm{\textbf{ILL}}} P'(u)(v)}}{\syncat{VP/NP}}}
\dottedLine
\RightLabel{anaphoric retrieval}
\UnaryInfC{\LexEnt{\pt{won't}}{\sem{ λu λv.\neg\,\textrm{\textbf{W}}\scalebox{.75}{\textrm{\textbf{ILL}}} \trns{read}(u)(v)}}{\syncat{VP/NP}}}
\AxiomC{\MultiLine{\ptfont novels; \\ \sem{ \trns{novels} }; \\\syncat{  NP}}}
\RightLabel{\scalebox{.8}{\ensuremath{/}E}}
\BinaryInfC{\MultiLine{\ptfont won't \BobsO novels; \\ \sem{ λv.\neg\,\textrm{\textbf{W}}\scalebox{.75}{\textrm{\textbf{ILL}}} \trns{read}(\trns{novels})(v)}; \\\syncat{  VP}}}
\AxiomC{\MultiLine{\ptfont he; \\ \sem{ 3 \trns{masc} }; \\\syncat{  NP}}}
\RightLabel{\scalebox{.8}{\bsl E}}
\BinaryInfC{\MultiLine{\ptfont he \BobsO won't \BobsO novels; \\ \sem{ \neg\,\textrm{\textbf{W}}\scalebox{.75}{\textrm{\textbf{ILL}}} \trns{read}(\trns{novels})(3 \trns{masc})}; \\\syncat{  S}}}
\DisplayProof
}}
\end{tabular}
\end{exe}


\noindent Intuitively, the application of the generalized operator in
(\ref{generalized}) is to repackage an auxiliary and a transitive verb as a
somewhat longer and more complex transitive verb, so that rather than
composing \textit{read novels} as a VP and applying a standardly typed
auxiliary to derive a VP, we in effect repackage \textit{won't} and \textit{read} as
a transitive verb \textit{won't}, whose type is the same as \textit{read} itself,
but which, after the retrieval of the corresponding predicate in the
antecedent clause, applies the semantics of \textit{won't} to the proposition
derived by supplying this transitive verb with its arguments.

With our generalized account of ellipsis, we are now in a position to
see how the proof-theoretic approached introduced in the preceding
sections can license examples such as (\ref{extrac-from-VPE}) without
recourse to any actual material corresponding to the gap `site' in the
antecedent clause ever being involved.

\subsubsection{Pseudo-extraction via pseudogapping}\label{subsubsec:pseudoextrac}

In the analysis that follows, apparent extraction from an ellipsed VP
arises as a result of Muskens-style extraction from one or another
argument of the `transitive' auxiliary which is associated with the
general ellipsis operator introduced in
\sectref{subsubsec:VPE}. That is, examples such as
(\ref{extrac-from-VPE}) involve not just a semantic object, as in purely
semantic accounts of ellipsis (e.g., that given in \citealt{hardt-diss})
analysis, but an actual syntactic extraction from an ordinary overt
VP, as we show below. Treatments such as Hardt's, or that given in
\citet{dalrymple-etal1991}, have, as noted in the citation above from
\citet{Elbourne2008}, a difficult time accounting in a simple way for cases
such as (\ref{extrac-from-VPE}); under the analysis which follows, in
contrast, these constructions are predicted to conform to whatever
conditions hold on extraction in general without any concomitant
assumption of covert structure corresponding to an `ellipsed' VP.

What is distinctive about the filler-gap relationship, as vs.\  the
standard picture of valence, is while in both cases we have material
that is missing other material required to compose a constituent, in
the case of the former, the gap can be missing from anywhere within
the partial constituent. That is, while \textit{Y/X} is a sign that must
compose on the left with a sign of type \textit{X}, and $X\bsl Y$ is the same
but seeking an \textit{X} arguent on its left to yield an object typed \textit{Y},
the material missing from the string that is required for \textit{Y} in
\textit{Y}\ensuremath{\vs}\textit{X} can, as noted in \sectref{sec:argstruc}, appear
anywhere; e.g., Thus, in \textit{I wonder what John said to Mary}, the
subconstituent \textit{said {\gp}\xspace to Mary} constitutes a VP with a medial NP
gap, meeting the description \syncat{VP\vs NP}. In terms of sentences such as
(\ref{extrac-from-VPE}), what we want is a way to get \textit{did} to have the type
VP\vs NP, in which case we would, roughly speaking, apply a Muskens-style
operator \textit{what} to a clause composed from this VP\vs NP. As I show
directly, given a sign \textit{did} typed VP\ensuremath{\vs}NP, we can use hypothetical
reasoning to deduce S\ensuremath{\vs}NP and then apply the \textit{what} operator to obtain
signs of expressions such as \textit{what John did}. Furthermore, we predict
on such an approach the well-formedness of e.g.,

\begin{exe}
 \ex\label{internal}
  \begin{xlist}
 \ex\label{}
    Do you think the British know something (about this) that we don't
(at this point)? (Penn Treebank/Wall Street Journal corpus, cited in
\citet{bos-spenader2011}, slightly modified)
 \ex\label{}
    Kollberg suspects Petrus, who Beck does {\gp}\xspace as well (\citet[666]{kennedy1995})
  \end{xlist}
\end{exe}
(\ref{generalized}) will not do the trick here here, since it only gives us
the possibility of elements missing on the right, not medially. What's
needed, clearly is some way to extend the generalized ellipsis rule
still further. Fortunately, just as we were able to show that terms
typed VP/VP can, by the Geach theorem proof given in (\ref{GeachRule}), be
extended to the type (VP/NP)/(VP/NP), we can prove that for any term
inhabiting VP/VP, there is a corresponding term with functional
prosody having the schematic form \syncat{(VP\vs XP)\vs (VP\vs XP)} for any type XP.
The structure of the proof is essentially the same as that of
(\ref{GeachRule}), but involving higher order terms.\footnote{In
(\ref{vertical-proof}), I gloss over certain important technical details in
order to lay out most clearly the proof narrative.}

\begin{exe}
 \ex\label{vertical-proof}
\AxiomC{\LexEnt{\pt{\ensuremath{\greekp_1}}}{\sem{ \ensuremath{\mathscr{O}} }}{\syncat{VP/VP}}}
\AxiomC{\ensuremath{[\LexEnt{\pt{\ensuremath{\greeks_1}}}{\sem{ f}}{\syncat{VP\vs NP}}]^1\,\,}}
\AxiomC{\ensuremath{[\LexEnt{\pt{\ensuremath{\greekp_2}}}{\sem{ x}}{\syncat{NP}}]^2\,\,}}
\BinaryInfC{\LexEnt{\pt{\ensuremath{\greeks_1}(\ensuremath{\greekp_2}) }}{\sem{ f(x)}}{\syncat{VP}}}
\BinaryInfC{\LexEnt{\pt{\ensuremath{\greekp_1} \BobsO \ensuremath{\greeks_1}(\ensuremath{\greekp_2})}}{\sem{ \ensuremath{\mathscr{O}}(f(x))}}{\syncat{VP}}}
\RightLabel{\scalebox{.8}{\ensuremath{\vs}I\ensuremath{^2}}}
\UnaryInfC{\LexEnt{\pt{\ensuremath{\lambda} \ensuremath{\greekp_2}. \ensuremath{\greekp_1} \BobsO \ensuremath{\greeks_1}(\ensuremath{\greekp_2})}}{\sem{ λx. \ensuremath{\mathscr{O}}(f(x))}}{\syncat{VP\vs NP}}}
\RightLabel{\scalebox{.8}{\ensuremath{\vs}I\ensuremath{^1}}}
\UnaryInfC{\LexEnt{\pt{\ensuremath{\lambda} \ensuremath{\greeks_1} \ensuremath{\lambda} \ensuremath{\greekp_2}. \ensuremath{\greekp_1} \BobsO \ensuremath{\greeks_1}(\ensuremath{\greekp_2})}}{\sem{ λf λx. \ensuremath{\mathscr{O}}(f(x))}}{\syncat{(VP\vs NP)\vs (VP\vs NP)}}}
\DisplayProof
\end{exe}
With this result in hand, all that is needed to derive any given
auxiliary as a VP seeking a gap-filling NP constituent
\textsl{somewhere} is a further extension of
the already-generalized ellipsis operator to such `vertically Geached'
auxiliaries, mapping them to type \syncat{VP\vs XP}, anaphorically supplying the
meaning of the gapped VP. In (\ref{zzzz}), I give a `local' form of this
extension of the ellipsis operator to internal gaps.








\begin{exe}
 \ex\label{zzzz}
  \LexEnt{\pt{ \ensuremath{\lambda} \ensuremath{\greekr} \ensuremath{\lambda} \ensuremath{\greekp_1}. \ensuremath{\greekr}(\ensuremath{\lambda} \ensuremath{\greekp_0}. \ensuremath{\greekp_0})(\ensuremath{\greekp_1})}}{\sem{ λ\ensuremath{\mathscr{F}}. \ensuremath{\mathscr{F}}(R\,')}}{\syncat{(VP\vs NP)\vs ((VP\vs NP)\vs (VP\vs NP)) }}

  ---where $R\,'$ is the semantic term of a sign retrieved from the
  context whose type is \syncat{VP\vs NP}
\end{exe}
As before, we first specify how the antecedent clause of
(\ref{extractionVPE-rep}) makes available the predicate which is retrieved
in the ellipsed clause, per (\ref{part1}).

\begin{exe}
 \ex\label{extractionVPE-rep}
  I know what John ate for lunch,  but I don't know what\ensuremath{_i} Bill did
  \sout{eat}  {\ensuremath{\gp_i}}\xspace \sout{for lunch}.
\end{exe}
\begin{exe}
 \ex\label{part1}
\oneline{%
\AxiomC{\ensuremath{[\LexEnt{\pt{\ensuremath{\greekp_1}}}{\sem{ x }}{\syncat{NP}}]^1\,\,}}
\noLine
\UnaryInfC{\Lemma}
\noLine
\UnaryInfC{\MultiLine{\ptfont ate \BobsO \ensuremath{\greekp_1} \BobsO for \BobsO lunch; \\ \sem{ \trns{ate}(x)(\trns{lunch})}; \syncat{ VP}}}
\LeftLabel{①\ensuremath{\rightarrow}}
\RightLabel{\scalebox{.8}{\ensuremath{\vs}I\ensuremath{^1}}}
\UnaryInfC{\MultiLine{\ptfont λ\ensuremath{\greekp_1}.ate \BobsO \ensuremath{\greekp_1} \BobsO for \BobsO lunch; \\ \sem{ \mbox{\grey{\sem{ λx. \trns{ate}(x)(\trns{lunch}) }}} }; \syncat{ VP\vs NP}}}
\AxiomC{\ensuremath{\ThreeColHyp{\MultiLine{\ptfont \ensuremath{\greekp_2}; \\ \sem{ u}; \\\syncat{  NP}}}^2}}
\BinaryInfC{\LexEnt{\pt{ate \BobsO \ensuremath{\greekp_2} \BobsO for \BobsO lunch}}{\sem{ \trns{ate}(u)(\trns{lunch})}}{\syncat{VP}}}
\AxiomC{\MultiLine{\ptfont john; \\ \sem{ \trns{j} }; \\\syncat{  NP}}}
\BinaryInfC{\LexEnt{\pt{john \BobsO ate \BobsO \ensuremath{\greekp_2} \BobsO for \BobsO lunch}}{\sem{ \trns{ate}(u)(\trns{lunch})(\trns{j})}}{\syncat{S}}}
\RightLabel{\scalebox{.8}{\ensuremath{\vs}I\ensuremath{^2}}}
\UnaryInfC{\LexEnt{\pt{\ensuremath{\lambda} \ensuremath{\greekp_2}. john \BobsO ate \BobsO \ensuremath{\greekp_2} \BobsO for \BobsO lunch}}{\sem{ λu. \trns{ate}(u)(\trns{lunch})(\trns{j})}}{\syncat{S\vs NP}}}
\AxiomC{\MultiLine{\ptfont λ\ensuremath{\greeks}. what \BobsO \ensuremath{\greeks}(\E); \\ \sem{  λP. \trns{what}(P) }; \\\syncat{  Q\vs (S\vs NP)}}}
\BinaryInfC{\LexEnt{\pt{what \BobsO john \BobsO ate \BobsO \E \BobsO for \BobsO lunch}}{\sem{ \trns{what}(λu. \trns{ate}(u)(\trns{lunch})(\trns{j}))}}{\syncat{Q}}}
\DisplayProof}
\end{exe}

\noindent The grayed-in semantic term in (\ref{part1}) is an available predicate
with which the free variable $R\,'$ obtained in the proof line ①
can be anaphorically identified. The first part of the proof for \textit{what
Bill did} then takes the following form:

\begin{exe}
 \ex\label{extractionVPE-drv-part2}
\oneline{%
\AxiomC{\Lemma}
\UnaryInfC{\MultiLine{\ptfont λ\ensuremath{\greeks} λ\ensuremath{\greekp}. did \BobsO \ensuremath{\greeks}(\ensuremath{\greekp}); \\ \sem{ λf λx λy. f(x)(y)}; \\\syncat{  (VP\vs NP)\vs (VP\vs NP)}}}
\AxiomC{\MultiLine{\ptfont λ\ensuremath{\greekr} λ\ensuremath{\greekp}. \ensuremath{\greekr}(λ\ensuremath{\greekp_0}. \ensuremath{\greekp_0})(\ensuremath{\greekp}); \\ \sem{ λ\ensuremath{\mathscr{F}}. \ensuremath{\mathscr{F}}(λx. \trns{ate}(x)(\trns{lunch}))}; \\\syncat{  (VP\vs NP)\vs }\\ \hspace{2ex}\syncat{((VP\vs NP)\vs (VP\vs NP))}}}
\BinaryInfC{\LexEnt{\pt{\ensuremath{\lambda} \ensuremath{\greekp}.did \BobsO \ensuremath{\greekp}}}{\sem{ λx λy. \trns{ate}(x)(\trns{lunch})(y)}}{\syncat{VP\vs NP}}}
\AxiomC{\ensuremath{\ThreeColHyp{\MultiLine{\ptfont \ensuremath{\greekp_3}; \\ \sem{ v}; \\\syncat{  NP}}}^3}}
\BinaryInfC{\LexEnt{\pt{did \BobsO \ensuremath{\greekp_3}}}{\sem{ λy. \trns{ate}(v)(\trns{lunch})(y)}}{\syncat{VP}}}
\AxiomC{\MultiLine{\ptfont bill; \\ \sem{ \trns{b} }; \\\syncat{  NP}}}
\BinaryInfC{\LexEnt{\pt{bill \BobsO did \BobsO \ensuremath{\greekp_3}}}{\sem{ \trns{ate}(v)(\trns{lunch})(\trns{b})}}{\syncat{S}}}
\RightLabel{\scalebox{.8}{\ensuremath{\vs}I\ensuremath{^3}}}
\UnaryInfC{\LexEnt{\pt{\ensuremath{\lambda} \ensuremath{\greekp_3}. bill \BobsO did \BobsO \ensuremath{\greekp_3}}}{\sem{ λv. \trns{ate}(v)(\trns{lunch})(\trns{b})}}{\syncat{S\vs NP}}}
\DisplayProof}
\end{exe}

\noindent The term obtained at the last step of this proof, supplied as an
argument to the extraction operator, yields an interpretation
identical to the unellipsed embedded question \textit{what Bill ate for
lunch}. Note that the prosodic term derived in the last proof step,
\pt{\ensuremath{\lambda} \ensuremath{\greekp_3}. bill \BobsO did \BobsO \ensuremath{\greekp_3}}, is exactly what we would have obtained via
the earlier version of the generalized ellipsis operator; the
associated type would however been S/NP, and therefore ineligible to
compose with \textit{what}. Moreover, as noted above, only the
vertically-slashed version of the ellipsis operator would allow us to
derive a sentence with a non-peripheral `gap' as in (\ref{internal}).
But the larger point is that long-distance dependencies into what
appear to be ellipsis contexts are, on this analysis, based on what is
in effect the extraction of a pseudogapping remnant. For example,
a proof along the lines of that began along the lines of
(\ref{extractionVPE-drv-part2}) might have continued as in (\ref{other}):

\begin{exe}
 \ex\label{other}
\oneline{%
\AxiomC{\Lemma}
\UnaryInfC{\MultiLine{\ptfont λ\ensuremath{\greeks} λ\ensuremath{\greekp}. did \BobsO \ensuremath{\greeks}(\ensuremath{\greekp}); \\ \sem{ λf λx λy. f(x)(y)}; \\\syncat{  (VP\vs NP)\vs (VP\vs NP)}}}
\AxiomC{\MultiLine{\ptfont λ\ensuremath{\greekr} λ\ensuremath{\greekp}. \ensuremath{\greekr}(λ\ensuremath{\greekp_0}. \ensuremath{\greekp_0})(\ensuremath{\greekp}); \\ \sem{ λ\ensuremath{\mathscr{F}}. \ensuremath{\mathscr{F}}(λx. \trns{ate}(x)(\trns{lunch}))}; \\\syncat{  (VP\vs NP)\vs }\\ \hspace{2ex}\syncat{((VP\vs NP)\vs (VP\vs NP))}}}
\BinaryInfC{\LexEnt{\pt{\ensuremath{\lambda} \ensuremath{\greekp}.did \BobsO \ensuremath{\greekp}}}{\sem{ λx λy. \trns{ate}(x)(\trns{lunch})(y)}}{\syncat{VP\vs NP}}}
\AxiomC{\LexEnt{\pt{breakfast}}{\sem{ \trns{brkfst} }}{\syncat{NP}}}
\BinaryInfC{\LexEnt{\pt{did \BobsO breakfast}}{\sem{ λy. \trns{ate}(x)(\trns{brkfst})(y)}}{\syncat{VP}}}
\AxiomC{\LexEnt{\pt{he}}{\sem{ 3masc }}{\syncat{S}}}
\BinaryInfC{\LexEnt{\pt{he \BobsO did \BobsO breakfast}}{\sem{ \trns{ate}(\trns{brkfst})(3masc)}}{\syncat{S}}}
\DisplayProof}
\end{exe}
This would then be an ordinary instance of pseudogapping as in \textit{John
ate lunch much faster than he did breakfast}. The upshot is that
apparent extraction from ellipsis sites as in (\ref{extrac-from-VPE}) is nothing other than
the interaction of Muskens-style \textit{wh} operators with the object of a
transitive auxiliary---a possibility that we would predict in advance
on the analysis given above.

The reader might suppose that the possibility of this kind of
extraction depends on some kind of parallel interpretation between the
antecedent and the ellipsed clauses in (\ref{extrac-from-VPE}), based on the
extraction already visible in the former. But we also have examples
where the there is no extraction in the antecedent, such as (\ref{internal}c) and (\ref{noparallel}):

\begin{exe}
 \ex\label{noparallel}
  John is certain he would buy \textsl{this} kind of sports car, but I have no
  idea what kind \textsl{I} would.
\end{exe}
To obtain such examples, we derive the antecedent by a derivation
which includes the following subproof.

\begin{exe}
 \ex\label{subproof}
%\hspace{-8ex}
\oneline{%
\AxiomC{\LexEnt{\pt{buy}}{\sem{ \trns{buy} }}{\syncat{VP/NP}}}
\AxiomC{\LexEnt{\pt{\ensuremath{\greekp_1}}}{\sem{ x}}{\syncat{NP}}}
\BinaryInfC{\MultiLine{\ptfont buy \BobsO \ensuremath{\greekp_1}; \\ \sem{ \trns{buy}(x)}; \\\syncat{  VP}}}
\UnaryInfC{\MultiLine{\ptfont λ\ensuremath{\greekp_1}. buy \BobsO \ensuremath{\greekp_1}; \\ \sem{ \mbox{\grey{\sem{ λx. \trns{buy}(x) }}} }; \\\syncat{  VP\vs NP}}}
\AxiomC{\Lemma}
\UnaryInfC{\MultiLine{\ptfont this \BobsO kind \BobsO of \BobsO sports \BobsO car; \\ \sem{ \iota(\trns{kind}(\trns{spcr}))}; \\\syncat{  NP}}}
\BinaryInfC{\MultiLine{\ptfont buy \BobsO this \BobsO kind \BobsO of \BobsO sports \BobsO car; \\ \sem{ \trns{buy}(\iota(\trns{kind}(\trns{spcr})))}; \\\syncat{  VP}}}
\AxiomC{\MultiLine{\ptfont would; \\ \sem{ λP λy.\textrm{\textbf{W}}\scalebox{.75}{\textrm{\textbf{D}}}\, P(y)}; \\\syncat{  VP/VP}}}
\BinaryInfC{\MultiLine{\ptfont would \BobsO buy \BobsO this \BobsO kind \BobsO of \BobsO sports \BobsO car; \\ \sem{ λy.\textrm{\textbf{W}}\scalebox{.75}{\textrm{\textbf{D}}}\,\trns{buy}(\iota(\trns{kind}(\trns{spcr})))(y)}; \\\syncat{  VP}}}
\DisplayProof}
\end{exe}
From this point on, the proof for the ellipsed clause would proceed in
exactly the same fashion as in the derivation of (\ref{extractionVPE-rep}),
with the free variable \sem{ P\,' } instantiated as the grayed-in term
in (\ref{subproof}).

The above (re)analysis of `extraction out of an elided VP' as
extraction of a pseudogapping remnant gives us, in effect, a
proof-of-concept argument for rejecting the assumption that covert
structures in VP ellipsis necessarily exist in order that a `site of
origin' be available for filler/gap linkages that appear to implicate
material missing from deleted VPs.\footnote{This approach has been
challenged in \citet{johnson2001}, on the graounds that apparent extraction
from ellipsis sites is subject to different contraints from
pseudogapping, counterexamples to his claims are already familiar
from, inter alia, examples from corpora or naturally occurring data
presented in \citet{levin-diss}. For detailed discusson of this point, see
\citet[Section~8.4.2]{kubotalevineBook}.} There is, on the analysis
presented in this section, no extraction from a subsequently deleted
(or phonologically supressed) subpart of some structural arrangement
of linguistic expressions, as in a phrase structure tree. Rather, an
auxiliary is licensed whose type and semantics correspond to a VP
missing an NP, and which composes by hypothetical reasoning to the
type of a clause missing an NP. A \textit{wh} operator along the lines
proposed by Muskens can then take this clause as an argument. The
appearance in (\ref{extrac-from-VPE}) of an extraction from a subsequently
ellipsed constituent is, on this view, a illusion due to the
string-identity of a VP ellipsis on the one hand and displacement of a
pseudogapping remnant on the other.

\section{Conclusion: Peirce' linguistics, logic, and mathematics and the sources of type logical grammar}

It is important at this point to consider how the results reported
above have been achieved. Fundamentally, treatment of syntactic
categories \textsl{as} valence specifications means that grammatical
rules and operations can map the combinatorial possibilites of signs
to different possibilities without ever requiring those possibilities
to be realized as actual structures e.g., the operators for auxiliary
type-shifting given above. But just as basic to this kind of solution
is the fact that in type-logical systems, the `categories' of phrase
structure grammar are replaced by types which specify the argument
requirements of their own arguments. The \textit{what} operator discussed
above can apply to a sign typed \syncat{S\vs NP}, an object itself seeking an NP
to yield a clause of arbitrary depth. Since on the analysis in
\sectref{subsubsec:pseudoextrac} the auxiliary \textit{did} in (\ref{extrac-from-VPE}) is a VP
missing an NP and thus, by hypothetical reasoning, \textit{Bill did} is an S
missing an NP, a \textit{wh} operator such as \textit{what} can take the latter as
an argument without there ever having been any material in its
licensing corresponding to the transitive verb \textit{eat} per
(\ref{extractionVPE-rep}). The interpretation of (\ref{extrac-from-VPE}) involves
the sign \textit{eat} only in the antecedent; in the ellipsis clause, the
predicate \textbf{eat} is understood in the meaning only as a result of
anaphoric retrieval from the antecedent clause. The heavy lifting in
this proposal is carried out entirely by valence-shifting operators
and the treatment of extraction as just one more instance of a
dependency mediated by valence satisfaction.

The possibilities of this kind of framework depend on a \textsc{residuated}
logic, i.e., a logic in which the connectives, viewed as
type-constructors, have the property that, in the notation of
classical implication (but necessarily modulo the directionality of the
type-constructor slashes), and with $\Lleftarrow\!\Rrightarrow$
denoting metalogical equivalence:

\begin{exe}
 \ex\label{residuation}
  \sem{ (\psi  \ensuremath{\vdash\xspace } \psi\supset\varrho)\Lleftarrow\!\Rrightarrow (\psi,\phi  \ensuremath{\vdash\xspace } \varrho)\Lleftarrow\!\Rrightarrow (\phi  \ensuremath{\vdash\xspace } \psi\supset\varrho) }
\end{exe}
(For detailed discussion, see \citet{sep-logic-substructural}). Residuation
is a property of the type-constructors /,$\bsl$ introduced in
\citet{lambek58}, for all practical purposes the founding document of
contemporary type-logical formalisms, and so far as type-logic is
concerned, can be understood in the following way: there is a natural
relationship between the entailment/equivalence relations in
(\ref{residuation}), whereby if inhabiting a given type $\tau_1$ entails
inhabiting some other type $\tau_2$, then \sem{ \tau_1  \ensuremath{\vdash\xspace } \tau_2 },
i.e., \sem{   \ensuremath{\vdash\xspace } \tau_1 \ensuremath{ \rightarrow } \tau_2 }. Suppose that, given two types $A,B$,
we can compose each member of $A$ with each member of $B$ to yield a
term belonging to type $C$, i.e., \sem{ A\bullet B\,  \ensuremath{\vdash\xspace } \,\textit{C}  }. Then
necessarily every member of $A$ belongs to the set of terms which form
a member of $C$ when they compose with a member of $B$ on the right;
if we call this set \sem{ B/C } then
\sem{ A{}  \ensuremath{\vdash\xspace } B/C}, and likewise for
\textit{B}. We thus have the relations

\begin{exe}
 \ex\label{typeResiduation}
   $ (A \bullet B\,\vdash\, C)\scalebox{.8}{$\Lleftarrow\!\Rrightarrow$}(A \vdash C/B)
   \Lleftarrow\!\Rrightarrow(\B \vdash B \vdash A\bsl C)$
\end{exe}

%\vspace{10mm}
\itdopt{Is the first left arrow/right arrow-combination really supposed to be smaller? And shouldn't there be a
  symbol for double arrows? \backslash B is missing in the PDF and here. vdash does not print.}

\noindent
(\ref{typeResiduation}) is nothing more than the residuated implication
relationship of standard logic displayed in residuation.  But as
discussed at length in \citet{pratt92}, Peirce himself developed a theory of
binary relations that incorporated the key components of residuated
relationships between terms, including a kind of proto-version of the
left and right `division' relations that, per (\ref{typeResiduation}), are
formally entailed by each of the arguments of the type composition
operator $\bullet$ (and which are in essence the upper adjoints of the
monotone Galois connection which frames residuation in terms of
partial orderings).\footnote{Specifically, assume that for any two
types $A,B$, $A\bullet B \leqslant C$, i.e., every inhabitant of the
concatenation of the types $A, B$ is an inhabitant of $C$. Then with
$f_\ast = \L\alpha. \alpha\bullet B$ and $f^\ast = \L \beta. \beta/B$,
there is a Galois connection between $f^\ast$ and $f^\ast \textrm{\,
iff\,} f_\ast(A) \leqslant C \Leftrightarrow A\leqslant f^\ast(C)$,
which, if we also define an upper adjoint $f^\ast\ast = \L
\gamma. A\bsl\gamma$, and take the entailement relation $X\vdash
Y$ to define a partial ordering $X\leqslant Y$, gives us exactly the
`triquivalence' in (\ref{typeResiduation}).} As Pratt notes, the upper and
lower adjoint operators are effectively the functions corresponding to
the composition and division connectives (which Peirce wrote with a
semicon and a horizonal-line franction notation respectively).

It seems fair to say, then, that---to extend Peirce's original
chemical metaphor only slightly---we can plausibly view Lambek's
seminal work in his 1958 and 1961 papers as the reaction product of an
imagined catalyst bonding Peirce's ideas about valence as the basis of
linguistic combinatorics to his work on the algebra of relations. Any
doubt about the correctness of such a view should be immediately
dispelled by Lambek's own words; in one of his papers on pregroup
grammars---an algebraic reformulation of type-logical grammar he
developed in order to make transparent the logical foundation of his
earlier systems as instances of (a fragment of) intuitionistic
noncommutative linear logic---he comments of a very basic skeleton for
the pregroup grammar formalism that the essential combinatorics `may
even be implicit in the ideas of C.S. Peirce [i.e. \citet{peirce1897}
---RDL], noting that certain combinators in this `rudimentary' version
may have been seen by Peirce as comparable to `the unsaturated bonds
of an atom. \textsl{I believe pregroup grammars developed from this
rudimentary setup}.' (\citealt[352]{lambek2007a}; emphasis added).

The system exhibited in (\ref{rules}) combines Lambek's earliest formulation
of a type-based logic for linguistic composition with the version of
type-logic developed in \citet{oehrle1994}; but note that Oehrle's system is
presented as itself an outcome of enriching the associated type-logic
of Lambek's 1958 paper with the structural rule of permutation; this
of course then requires word order to be somehow separated from type
combinatorics, and Oehrle's own deep insight was to allow the prosody
to contain functional operators. It is not unreasonable to see
Lambek's 1958 paper as the fountainhead for the two separate research
traditions that have developed under the broad heading of type-logical
grammar, and as I hope to have made clear, Peirce's work in both
linguistics and the algebra of relations had already provided the
materials for Lambek's profound synthesis, as Lambek himself stressed.
It is to be hoped that future overviews of the history of type-logical
systems along the lines of e.g. \citet{moortgat2010} will take due note of Peirce's right to
ancestral status in the lineage of the Lambek calculus, and therefore
of all contemporary versions of type-logical grammar. And it strikes
me as extremely likely that Peirce would have been
particularly glad had he known the degree to which his key
linguistic principles---valence satisfaction as the driver of
grammatical composition and language as an extension of logic---would
be unified so precisely and rigorously in Lambek's brilliant fusion
of developments in logic and mathematics that can be traced, to a large
extent, back to Peirce himself.




\section*{Acknowledgements}
The analyses and results
presented in this chapter represent joint work with Yusuke Kubota, with whom I
have collaborated on research on the theory and empirical application of
categorial grammar for a decade and a half. My thanks to Andras Kornai
for carefully reading and correcting an earlier draft of this paper
and providing me with valuable feedback pointing me to parts of
the presentation that I needed to compile out a bit more clearly.

{\sloppy\printbibliography[heading=subbibliography,notkeyword=this]}

\end{document}

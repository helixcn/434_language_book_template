\addchap{Preface}
\begin{refsection}

It is my pleasure to introduce this Festschrift in honor of Dan Everett. I've known Dan since 1987, when he interviewed for a job at the University of Pittsburgh in the Department of Linguistics. I was a second year grad student in a joint program in computational linguistics between Carnegie Mellon and the University of Pittsburgh, and Dan was interviewing for a syntax and morphology position at Pitt. Dan was striking for his fascinating material on several Native American languages including Pirahã, Wari', and Yagua. He was also striking for his attire: He wore old blue jeans and a red t-shirt with a picture of a parrot on it.  The t-shirt was also noticeably worn out with holes in it. He was a \textit{cool} academic. Dan got the job. 

After knowing him for a while, I asked Dan to be on my thesis committee because of his breadth of knowledge of language and linguistics, and his friendly manner. He ended up being my co-advisor for my PhD thesis. He was an enormous help to me for his advice on the work that I did in my PhD at CMU. Dan of course was working with the Pirahã at the time, and he invited me then to work with the Pirahã, in the 80s and the 90s, but I wasn't able to go at the time. In 2005, he wrote the famous Current Anthropology paper, and he invited me again. This time I was able to accept. I visited in 2007 with Mike Frank, who was my then-student, and in collaboration with Ev Fedorenko, who wasn't able to come. We started some fascinating projects on number words in Pirahã and syntactic recursion in Pirahã, some of which is alluded to in this Festschrift. 

Dan grew up in extreme poverty on the U.S Mexican border. He and his mother lived happily in a trailer park, when she died suddenly of an aneurysm when at only 29 years old. Dan was only 11, and he had to go to live with his estranged father in San Diego, who Dan did not get along with. Sadly, Dan's father was abusive, and so Dan had to spend a lot of time on the streets. In 1968, Dan was 16 years old, selling drugs at a rock concert when he met the children of missionaries who were trying to help troubled young people like Dan. Dan got along well with the them, and was asked to join them at their house for dinner. Soon thereafter Dan met and fell in love with the missionaries' daughter Keren, who was also 16 at the time.

Dan told me that he'd never met happy people before this. The missionaries attributed their happiness to their Christian beliefs, so Dan converted. By the time Dan was 19, he and Keren were married, and had a child, Caleb (who is a now professor of anthropology at the University of Miami, and who has written a paper for this volume: Chapter~\ref{chap-13_everett}). Dan became a missionary after studying at the Moody Bible Institute in Chicago and learning some linguistics at the University of Oklahoma. He moved to Brazil in 1977 to work with the Pirahã, by which time he and Keren had three children. Along the way he got a masters in linguistics from Unicamp in Brazil, and he got the first PhD ever awarded in Linguistics in Brazil in 1983, also from Unicamp. 

If you don't know him, one of Dan's great talents is that he can learn a language extraordinarily quickly and well. He knows many, many languages and sounds native in many of them. I know that when I travel with him in Brazil, Brazilians always try to figure out where exactly he is from in Brazil. They're surprised when he says San Diego. Although he only learned Portuguese as an adult in the 1970s, he sounds native. He is also a rare linguist who can figure out the sounds, morphemes, and structure of a language, even with no bilinguals. And that's what he did with the Pirahã. He has worked on many languages, perhaps most with the Pirahã, where he started as a missionary in 1977 and spent over seven full years working and living with them, by which time he became close to bilingual, more so than anyone else ever, according to the Pirahã people. Although he went to the Pirahã to convert them to Christianity, he likes to tell the story that he never converted any of them. In fact, he'll tell you that they helped convert him away from Christianity. Dan ended up getting divorced from Karen (and Christianity) and marrying Linda, to whom he has been happily married now for 15 years. 

Dan has published in almost every area of linguistics: in phonetics, phonology, morphology, sociolinguistics, psycholinguistics, historical linguistics, syntax, semantics, philosophy of language, and philosophy of linguistics. He started as an assistant professor in Unicamp in the 1980s, then moved to the University of Pittsburgh in the late 1980s, where I met him, and he has had several other academic positions, culminating in his current position which, is a Trustee Professorship at  Bentley University. Dan has a great many academic achievements, only some of which will list here. He has done enormous descriptive work: He wrote  a grammar of Pirahã and a grammar of Wari', and he did descriptive work in around 20 languages of the Americas. He has done a lot of phonetic work documenting new sounds in Pirahã and Wari'i. He has done important morphological and language documentation work. Working with native speakers, he identified Oro Win as a distinct language in the Chapacuran family. The work I Know Dan best for is the 2005 \textit{Current Anthropology} paper, where he documented Pirahã, the first language known to lack number words, one of the simplest kinship systems ever documented, the first culture documented to lack origin myths, and possibly, most importantly -- depending on your perspective -- this language was claimed to lack syntactic recursion. This was of theoretical interest because of a paper that Hauser, Chomsky \& Fitch wrote in 2002 in the journal \textit{Science}, proposing that a core feature of human syntax (in human language) was recursion. So Dan's proposal that Pirahã lacked this feature created a bit of a stir that exists to this day.

Dan has also worked on linguistic anthropology, in a book called \textit{Dark Matter of the Mind} from Chicago Press, and his most recent work is coming out soon on the philosophy of linguistics, discussing the work of Charles Peirce.  Not only has Dan done all these theoretical and descriptive pieces of research, but he has done important expository work for the general public. He has at least three books that I know: \textit{Don't Sleep There Are Snakes}, \textit{Language: The Cultural Tool}, and \textit{How Language Began}. These books have been translated into around 20 languages around the world and they're lovely pieces of work on getting linguistics out to the general audience. 

This Festschrift includes 15 articles that are related to Dan's work over the years. It is being released after a tribute event for Dan Everett that was held at MIT on June 8th, 2023. 

~\medskip

\noindent
Cambridge, MA, \today\hfill Edward Gibson %and Moshe Poliak

{\sloppy\printbibliography[heading=subbibliography]}
\end{refsection}


\SetupAffiliations{output in groups = false,
                   orcid placement = after,
                   separator between two = {\bigskip\\},
                   separator between multiple = {\bigskip\\},
                   separator between final two = {\bigskip\\}
                   }

% ORCIDs in langsci-affiliations 
\definecolor{orcidlogocol}{cmyk}{0,0,0,1}
\RenewDocumentCommand{\LinkToORCIDinAffiliations}{ +m }
  {%
    \,\orcidlink{#1}%
  }


\makeatletter
\let\thetitle\@title
\let\theauthor\@author
\makeatother

\newcommand{\togglepaper}[1][0]{
   \bibliography{../localbibliography}
   \papernote{\scriptsize\normalfont
     \theauthor.
     \titleTemp.
     To appear in:
     E. Di Tor \& Herr Rausgeberin (ed.).
     Booktitle in localcommands.tex.
     Berlin: Language Science Press. [preliminary page numbering]
   }
   \pagenumbering{roman}
   \setcounter{chapter}{#1}
   \addtocounter{chapter}{-1}
}

\newbool{bookcompile}
\booltrue{bookcompile}
\newcommand{\bookorchapter}[2]{\ifbool{bookcompile}{#1}{#2}}


% Cite and cross-reference other chapters
\newcommand{\crossrefchaptert}[2][]{\citet*[#1]{chapters/#2}, Chapter~\ref{chap-#2} of this volume} 
\newcommand{\crossrefchapterp}[2][]{(\citealp*[#1]{chapters/#2}, Chapter~\ref{chap-#2} of this volume)}
\newcommand{\crossrefchapteralt}[2][]{\citealt*[#1]{chapters/#2}, Chapter~\ref{chap-#2} of this volume}
\newcommand{\crossrefchapteralp}[2][]{\citealp*[#1]{chapters/#2}, Chapter~\ref{chap-#2} of this volume}

\newcommand{\crossrefcitet}[2][]{\citet*[#1]{chapters/#2}} 
\newcommand{\crossrefcitep}[2][]{\citep*[#1]{chapters/#2}}
\newcommand{\crossrefcitealt}[2][]{\citealt*[#1]{chapters/#2}}
\newcommand{\crossrefcitealp}[2][]{\citealp*[#1]{chapters/#2}}


\newcommand{\sub}[1]{\textsubscript{\scriptsize\textrm{#1}}}

% Müller

\newcommand{\page}{}

\let\citew\citet

\def\underRevision{Revise and resubmit}

\let\textbfemph\emph

\newcommand{\todostefan}[1]{\todo[color=orange!80]{\footnotesize #1}\xspace}
\newcommand{\todosatz}[1]{\todo[color=red!40]{\footnotesize #1}\xspace}

\newcommand{\inlinetodostefan}[1]{\todo[color=green!40,inline]{\footnotesize #1}\xspace}

\newcommand{\inlinetodoopt}[1]{\todo[color=green!40,inline]{\footnotesize #1}\xspace}
\newcommand{\inlinetodoobl}[1]{\todo[color=red!40,inline]{\footnotesize #1}\xspace}

\newcommand{\itd}[1]{\inlinetodoobl{#1}}
\newcommand{\itdobl}[1]{\inlinetodoobl{#1}}
\newcommand{\itdopt}[1]{\inlinetodoopt{#1}}

\newcommand{\addpages}{\todostefan{add pages}}

%% % taken from https://tex.stackexchange.com/a/95079/18561
\newbox\usefulbox

\makeatletter
\def\getslant #1{\strip@pt\fontdimen1 #1}

\def\skoverline #1{\mathchoice
 {{\setbox\usefulbox=\hbox{$\m@th\displaystyle #1$}%
    \dimen@ \getslant\the\textfont\symletters \ht\usefulbox
    \divide\dimen@ \tw@ 
    \kern\dimen@ 
    \overline{\kern-\dimen@ \box\usefulbox\kern\dimen@ }\kern-\dimen@ }}
 {{\setbox\usefulbox=\hbox{$\m@th\textstyle #1$}%
    \dimen@ \getslant\the\textfont\symletters \ht\usefulbox
    \divide\dimen@ \tw@ 
    \kern\dimen@ 
    \overline{\kern-\dimen@ \box\usefulbox\kern\dimen@ }\kern-\dimen@ }}
 {{\setbox\usefulbox=\hbox{$\m@th\scriptstyle #1$}%
    \dimen@ \getslant\the\scriptfont\symletters \ht\usefulbox
    \divide\dimen@ \tw@ 
    \kern\dimen@ 
    \overline{\kern-\dimen@ \box\usefulbox\kern\dimen@ }\kern-\dimen@ }}
 {{\setbox\usefulbox=\hbox{$\m@th\scriptscriptstyle #1$}%
    \dimen@ \getslant\the\scriptscriptfont\symletters \ht\usefulbox
    \divide\dimen@ \tw@ 
    \kern\dimen@ 
    \overline{\kern-\dimen@ \box\usefulbox\kern\dimen@ }\kern-\dimen@ }}%
 {}}
\makeatother

% 1_intro.tex

% For the block quote:

\usepackage[most]{tcolorbox}
\definecolor{linequote}{RGB}{224,215,188}
\definecolor{backquote}{RGB}{249,245,233}
\newtcolorbox{myquote}[1][]{%
    enhanced, breakable, 
    size=minimal,
    frame hidden, boxrule=0pt,
    sharp corners,
    colback=backquote,
    #1
}

% 2_gibson.tex


% Example(s) Environments
% 12pt, No new-lines after example number is printed

\newcounter{examplectr}
\newcounter{fnexamplectr}

% Note: don't use subexamples in footnotes.

% This line is to overcome a bug in cmu-art style: it prints counter
% values to the aux file using \theaux... rather than using \the...
\def\theauxexamplectr{\theexamplectr}

\newcounter{subexamplectr}
\def\theauxsubexamplectr{\thesubexamplectr}
\def\theauxfnexamplectr{\thefnexamplectr}

\renewcommand{\theexamplectr}{\arabic{examplectr}}
% This command causes example numbers to appear without following periods

\renewcommand{\thefnexamplectr}{\roman{fnexamplectr}}
% This command causes example numbers to appear without following periods

\renewcommand{\thesubexamplectr}{\theexamplectr\alph{subexamplectr}}
% This command gives the number of an example and subexample as e.g. 1a, 2b

\newlength{\wdth}
\newcommand{\strike}[1]{\settowidth{\wdth}{#1}\rlap{\rule[.5ex]{\wdth}{1pt}}#1}

\newcommand{\exref}[1]{(\ref{#1})}
% This command puts reference numbers with parentheses
% surrounding them 

% The environment ``examples'' gives a list of examples, one on each line,
% numbered with a lower case alphabetic character
\newenvironment{examples}%
   { \vspace{-\baselineskip}
     \begin{list}%
     \textrm{\alph{subexamplectr}.}%
     {\usecounter{subexamplectr}
     \setlength{\topsep}{-\parskip}
     \setlength{\itemsep}{-2pt}
     \setlength{\leftmargin}{0.5in}
     \setlength{\rightmargin}{0in} } }%
   { \end{list}}

% The environment ``myexample'' outputs an arabic counter ``examplectr''
% surrounded by parentheses.
\newenvironment{myexample}
   { \vspace{20pt}
     \noindent
     \begin{minipage}{\textwidth}    % minipage environment disallows
                 % breaks across pages

     \refstepcounter{examplectr}     % step the counter and cause this
                 % section to be referenced by the
                 % counter ``examplectr''
     (\arabic{examplectr})}%
   { \vspace{20pt}
     \end{minipage}}

\newenvironment{myfnexample}
   { \vspace{2pt}
     \noindent
     \begin{minipage}{\textwidth}    % minipage environment disallows
                 % breaks across pages

     \refstepcounter{fnexamplectr}     % step the counter and cause this
                 % section to be referenced by the
                 % counter ``examplectr''
     (\roman{fnexamplectr})}%
   { \vspace{2pt}c
     \end{minipage}}
    
\newcommand*\circled[1]{\tikz[baseline=(char.base)]{
            \node[shape=circle,draw,inner sep=2pt] (char) {#1};}}



% 3_pullum.tex


% %%%  GKP:  I put in these pointless commands to kill off a bug elsewhere
% %%%        that tries to \newcommand \it (etc.) as \itshape (etc.), but
% %%%        fails because they haven't been defined.
% \newcommand{\it}{\relax}
% \newcommand{\bf}{\relax}
% \newcommand{\sc}{\relax}
% \newcommand{\rm}{\relax}

%%% GKP:  Two additional commands that I need
\newcommand{\data}[1]{\textit{#1}}
\newcommand{\blank}{\rule{1.2em}{0.5pt}}



% 8_levine

% ???
%\renewcommand{\emph}{\textit}
%\renewcommand{\em}{\it}

% not used \newcommand{\cites}[1]{\citeauthor{#1}'s~\citeyearpar{#1}}

% \renewcommand{\SetInfLen}{\setpremisesend{0pt}\setpremisesspace{10pt}\setnamespace{0pt}}

\newcommand{\pt}[1]{\ensuremath{\mathsf{#1}}}
\newcommand{\ptv}[1]{\ensuremath{\textsf{\textsl{#1}}}}

%\newcommand{\sv}[1]{\ensuremath{\bm{\mathcal{#1}}}}
\newcommand{\sv}[1]{\ensuremath{\mathcal{#1}}}

\newcommand{\sX}{\sv{X}}
\newcommand{\sF}{\sv{F}}
\newcommand{\sG}{\sv{G}}
%
% \renewcommand{\lex}{\SF}
% \renewcommand{\syncat}[1]{\ensuremath{\mathrm{#1}}}
% \newcommand{\syncatVar}[1]{\ensuremath{\mathit{#1}}}
%
% \newcommand{\RuleName}[1]{\textrm{#1}}
%
% \newcommand{\SemTyp}{\textsf{Sem}}
%
% \newcommand{\something}{\vdots\,\,\,\,\,\,\vdots}
%
% \newcommand{\pb}{\phantom{[}}
%
% \renewcommand{\E}{\ensuremath{\bm{\epsilon}}\xspace}
%
% \newcommand{\greeka}{\upalpha}
% \newcommand{\greekb}{\upbeta}
% \newcommand{\greekd}{\updelta}
\newcommand{\greekp}{\upvarphi}
\newcommand{\greekr}{\uprho}
\newcommand{\greeks}{\upsigma}
% \newcommand{\greekt}{\uptau}
% \newcommand{\greeko}{\upomega}
% \newcommand{\greekz}{\upzeta}
%
% % Do not do this!!!!!
% % \renewcommand{\labelenumi}{(\roman{enumi})}
%
% \newcommand{\upa}[2]{\ensuremath{\syncat{#1}|\syncat{#2}}}
% \newcommand{\dna}[2]{\upa{#2}{#1}}
%

%
% \newcommand{\Lemma}{\ensuremath{\vdots\hskip.5cm\vdots}\noLine}
%
% \newcommand{\la}{\ensuremath{\langle}}
% \newcommand{\ra}{\ensuremath{\rangle}}
%
% \renewcommand{\I}{\iota}
%
% \renewcommand{\sem}{\ensuremath}
%
% \newcommand{\LemmaShort}{\ensuremath{\vdots\hskip.2cm\vdots\hskip.2cm\vdots}\noLine}
% \newcommand{\LemmaShortAlt}{\ensuremath{\vdots\hskip.2cm\vdots}}
%
%
% \newcommand{\NoSem}{%
% \renewcommand{\LexEnt}[3]{##1; \syncat{##3}}
% \renewcommand{\LexEntTwoLine}[3]{\renewcommand{\arraystretch}{.8}%
% \begin{array}[b]{l} ##1;  \\ \syncat{##3} \end{array}}
% \renewcommand{\LexEntThreeLine}[3]{\renewcommand{\arraystretch}{.8}%
% \begin{array}[b]{l} ##1; \\ \syncat{##3} \end{array}}}
%
%
% \newcommand{\NoSemVar}{%
% \renewcommand{\LexEnt}[3]{##1; \syncat{##3}}
% \renewcommand{\LexEntTwoLine}[3]{\renewcommand{\arraystretch}{.8}%
% \begin{array}{l} ##1;  \\ \syncat{##3} \end{array}}
% \renewcommand{\LexEntThreeLine}[3]{\renewcommand{\arraystretch}{.8}%
% \begin{array}{l} ##1; \\ \syncat{##3} \end{array}}}
%
% \newcommand{\vs}{\raisebox{.05em}{\ensuremath{\upharpoonright}}}
%
% \newcommand{\AXX}[1]{\raisebox{-7mm}{\ensuremath{#1}}}
%
% \newcommand{\hypml}[2]{\left[\!\!#1\!\!\right]^{#2}}
%
% \newcommand{\alt}[2]{$\left\{\begin{array}{c}
% \hskip-.7ex\textrm{#1}\hskip-.7ex \\
% \hskip-.7ex\textrm{#2}\hskip-.7ex
%         \end{array}
% \right\}$}
%
% \newcommand{\altalt}[2]{\{#1/#2\}}
%
%
%
% \newcommand{\altt}[3]{$\left\{\begin{array}{c}
% \hskip-.7ex\textrm{#1}\hskip-.7ex \\
% \hskip-.7ex\textrm{#2}\hskip-.7ex \\
% \hskip-.7ex\textrm{#3}\hskip-.7ex
% \end{array}
% \right\}$}
%
% \newcommand{\alttt}[4]{$\left\{\begin{array}{c}
% \hskip-.7ex\textrm{#1}\hskip-.7ex \\
% \hskip-.7ex\textrm{#2}\hskip-.7ex \\
% \hskip-.7ex\textrm{#3}\hskip-.7ex \\
% \hskip-.7ex\textrm{#4}\hskip-.7ex
% \end{array}
% \right\}$}
%
%
% %%%%for bussproof
%
% \def\defaultHypSeparation{\hskip0.1in}
% \def\ScoreOverhang{0pt}
%
%
% %%\newcommand{\MultiLine}[1]{\renewcommand{\arraystretch}{.8}%
% %%\ensuremath{\begin{array}{l} #1 \end{array}}}
%
\newcommand{\MultiLine}[1]{\renewcommand{\arraystretch}{.8}%
\ensuremath{\begin{array}[b]{l} #1 \end{array}}}

%
%
% \newcommand{\MultiLineMod}[1]{%
% \ensuremath{\begin{array}[t]{l} #1 \end{array}}}
%
%
% %%%%%\AFourMargin
% %%\JLSubmissionMargin
%
% %%\setlength\topmargin{-1cm}
% %%\setlength\textheight{23cm}
% %%%%%\setlength\textwidth{13.5cm}
%
% %\setstretch{1.2}
%
% % not used \newcommand{\hyp}[2]{[ #]^{#2}}
%
\newcommand{\LexEnt}[3]{#1; \ensuremath{#2}; \syncat{#3}}
%
% \newcommand{\LexEntTwoLine}[3]{\renewcommand{\arraystretch}{.8}%
% \begin{array}[b]{l} #1; \\ \ensuremath{#2};  \syncat{#3} \end{array}}
%
% \newcommand{\LexEntThreeLine}[3]{\renewcommand{\arraystretch}{.8}%
% \begin{array}[b]{l} #1; \\ \ensuremath{#2}; \\ \syncat{#3} \end{array}}
%
%
% \newcommand{\LexEntFiveLine}[5]{\renewcommand{\arraystretch}{.8}%
% \begin{array}{l} #1 \\ #2; \\ \ensuremath{#3} \\ \ensuremath{#4}; \\ \syncat{#5} \end{array}}
%
%
% \newcommand{\LexEntFourLine}[4]{\renewcommand{\arraystretch}{.8}%
% \begin{array}{l} \pt{#1} \\ \pt{#2}; \\ \syncat{#4} \end{array}}
%
% \newcommand{\ManySomething}{\renewcommand{\arraystretch}{.8}%
% \raisebox{-3mm}{\begin{array}[b]{c} \vdots \,\,\,\,\,\, \vdots \\
% \vdots \,\,\,\,\,\, \vdots \end{array}}}
%
%
% \newcommand{\lemma}[1]{\renewcommand{\arraystretch}{.8}%
% \begin{array}[b]{c} \vdots \,\,\,\,\,\, \vdots \\ #1 \end{array}}
%
% \newcommand{\lemmarev}[1]{\renewcommand{\arraystretch}{.8}%
% \begin{array}[b]{c} #1 \\ \vdots \,\,\,\,\,\, \vdots \end{array}}
%
% \newcommand{\p}{\ensuremath{\upvarphi}}
%
% \newcommand{\Not}{\leavevmode\llap{\textbf{\smc{NOT:}} }}
%
% \newcommand{\Conj}{\fs{\bsp{\mathit{X}}{\mathit{X}}}{\mathit{X}}}
% \newcommand{\ConjY}{\fs{\bsp{\mathit{Y}}{\mathit{Y}}}{\mathit{Y}}}
% \newcommand{\sameLE}{\dna{(\upa{(\upa{S}{\mathit{X}})}{NP})}{(\upa{S}{\mathit{X}})}}
%
% \newcommand{\derivcenter}[2][1.1]{%
% \SetInfLen
% \attop{\vskip3ex
% \resizebox{#1\linewidth}{!}{\hskip-#1in
% #2}}}
%
% \newcommand{\derivcenterAlt}[2][.98]{%
% \SetInfLen
% \attop{\vskip3ex
% \resizebox{#1\linewidth}{!}{\hskip-#1in \hskip.5in
% #2}}\vspace{.5ex}}
%
% \renewcommand{\O}{\circ}  Do not recommand!
\newcommand{\BobsO}{\circ}
%
% \newcommand{\derivcenterMod}[2][1.1]{%
% \renewcommand{\LexEntThreeLine}[3]{\renewcommand{\arraystretch}{.8}%
% \raisebox{.4ex}{\ensuremath{\begin{array}{l} ##1; \\ \ensuremath{##2}; \\ \syncat{##3} \end{array}}}}
% \SetInfLen
% \attop{\vskip3ex
% \resizebox{#1\linewidth}{!}{\hskip-#1in
% #2}}}
%
%
% \newcommand{\shortarrow}{\xspace\hskip-1.2ex\scalebox{.5}[1]{\ensuremath{\bm{\rightarrow}}}\hskip-.5ex\xspace}
%
% \newcommand{\SemInt}[1]{\mbox{$[\![ \textrm{#1} ]\!]$}}
%
% \def\maru#1{{\ooalign{\hfil
%   \ifnum#1>999 \resizebox{.25\width}{\height}{#1}\else%
%   \ifnum#1>99 \resizebox{.33\width}{\height}{#1}\else%
%   \ifnum#1>9 \resizebox{.5\width}{\height}{#1}\else #1%
%   \fi\fi\fi%
% \/\hfil\crcr%
% \raise.167ex\hbox{\mathhexbox20D}}}}
%
\newcommand{\HypSpace}{\hskip-.8ex}
\newcommand{\RaiseHeight}{\raisebox{2.2ex}}
% \newcommand{\RaiseHeightLess}{\raisebox{1ex}}
%
% \newcommand{\fW}{\ensuremath{\mathfrak{W}}}
%
\newcommand{\ThreeColHyp}[1]{\RaiseHeight{\Bigg[}\HypSpace#1\HypSpace\RaiseHeight{\Bigg]}}
% \newcommand{\TwoColHyp}[1]{\RaiseHeightLess{\Big[}\HypSpace#1\HypSpace\RaiseHeightLess{\Big]}}
%
%
% %\newcommand{\maskref}[1]{\textsl{\textbf{[reference omitted for refereeing]}}}
% \newcommand{\maskref}[1]{#1}
%
% \newcommand{\DerivSize}{\small}
% \newcommand{\AppDerivSize}{\footnotesize}
%
% \renewcommand{\sem}{\ensuremath}


% \newcommand{\greekp}{{\color{green}π}}
% \newcommand{\greekr}{{\color{green}\textrho}}
% \newcommand{\greeks}{{\color{green}\textsigma}}
\newcommand{\ptfont}{\color{purple}}                % what does ptfont do? Where is it defined?
                                % Question
%\newcommand{\ptfont}{\ttfamily}

%\newcommand{\grey}{\color{gray}}
\newcommand{\grey}[1]{\colorbox{mycolor}{#1}}
\definecolor{mycolor}{gray}{0.8}

\newcommand{\gap}{\longrule}
\newcommand{\gp}{\gap}
\newcommand{\vs}{\raisebox{.05em}{\ensuremath{\upharpoonright}}}
% \newcommand{\sub}[1]{\textsubscript[#1]}
\newcommand{\E}{\emph}
\newcommand{\B}{\textbf}
\newcommand{\f}{{\color{green}f}}
%\newcommand{\Lemma}{{\color{pink}Lemma}}
\newcommand{\Lemma}{\ensuremath{\vdots\hskip.5cm\vdots}\noLine}

%\newcommand{\calP}{{\color{pink}calP}} % Sebastian
\newcommand{\calP}{\ensuremath{\mathcal{P}}}


\newcommand{\maru}[1]{\ooalign{\hfil#1\/\hfil\crcr
      \raise.05ex\hbox{\LARGE\mathhexbox20D}}}


%\newcommand{\sem}[2][M\!,g]{\mbox{$[\![ \mathrm{#2} ]\!]^{#1}$}}
\newcommand{\sem}{\ensuremath}

%
\newcommand{\trns}[1]{\textbf{#1}\xspace}

\newcommand{\bs}{{\textbackslash}}
\newcommand{\bsl}{{\bs}}


\newcommand{\fb}[1]{\textsubscript{#1}}

\newcommand{\syncat}[1]{\ensuremath{\mathrm{#1}}}
\newcommand{\term}[1]{\textit{#1}}
\newcommand{\LemmaAlt}{\ensuremath{\vdots\hskip.5cm\vdots}}

%\renewcommand{\O}{ø}
